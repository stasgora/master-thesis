\newcommand\img[5][]{
	\begin{figure}[H]
		\begin{center}
			\includegraphics[width=#5\textwidth]{#2}
			\bigskip
			\caption[#3]{#3 #1}
			\label{fig:#4}
		\end{center}
	\end{figure}
} 

\newcommand\drow[1]{\multirow{2}{*}{#1}}
\newcolumntype{x}[1]{>{\centering\arraybackslash\hspace{0pt}}m{#1}}
\newcommand\centertable[4]{
	\begin{table}[H]
		\begin{center}
			\bgroup
			\def\arraystretch{1.2}
			\begin{tabular}{ #2 }
				#1
			\end{tabular}
			\egroup
		\end{center}
		\caption{#3}
		\label{tab:#4}
	\end{table}
}

\newcommand\codeinline[1]{\lstinline[basicstyle=\normalsize\ttfamily]{#1}}

\makeatletter
\newcommand{\sectionauthor}[1]{
{\vspace*{-5pt}\hspace*{4pt}\footnotesize\itshape\MakeUppercase{#1}
\par\nobreak\vspace*{10pt}}
\@afterheading}
\makeatother

\newcommand\blankpage{%
    \null
    \thispagestyle{empty}%
    \addtocounter{page}{-1}%
    \newpage}

\newcounter{appendices}
\setcounter{appendices}{0}

\newcommand\annex[2]{
	\cleardoublepage
	\phantomsection
	\refstepcounter{appendices}
	\addcontentsline{toc}{chapter}{Załącznik nr \theappendices: #1}
	\section*{Załącznik nr \theappendices: #1}
	\label{anx:#2}
}

\newcommand\annexref[2][]{\hyperref[anx:#2]{Załącznik#1~nr~\ref{anx:#2}}}

\newcommand\ang[1]{(ang. {\it #1})}
