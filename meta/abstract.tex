\chapter*{Abstract}
Steadily increasing use of computers, programs, websites and mobile applications creates a demand for tools and solutions that monitor their operation and analyze user behavior. Those tools support developers in improving the quality of their products and help customers achieve their goals. At the same time, the price of electronic components such as sensors is rapidly falling, while the available computing power is increasing. Those factors have enabled the creation of a new category of products that interpret and use the signals emitted by sensors.

This paper summarizes the current state of development of a wide range of monitoring solutions created by both academia and commercial sectors. It includes examples of frameworks, device and user behavior monitoring tools and mobile sensor applications. Both their form and intended use varies considerably, even in the limited scope of chosen examples. Systematic literature review of scientific publications was conducted to assess the progress in research on the use of sensors to detect human activities. Except for a handful of original publications, most of the analyzed research papers approach a common problem of detecting the same set of basic activities attempting to achieve an increase in accuracy.

The second part of this document focuses on the issue of creating heat maps from user interactions with mobile applications. The process of creating an original solution for that purpose is described in detail, starting with the design and architecture, going through the implementation and encountered challenges and finishing with tests and release. The final part of this paper describes its evaluation process. A mobile application outfitted with the created tool was distributed to the group of testers. Collected data was processed into heat maps which were then used to identify issues with the application user interface, come up with good design practices and better understand other potential uses of such tool. Conclusions that were based on the insight gained into the application usage validate the usefulness of using heat maps to visualize user interactions. \\

\noindent\textbf{Keywords:} monitoring, activities, analysis, users, behavior, sensors, applications, heat maps, interactions, visualization, interface, accessibility, processing, evaluation
