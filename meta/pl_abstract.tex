\chapter*{Streszczenie}
Wraz~ze~stale postępującym wzrostem użycia komputerów, programów, serwisów internetowych~i~aplikacji mobilnych zwiększa się zapotrzebowanie~na~narzędzia~i~rozwiązania pomagające monitorować ich działanie oraz analizować działania~i~zachowanie użytkowników~w~celu zwiększania ich zadowolenia~i~poprawy jakości używanych~przez~nich produktów. Jednocześnie postępująca miniaturyzacja~i~spadek ceny komponentów elektronicznych takich jak czujniki, oraz stały wzrost dostępnej mocy obliczeniowej umożliwiły powstanie nowej kategorii rozwiązań które interpretują sygnały dostępnych~w~urządzeniach sensorów. 

Niniejsza praca stanowi podsumowanie aktualnego stanu rozwoju szeroko pojętych rozwiązań monitorujących powstających zarówno~w~środowiskach akademickich jak~i~sektorze komercyjnym. Omówione~i~porównane zostały przykłady platform~i~aplikacji monitorujących urządzenia, zachowania użytkowników oraz wykorzystujących~do~działania różnego rodzaju czujniki. Nawet przy~tak~ograniczonym przekroju istniejących rozwiązań monitorujących widoczna jest ich różnorodność wyróżniająca się zarówno~w~formie jak~i~zastosowaniu. Przeprowadzony przegląd literatury naukowej ocenia postępy~w~pracach badawczych~nad~wykorzystaniem czujników~do~wykrywania aktywności ludzkich. Z~wyjątkiem paru publikacji zajmujących się oryginalnymi tematami większość~z~przeanalizowanych prac naukowych stanowi próbę ponownego podejścia~do~problemu wykrywania tego samego zestawu podstawowych aktywności~ze~zwiększoną dokładnością. 

Druga część dokumentu skupia się~na~zagadnieniu tworzenia map cieplnych~z~interakcji użytkowników~z~interfejsami aplikacji mobilnych. Szczegółowo opisany został proces tworzenia oryginalnego rozwiązania tego typu, rozpoczynając~od~projektu~i~architektury, przechodząc~przez~implementację~i~napotkane~w~jej trakcie wyzwania, kończąc~na~testach~i~publikacji. Końcowa część pracy jest poświęcona ewaluacji stworzonego rozwiązania,~w~ramach której odpowiednio przygotowana aplikacja została rozprowadzona~w~grupie testujących~ją~użytkowników. Zebrane dane~o~ich działaniach,~po~przetworzeniu~na~mapy cieplne, posłużyły~do~identyfikacji problemów~z~obsługą interfejsu aplikacji, identyfikacji dobrych praktyk jego projektowania oraz lepszego zrozumienia innych potencjalnych zastosowań tego typu rozwiązania. Wnioski których wyciągnięcie było możliwe dzięki uzyskanemu wglądowi~w~korzystanie~z~aplikacji~z~punktu widzenia użytkownika jednoznacznie wskazują~na~przydatność zastosowania map cieplnych~do~wizualizacji interakcji. \\

\noindent\textbf{Słowa kluczowe:} monitorowanie, aktywności, analiza, użytkownicy, zachowania, czujniki, aplikacje, mapy cieplne, interakcje, wizualizacja, interfejs, dostępność, przetwarzanie, ewaluacja \\

\noindent\textbf{Dziedzina nauki~i~techniki, zgodnie~z~wymogami OECD:} Nauki inżynierskie~i~techniczne, Elektrotechnika, elektronika, inżynieria informatyczna, Sprzęt komputerowy~i~architektura komputerów
