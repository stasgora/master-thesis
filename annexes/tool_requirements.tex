\annex{Wymagania narzędzia monitorującego}{tool_requirements}

\section*{Wymagania funkcjonalne}
\newcommand\tabreq[3]{\ref{req:#1} & #2 & #3 \\\midrule}
\newcommand\reqheader[2]{\item \label{req:#1} {\it #2}:}

Poniżej znajduje się spis zidentyfikowanych dla tworzonego narzędzia wymagań funkcjonalnych. Priorytety zostały wyrażone w następującej skali: {\bf 3} - {\it największy} do {\bf 1} - {\it najmniejszy}.

\centertable{\toprule
	\textbf{ID} & \textbf{Nazwa} & \textbf{Priorytet} \\\toprule 
	\tabreq{screen_touch}{Rejestrowanie dotknięć ekranu}{3}
	\tabreq{heat_map}{Tworzenie map cieplnych}{3}
	\tabreq{per_screen}{Podział danych na ekrany aplikacji}{3}
	\tabreq{time_periods}{Podział danych na okresy czasowe}{3}
	\tabreq{scrolling}{Wsparcie dla przewijanej treści}{3}
	\tabreq{numer_data}{Eksport nieprzetworzonych danych}{2}
	\tabreq{session_config}{Konfiguracja parametrów zbieranych danych}{2}
	\tabreq{store_config}{Wybór sposobu składowania danych}{2}
	\tabreq{event_grouping}{Grupowanie interakcji ze względu na współrzędne}{2}
	\tabreq{screen_selection}{Wybór nagrywanych ekranów aplikacji}{1}
	\tabreq{on_off}{Możliwość wyłączenia działania narzędzia}{1}
	\tabreq{dynamic_config}{Opcja zmiany konfiguracji w trakcie działania}{1}
	\tabreq{ui_size}{Możliwość dostosowania rozmiaru interfejsu}{1}
}{l l r}{Wykaz wymagań funkcjonalnych narzędzia}{tool_requirements}

\begin{enumerate}[label=\textbf{F.\arabic*}]
	\reqheader{screen_touch}{Rejestrowanie dotknięć ekranu} Jako postawowa forma interakcji z urządzeniem mobilnym dotknięcia ekranu stanowią główne źródło danych do przetwarzania przez narzędzie.
	\reqheader{heat_map}{Tworzenie map cieplnych} Podstawowym celem narzędzia jest przetwarzanie interakcji użytkownika z aplikacją na ich graficzną reprezentację w formie mapy cieplnej. 
	\reqheader{per_screen}{Podział danych na ekrany aplikacji} Interakcje dotyczą konkretnego elementu interfejsu. Zbierane dane muszą być przetwarzane w kontekście ekranu aplikacji którego dotyczą. Mapy cieplne \ref{req:heat_map} powinny być nałożona na zdjęcie odpowiedniego ekranu w celu łatwej identyfikacji elementów interfejsu z którymi zachodzi interakcja.
	\reqheader{time_periods}{Podział danych na okresy czasowe} Dane zbierane w ramach każdego ekranu muszą zostać podzielone na rozsądne grupy na podstawie których będą tworzone pojedyncze mapy cieplne. Podział ten powinien jednocześnie nie być zbyt drobny, aby zapewnić optymalną liczbę punktów do przetwarzania, oraz w miarę możliwości zgadzać się z sesjami użycia aplikacji rozdzielonymi okresami bezczynności.
	\reqheader{scrolling}{Wsparcie dla przewijanej treści} Narzędzie powinno identyfikować obszary interfejsu które mogą być przewijane i brać je pod uwagę przy przetwarzaniu danych. W przeciwnym przypadku interakcje dotyczące przewijanych elementów będą wizualizowane w nieprawidłowych miejscach.
	\reqheader{numer_data}{Eksport nieprzetworzonych danych} Oprócz reprezentacji graficznej narzędzie powinno oferować możliwość zapisywania surowych danych o współrzędnych i czasie pojedynczych interakcji. Pozwoli to na ich swobodne przetworzenie według potrzeby twórcy aplikacji używającego narzędzia.
	\reqheader{session_config}{Konfiguracja parametrów zbieranych danych} Parametry umożliwiające skonfigurowanie procesu zbierania danych takie jak minimalna ilość wykonanych dotknięć ekranu oraz maksymalna długość bezczynności przed zakończeniem rejestrowania interakcji.
	\reqheader{store_config}{Wybór sposobu składowania danych} Aby dane zbierane przez narzędzie mogły być przeanalizowane muszą zostać wysłane i zapisane. Narzędzie musi oferować mechanizm przekazywania przetworzonych przez siebie danych w celu ich zapisania.
	\reqheader{event_grouping}{Grupowanie interakcji ze względu na współrzędne} Blisko położone siebie dotknięcia ekranu najczęściej dotyczą tej samej akcji. Narzędzie powinno grupować takie interakcje zaznaczając ich intensywność aby niepotrzebnie nie zaszumiać przekazu graficznego.
	\reqheader{screen_selection}{Wybór nagrywanych ekranów aplikacji} Możliwość selekcji podzbioru ekranów na których interakcje będą rejestrowane i przetwarzane w celu ograniczenia ilości generowanych danych przy rozbudowanych aplikacjach.
	\reqheader{on_off}{Możliwość wyłączenia działania narzędzia} Użytkownicy końcowi zawsze powinni mieć wybór czy chcą uczestniczyć w zbieraniu jakichkolwiek danych. Dodatkowo pojawia się potrzeba ręcznej kontroli nad pracę narzędzia w przypadku przeprowadzania zorganizowanych testów. Narzędzie musi zatem oferować możliwość łatwego włączenia i wyłączenia.
	\reqheader{dynamic_config}{Opcja zmiany konfiguracji w trakcie działania} W środowisku produkcyjnym aplikacje często działają w tle przez długi czas. Z tego powodu ważne jest zapewnienie możliwości zmiany konfiguracji podczas działania aplikacji.
	\reqheader{ui_size}{Możliwość dostosowania rozmiaru interfejsu} Zależnie od używanej przez aplikację wielkości elementów interfejsu rozmiar tworzonej mapy cieplnej \ref{req:heat_map} oraz grupowany obszar interakcji \ref{req:event_grouping} powinno dać się dostosować. 
\end{enumerate}
