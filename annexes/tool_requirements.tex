\annex{Wymagania narzędzia do analizy interakcji z aplikacją}
(dwa akapity do ew. przeniesienia do tekstu pracy)
\section*{Opis}
Tworzone narzędzie ma na celu umożliwienie wygodnej analizy wzorców interakcji użytkowników z aplikacją mobilną poprzez ich rejestrację, przetworzenie i wysłanie do centralnej lokalizacji. Ten mechanizm umożliwia twórcom łatwe przeglądanie i analizę zapisanych interakcji pod kątem identyfikacji problemów z interfejsem użytkownika. W szczególności narzędzie będzie tworzyło mapę cieplną dotknięć ekranu na danym ekranie aplikacji nałożoną na obraz tego ekranu. 

\section*{Korzyści}
Narzędzie pozwala na automatyczne zbieranie cennych danych których pozyskanie w innym przypadku wymagałyby zorganizowania czasochłonnych testów. Dzięki intuicyjnej, graficznej reprezentacji interakcji połączonej z dobrą znajomością interfejsu jego twórca jest w stanie znacznie szybciej diagnozować problemy które mają użytkownicy w trakcie używania aplikacji. Dla porównania dane zebrane z testów (zwykle nagrania telefonu) są zazwyczaj cięższe w analizie z powodu częściowego zasłonięcia ekranu ręką i konieczności domyślania się dokładnych miejsc dotknięć ekranu. 

\section*{Wymagania funkcjonalne}
\newcommand\tabreq[3]{\ref{req:#1} & #2 & #3 \\\hline}
\newcommand\reqheader[2]{\item \label{req:#1} {\it #2}:}

\centertable{
	\hline \textbf{ID} & \textbf{Nazwa} & \textbf{Priorytet} \\\hline 
	\tabreq{heat_map}{Tworzenie mapy cieplnej dotknięć ekranu}{x}
	\tabreq{numer_data}{Eksport nieprzetworzonych danych}{x}
	\tabreq{session_config}{Konfiguracja parametrów zbieranych danych}{x}
	\tabreq{dynamic_config}{Możliwość zmiany konfiguracji w trakcie działania}{x}
	\tabreq{store_config}{Wybór sposobu składowania danych}{x}
	\tabreq{on_off}{Możliwość wyłączenia działania}{x}
}{| c | l | c|}{Wykaz wymagań funkcjonalnych narzędzia}{tool_requirements}

\begin{enumerate}[label=\textbf{F.\arabic*}]
	\reqheader{heat_map}{Tworzenie mapy cieplnej dotknięć ekranu} Głównym celem narzędzia jest przetwarzanie interakcji użytkownika z aplikacją na ich graficzną reprezentację w formie mapy cieplnej dotknięć ekranu. Powinna być ona nałożona na zdjęcie ekranu aplikacji którego dotyczy w celu łatwej identyfikacji elementów interfejsu z którymi zachodzi interakcja.
	\reqheader{numer_data}{Eksport nieprzetworzonych danych} Oprócz reprezentacji graficznej narzędzie powinno oferować możliwość zapisywania surowych danych o współrzędnych i czasie pojedynczych dotknięć ekranu. Pozwoli to na ich swobodne przetworzenie według potrzeby twórcy aplikacji używającego narzędzia.
	\reqheader{session_config}{Konfiguracja parametrów zbieranych danych} Parametry umożliwiające skonfigurowanie procesu zbierania danych takie jak minimalna ilość wykonanych dotknięć ekranu oraz maksymalna długość bezczynności przed zakończeniem nagrywania.
	\reqheader{dynamic_config}{Możliwość zmiany konfiguracji w trakcie działania} W środowisku produkcyjnym aplikacje często działają w tle przez długi czas. Z tego powodu ważne jest zapewnienie możliwości zmiany konfiguracji podczas jego działania.
	\reqheader{store_config}{Wybór sposobu składowania danych} Aby dane zbierane przez narzędzie mogły być przeanalizowane muszą zostać wysłane i zapisane. Narzędzie musi oferować mechanizm przekazywania przetworzonych przez siebie danych w celu ich zapisania.
	\reqheader{on_off}{Możliwość wyłączenia działania} Narzędzie służy do testów jednak działać będzie w środowisku docelowym. Pojawia się zatem potrzeba możliwości włączenia i wyłączenia jego działania.
\end{enumerate}
