\section{Decyzje projektowe}

\paragraph{Wymagania} Przed rozpoczęciem implementacji został zdefiniowany zestaw wymagań projektowych dostępny w \annexref[u]{tool_requirements}. Określają podstawowe charakterystyki działania oraz dostępne opcje konfiguracji.

\subsection{Platforma}
Narzędzie zostało zaimplementowane dla nowej, aktywnie rozwijanej platformy Flutter. Została ona wybrana z powodu swojej innowacyjności, oferowanych możliwości oraz rosnącej popularności w śród twórców aplikacji. Dodatkowo, ponieważ platforma jest stosunkowo młoda, do tej pory nie istniało otwarte, darmowe rozwiązanie tego typu. Istniejące narzędzia oferujące podobne możliwości opierają się w dużej większości na platformie webowej.

Ponieważ narzędzie zostało pomyślane jako wtyczka \ang{plugin}, dołączana i integrowana z kodem aplikacji opartej o Flutter, do wstępnych testów i ewaluacji potrzebna była gotowa aplikacja, znajomość jej budowy, dostęp do kodu źródłowego oraz możliwość zbudowania nowej wersji. W ramach projektu grupowego odbywającego się przez pierwsze dwa semestry studiów magisterskich autor tej pracy kierował zespołem tworzącym aplikację Fokus, która stanowiła idealne medium do przeprowadzenia testów i ewaluacji tworzonego narzędzia.

\paragraph{Flutter} Zestaw narzędzi stworzony przez Google, służący do tworzenia aplikacji na wiele platform. Pozwala na zbudowanie aplikacji mobilnych na systemy Android i iOS, programów komputerowych działających na Windowsie, Linuxie i MacOS oraz stron internetowych używając wspólnej bazy kodu \ang{codebase}. Flutter osiąga ten efekt dzięki stworzeniu własnego, niezależnego od platformy mechanizmu budowania drzewa interfejsu definiowanego w języku Dart. 

Zależnie od specyfiki i możliwości oferowanych przez platformę docelową drzewo interfejsu jest przetwarzane i prezentowane w oknie aplikacji w inny sposób. W przypadku budowania strony internetowej elementy są rysowane na płótnie (znacznik~\inline{<canvas>}). Pozostałe platformy oferują analogiczne rozwiązania. Dzięki użyciu wspomagania sprzętowego Flutter osiąga wydajność i płynność działania porównywalną z programami budowanymi natywnie. Dodatkowymi zaletami tego podejścia jest danie deweloperowi absolutnej kontroli nad każdym wyświetlanym na ekranie pikselem oraz identyczny wygląd interfejsu niezeleżnie od platformy.

\subsection{Specyfika platformy}
Screenshoty, widoki przewijane, wydarzenia

\subsection{Architektura}
\img{\chapterPath/RoundSpot_Architecture.png}{Diagram architektury narzędzia}{round_spot_diagram}{.9}

komponenty, diagram
