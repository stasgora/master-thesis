\section{Specyfika działania}

\subsection{Konfiguracja}
Narzędzie posiada wiele parametrów konfiguracyjnych pozwalających na dostosowanie jego działania do potrzeb twórcy aplikacji \cite{RoundSpot_Config_Docs}:
\begin{itemize}
	\item {\bf enabled}: Określa, czy biblioteka jest aktualnie aktywna i zbiera dane.
	\item {\bf uiElementSize}: Umożliwia dostosowanie mechanizmu grupowania interakcji i rysowania map cieplnych do średniego rozmiaru elementów interfejsu użytkownika.
	\item {\bf disabledRoutes}: Zawiera nazwy stron aplikacji, na których nie mają być tworzone mapy cieplne.
	\item {\bf outputTypes}: Określa rodzaje danych wyjściowych które powinny być generowane. Dostępne typy to {\it mapy cieplne} oraz {\it surowe dane}.
	\item {\bf maxSessionIdleTime}: Ustawia czas bezczynności aplikacji, po którym wszystkie bieżące sesje zostaną zamknięte.
	\item {\bf minSessionEventCount}: Ustawia minimalną liczbę zdarzeń potrzebnych do zamknięcia sesji.
	\item {\bf heatMapStyle}: Określa styl generowanych map cieplnych. Dostępne wartości to {\it gładki} oraz {\it warstwowy}.
	\item {\bf heatMapTransparency}: Ustawia przezroczystość warstwy mapy cieplnej rysowanej na tle zdjęcia ekranu. 
	\item {\bf heatMapPixelRatio}: Określa rozdzielczość tworzonych obrazów map cieplnych.
\end{itemize}

Aby umożliwić prostą zdalną zmianę konfiguracji, która może być potrzebna w środowisku produkcyjnym, umożliwiony został mechanizm pozwalający na wczytanie konfiguracji z pliku json, który może być łatwo przesłany na urządzenia użytkowników. Stworzony został także schemat pliku konfiguracyjnego \ang{JSON Schema} pozwalający zweryfikować jego poprawność \cite{RoundSpot_Config_Schema}.

\begin{lstlisting}[language=json,caption={Przykładowy plik konfiguracyjny w formacie json},label=lst:rs_config_json]
{
  "enabled": true,
  "uiElementSize": 10,
  "disabledRoutes": [
    "some-unimportant-route"
  ],
  "outputTypes": [
    "graphicalRender"
  ],
  "session": {
    "minEventCount": 1,
    "maxIdleTime": 60
  },
  "heatMap": {
    "style": "smooth",
    "transparency": 0.6,
    "pixelRatio": 2
  }
}
\end{lstlisting}

\subsection{Typy sesji i sposoby ich kontroli}
Każdy \hyperref[par:rs_detectors]{detektor} umieszczony w interfejsie aplikacji odpowiada jednemu obszarowi z którego zbiera dane. Są one dzielone na okresy czasowe tworząc \hyperref[par:rs_session]{sesje} z których powstają, po ich zakończeniu, pojedyncze mapy cieplne. Istnieją trzy typy \hyperref[par:rs_session]{sesji}. Domyślnie, bez umieszczania dodatkowych \hyperref[par:rs_detectors]{detektorów}, każdy ekran aplikacji który nie został wskazany w konfiguracji jako ignorowany, jest nagrywany w celu tworzenia map cieplnych.

ekrany, przewijane, globalne

wyjście, bezczynność, wyłączenie działania

\subsection{Tworzenie map cieplnych}
dbscan, paths, gradients

\subsection{Formaty wyjściowe}
mapy - wariacje, raw

bajty + info

\section{Problemy implementacyjne}
W celu optymalizacji i przyspieszenia działania Flutter obsługuje mechanizm rysujący tylko fragmenty interfejsu które w danym momencie są wyświetlane na ekranie.

\subsection{Podejście pierwsze}
prostota i ograniczenia

Blokada interakcji podczas przewijania

\subsection{Podejście drugie}
skomplikowanie i możliwości

Składanie tła, monitorowanie pozycji
Wyścigi przy przewijaniu
Synchronizacja obrazu z pozycją
Dostęp do danych o statusie
