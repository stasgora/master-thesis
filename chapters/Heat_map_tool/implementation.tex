\section{Specyfika działania}

\subsection{Konfiguracja}
Narzędzie posiada wiele parametrów konfiguracyjnych pozwalających na dostosowanie jego działania do potrzeb twórcy aplikacji \cite{RoundSpot_Config_Docs}:
\begin{itemize}
	\item {\bf enabled}: Określa, czy biblioteka jest aktualnie aktywna i zbiera dane.
	\item {\bf uiElementSize}: Umożliwia dostosowanie mechanizmu grupowania interakcji i rysowania map cieplnych do średniego rozmiaru elementów interfejsu użytkownika.
	\item {\bf disabledRoutes}: Zawiera nazwy stron aplikacji, na których nie mają być tworzone mapy cieplne.
	\item {\bf outputTypes}: Określa rodzaje danych wyjściowych które powinny być generowane. Dostępne typy to {\it mapy cieplne} oraz {\it surowe dane}.
	\item {\bf maxSessionIdleTime}: Ustawia czas bezczynności aplikacji, po którym wszystkie bieżące sesje zostaną zamknięte.
	\item {\bf minSessionEventCount}: Ustawia minimalną liczbę zdarzeń potrzebnych do zamknięcia sesji.
	\item {\bf heatMapStyle}: Określa styl generowanych map cieplnych. Dostępne wartości to {\it gładki} oraz {\it warstwowy}.
	\item {\bf heatMapTransparency}: Ustawia przezroczystość warstwy mapy cieplnej rysowanej na tle zdjęcia ekranu. 
	\item {\bf heatMapPixelRatio}: Określa rozdzielczość tworzonych obrazów map cieplnych.
\end{itemize}

Aby umożliwić prostą zdalną zmianę konfiguracji, która może być potrzebna w środowisku produkcyjnym, umożliwiony został mechanizm pozwalający na wczytanie konfiguracji z pliku json, który może być łatwo przesłany na urządzenia użytkowników. Stworzony został także schemat pliku konfiguracyjnego \ang{JSON Schema} pozwalający zweryfikować jego poprawność \cite{RoundSpot_Config_Schema}.

\begin{lstlisting}[language=json,caption={Przykładowy plik konfiguracyjny w formacie json},label=lst:rs_config_json]
{
  "enabled": true,
  "uiElementSize": 10,
  "disabledRoutes": [
    "some-unimportant-route"
  ],
  "outputTypes": [
    "graphicalRender"
  ],
  "session": {
    "minEventCount": 1,
    "maxIdleTime": 60
  },
  "heatMap": {
    "style": "smooth",
    "transparency": 0.6,
    "pixelRatio": 2
  }
}
\end{lstlisting}

\subsection{Sesje}

\paragraph{Mapy ekranów} Każdy detektor umieszczony w interfejsie aplikacji odpowiada jednemu obszarowi z którego zbiera dane. Są one dzielone na okresy czasowe tworząc sesji z których powstają, po ich zakończeniu, pojedyncze mapy cieplne. Istnieją trzy typy sesji. Domyślnie, bez umieszczania dodatkowych detektorów, każdy ekran aplikacji który nie został wskazany w konfiguracji jako ignorowany jest nagrywany w celu tworzenia map cieplnych.

\paragraph{Mapy obszarów} Dodatkowo używający biblioteki twórca aplikacji może dodawać detektory monitorujące wybrane części ekranów w celu izolacji zachodzących na nich interakcji i umieszczenia ich na osobnej mapie cieplnej. Przypadkiem w którym jest to niezbędne do otrzymania poprawnego, zgodnego z rzeczywistością zapisu działań użytkownika są przewijane części interfejsu. Bez dodatkowego detektora dotknięcia zarejestrowane na przewijanym fragmencie zostałyby narysowane na mapie cieplnej obejmującej cały ekran. Ponieważ na statycznym, dwuwymiarowym obrazie nie jest możliwe przestawienie niemieszczącej się na ekranie zawartości bez zasłonienia jego części, względne odległości tych interakcji i ich pozycja względem tła byłaby nieprawidłowa. Osobna mapa cieplna obejmująca tylko i wyłącznie obszar przewijany może objąć całą jego powierzchnię i poprawnie narysować wszystkie interakcje użytkownika.

\paragraph{Mapy globalne} Trzeci typ map cieplnych może zawierać interakcje z więcej niż jednego ekranu aplikacji. Jest on użyteczny w sytuacjach, w których celem jest przeanalizowanie skumulowanego użycia elementu interfejsu występującego w tej samej postaci na wielu stronach aplikacji, takiego jak pasek nawigacji lub menu. Dzięki tej możliwości oprócz rozbitych na ekrany interakcji dotyczących wspólnego elementu stworzona zostanie dodatkowa mapa cieplna zbierająca je wszystkie.

\paragraph{Kontrola sesji}
Istnieją trzy mechanizmy powodujące zakończenie sesji i wygenerowanie z nich map cieplnych. Są to wyjście z aplikacji, czas bezczynności oraz bezpośrednie wyłączenia działania narzędzia. Pierwsze dwa z nich tworzą naturalne przerwy między kolejnymi okresami użycia aplikacji przez użytkowników co czyni je dobrym kryterium podziału interakcji na grupy znajdujące się na pojedynczej mapie cieplnej.

\subsection{Tworzenie map cieplnych}
dbscan, paths, gradients

\subsection{Formaty wyjściowe}
mapy - wariacje, raw

bajty + info

\section{Problemy implementacyjne}
W celu optymalizacji i przyspieszenia działania Flutter obsługuje mechanizm rysujący tylko fragmenty interfejsu które w danym momencie są wyświetlane na ekranie.

\subsection{Podejście pierwsze}
prostota i ograniczenia

Blokada interakcji podczas przewijania

\subsection{Podejście drugie}
skomplikowanie i możliwości

Składanie tła, monitorowanie pozycji
Wyścigi przy przewijaniu
Synchronizacja obrazu z pozycją
Dostęp do danych o statusie
