\section{Wstęp}
W ramach praktycznej części pracy zostało stworzone narzędzie służące do tworzenia map cieplnych na podstawie interakcji użytkowników z programami, aplikacjami oraz stronami internetowymi. Jego celem jest umożliwienie wygodnej analizy dostępności interfejsów oraz wzorców interakcji użytkowników poprzez ich rejestrację, przetworzenie i przygotowanie do zeskładowania. Narzędzie umożliwia twórcom łatwe przeglądanie i analizę zapisanych interakcji pod kątem identyfikacji problemów i ulepszania interfejsu użytkownika.

\subsection{Mapy cieplne}
Mapa ciepła to technika wizualizacji danych, która pokazuje wielkość zjawiska jako kolor w dwóch wymiarach. Różnice w wartościach mogą być wyrażone poprzez zmianę odcienia lub intensywności, dając odbiorcy pogląd dotyczący skupienia zjawiska i jego rozłożenia w przestrzeni \cite{Heat_map_definition}.

Ta technika wizualizacji szczególnie dobrze nadaje się do wizualizacji danych przestrzennych, które często są prezentowane w formie warstwy (będącej mapą cieplną) nałożonej na reprezentację przestrzeni której dotyczą. Dobrym przykładem są dane geograficzne które, aby były zrozumiałe, muszą być nałożone na mapę dostarczającą niezbędny dla nich kontekst.

\img{\chapterPath/HeatMiner_traffic_noise.jpg}{Wizualizacja poziomu hałasu w Helsinkach nałożona na mapę miasta \cite{Traffic_noise_Helsinki}.}{traffic_noise_map}{.8}

Innym obszarem w którym mapy cieplne znalazły zastosowanie jest wizualizacja interakcji użytkownika z urządzeniami elektronicznymi. Akcje takie jak dotknięcia i gesty na ekranach dotykowych oraz kliknięcia i ruchy myszy są danymi przestrzennymi znajdującymi się w układzie współrzędnych ekranu urządzenia. Dzięki temu intuicyjnym  sposobem ich reprezentacji są mapy cieplne nałożone na zrzut ekranu którego dotyczą.

\img{\chapterPath/hotjar_example.jpg}{Wizualizacja interakcji użytkownika ze stroną internetową \cite{Hotjar_example}.}{hotjar_example}{.8}


przydatność, wartość

skomplikowanie implementacji
