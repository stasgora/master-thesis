Mapa cieplna~to~technika wizualizacji danych, która pokazuje wielkość zjawiska jako kolor~w~dwóch wymiarach. Różnice~w~wartościach mogą być wyrażone poprzez zmianę odcienia~lub~intensywności, dając odbiorcy pogląd dotyczący skupienia zjawiska~i~jego rozłożenia~w~przestrzeni \cite{Heat_map_definition}.

Główną wartością wizualizacji~w~formie mapy cieplnej jest jej intuicyjność zachowywana pomimo możliwości jednoczesnego reprezentowania dużej ilości danych. Ta~charakterystyka wynika~z~ludzkiej ewolucyjnej zdolności~do~szybkiego analizowania obrazów~i~rozróżniania kolorów. Wartości numeryczne będące źródłem danych~do~wizualizacji~są~za~to~w~nieprzetworzonej formie ciężkie~i~żmudne~w~ręcznej analizie.

\section{Przykłady użycia}
Mapa cieplna jako technika wizualizacji szczególnie dobrze nadaje się~do~wizualizacji danych przestrzennych. Często przybiera formę półprzeźroczystej warstwy nałożonej~na~obraz będący reprezentacją przestrzeni, której dotyczą wizualizowane dane. Dobrym przykładem~są~dane geograficzne które, aby były zrozumiałe, muszą być nałożone~na~mapę dostarczającą niezbędny dla nich kontekst.

\bigskip
\img[\footnotemark]{\chapterPath/HeatMiner_traffic_noise.jpg}{Wizualizacja poziomu hałasu~w~Helsinkach nałożona~na~mapę miasta}{traffic_noise_map}{.8}
\footnotetext{\url{http://cloudnsci.fi/wiki/index.php?n=Archive.Traffic-Noise-in-Helsinki}}

Innym obszarem,~w~którym mapy cieplne znalazły zastosowanie, jest wizualizacja interakcji użytkownika~z~urządzeniami elektronicznymi. Akcje takie jak dotknięcia~i~gesty~na~ekranach dotykowych oraz kliknięcia~i~ruchy myszy~są~danymi przestrzennymi znajdującymi się~w~układzie współrzędnych ekranu urządzenia. Dzięki temu intuicyjnym sposobem ich reprezentacji mogą być mapy cieplne nałożone~na~zrzuty ekranów, których dotyczą.

\bigskip
\img[\footnotemark]{\chapterPath/hotjar_example.jpg}{Wizualizacja interakcji użytkownika~ze~stroną internetową}{hotjar_example}{.8}
\footnotetext{\url{https://digitalcommunications.wp.st-andrews.ac.uk/2021/01/07/how-we-are-using-hotjar-to-improve-our-web-pages/}}
