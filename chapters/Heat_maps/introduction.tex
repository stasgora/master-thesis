Mapa cieplna to technika wizualizacji danych, która pokazuje wielkość zjawiska jako kolor w dwóch wymiarach. Różnice w wartościach mogą być wyrażone poprzez zmianę odcienia lub intensywności, dając odbiorcy pogląd dotyczący skupienia zjawiska i jego rozłożenia w przestrzeni \cite{Heat_map_definition}.

Główną wartością wizualizacji w formie mapy cieplnej jest jej intuicyjność zachowywana pomimo możliwości jednoczesnego reprezentowania dużej ilości danych. Ta charakterystyka wynika z ewolucyjnej zdolności ludzkiej do szybkiego analizowania obrazów i rozróżniania kolorów. Wartości numeryczne będące źródłem danych do wizualizacji są przeciwnie ciężkie i żmudne w ręcznej analizie.

\section{Przykłady użycia}
Mapa cieplna jako technika wizualizacji szczególnie dobrze nadaje się do wizualizacji danych przestrzennych. Są one często są prezentowane w formie półprzeźroczystej warstwy nałożonej na obraz będący reprezentacją przestrzeni której dotyczą wizualizowane dane. Dobrym przykładem są dane geograficzne które, aby były zrozumiałe, muszą być nałożone na mapę dostarczającą niezbędny dla nich kontekst.

\img{\chapterPath/HeatMiner_traffic_noise.jpg}{Wizualizacja poziomu hałasu w Helsinkach nałożona na mapę miasta \cite{Traffic_noise_Helsinki}.}{traffic_noise_map}{.8}

Innym obszarem w którym mapy cieplne znalazły zastosowanie jest wizualizacja interakcji użytkownika z urządzeniami elektronicznymi. Akcje takie jak dotknięcia i gesty na ekranach dotykowych oraz kliknięcia i ruchy myszy są danymi przestrzennymi znajdującymi się w układzie współrzędnych ekranu urządzenia. Dzięki temu intuicyjnym  sposobem ich reprezentacji są mapy cieplne nałożone na zrzut ekranu którego dotyczą.

\img{\chapterPath/hotjar_example.jpg}{Wizualizacja interakcji użytkownika ze stroną internetową \cite{Hotjar_example}.}{hotjar_example}{.8}
