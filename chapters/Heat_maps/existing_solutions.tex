\section{Istniejące rozwiązania}

\paragraph{Rodzaje map}
\label{par:heat_map_types}
Przeanalizowane rozwiązania oferują tworzenie trzech typów map cieplnych:

\begin{itemize}
	\item {\bf Mapy kliknięć}: mapa cieplna której intensywność w danym obszarze odpowiada ilości kliknięć myszy w tym miejscu.
	\item {\bf Mapy przewijania ekranu}: mapa przewijanej strony której intensywność na danej wysokości odpowiada procentowi użytkowników którzy przewinęli stronę do tego miejsca.
	\item {\bf Mapy ruchu myszy}: mapa cieplna której intensywność w danym miejscu jest proporcjonalna do czasu który użytkownik trzymał w nim kursor myszy.
\end{itemize}

\subsection{Hotjar}
Wiodące w branży rozwiązanie oferujące usługi w zakresie analizy doświadczenia użytkownika \ang{user experience} prezentujące interakcje użytkowników w formie map cieplnych oraz nagrań. Firma oferuje też gotowe rozwiązania zbierające opinie użytkowników w formie tekstu, oceny oraz ankiet \cite{Hotjar_website}. Oprócz wymienionych wyżej \hyperref[{par:heat_map_types}]{rodzajów map} Hotjar oferuje także nagranie sesji użytkowników prezentowane w formie filmu z zaznaczonymi interakcjami użytkownika. Ruchy myszy pozostawiają ślad rysowany na ekranie jako linia łamana.

\paragraph{Wsparcie platform} 
Środowiskiem uruchomieniowym wspieranym przez Hotjar jest przeglądarka internetowa oraz język JavaScript. Usługi firmy są projektowane z myślą o stronach internetowych, jednak działają też na hybrydowych aplikacjach mobilnych opierających się na silniku przeglądarki internetowej. Zależność od stosu webowego znacznie ogranicza grupę klientów zupełnie wykluczając wszystkie aplikacje bazujące na rozwiązaniach natywnych dla ich systemu operacyjnego.

\paragraph{Ograniczenia} 
Hotjar wymienia szereg ograniczeń związanych z tworzonymi przez ich narzędzie mapami cieplnymi \cite{Hotjar_limitations}. Najpoważniejsze z nich dotyczą braku wsparcia dla zawartości dynamicznych oraz przewijanych (z wyłączeniem znacznika \inline{<body>}). Z powodu tego ograniczenia części strony które pokazują się w reakcji na działanie użytkownika, takie jak rozwijane menu lub okna dialogowe, nie będą zawarte w mapie cieplnej.

\subsection{Matomo}
Otwarte rozwiązanie monitorujące opisane w rozdziale \ref{sec:matomo}. Jedną z oferowanych przez nie funkcji jest tworzenie map cieplnych. Podobnie jak Hotjar oferuje trzy rodzaje map: kliknięć, ruchu myszy oraz przewijania ekranu \cite{Matomo_heatmaps}. Pomimo tych podobieństw Matomo szczyci się zapewnianiem ochrony danych użytkowników oraz wskazuje na większy zestaw oferowanych funkcji analitycznych w stosunku do konkurencyjnego rozwiązania Hotjar \cite{Matomo_hotjar}.

\paragraph{Wsparcie platform} 
Główną platformą wspieraną przez Matomo jest środowisko webowe. Dodatkowo oferowane są też wtyczki obsługujące platformy mobilne Android oraz iOS. Dzięki temu że Matomo jest otwarte i darmowe istnieje też wiele narzędzi stworzonych przez społeczność ułatwiających integrację z wieloma systemami zarządzania treścią, sklepami internetowymi oraz forami.

\subsection{Smartlook}
Płatne rozwiązanie monitorujące oferujące podobne do firmy Hotjar rozwiązania w zakresie tworzenia map cieplnych. Wspiera platformy webowe oraz aplikacje mobilne tworzone natywnie oraz przy użyciu narzędzi takich jak React Native, Unity, Flutter, Ionic oraz Xamarin.
