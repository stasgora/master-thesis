\begin{chapter}{Wstęp}
	\newcommand{\chapterPath}{chapters/Introduction}

	%\input{\chapterPath/context.tex}
	\section{Kontekst pracy}
	Praca ma charakter badawczo-wdrożeniowy
	
	\section{Uzasadnienie problemu}
	
	\section{Cel pracy}
	Celem pracy jest przeanalizowanie i porównanie różnych mechanizmów monitorowania aktywności użytkowników aplikacji mobilnych oraz implementacja wybranego mechanizmu w aplikacji terapeutycznej. Analiza rozwiązań i osiągnięć, zarówno w sektorze otwartego oprogramowania, komercyjnym jak i akademickim, ma na celu stworzenie możliwie szerokiego przeglądu stanu aktualnie istniejącego oprogramowania monitorującego aktywności użytkowników. Celem tworzonego narzędzia jest ułatwienie analizy danych aktywności w wybranym aspekcie a także zaoferowanie oryginalnych funkcji wyróżniających je na tle innych istniejących rozwiązań z tej samej kategorii.
	
	\section{Organizacja dokumentu}
	Na początku przedstawione i opisane zostały przykłady istniejących rozwiązań monitorujących, aplikacji przetwarzających dane z czujników, obserwujących użycie urządzeń na których działają oraz platform mających za zadanie monitorowanie aplikacji i ich użytkowania. Kolejny rozdział zawiera sprawozdanie z przeprowadzonego w ramach pracy systematycznego przeglądu literatury naukowej dotyczącej tematu wykorzystania danych z sensorów do przewidywania zachowań i aktywności. Pytanie przewodnie brzmiało: {\it ``Jakie metryki są używane do opisu ludzkiej aktywności?''}. 
	
	Dalsza część pracy skupia się na wybranym zagadnieniu z obszaru monitorowania, czyli wizualizacji interakcji użytkowników z urządzeniami w postaci map cieplnych. Po przedstawieniu tej techniki i jej najczęstrzych zastosowań wymienione zostały przykłady istniejących produktów oferujących usługi z zakresu zbierania i prezentacji interakcji w formie map cieplnych. Następnie szczegółowo opisany jest proces projektowania, tworzenia i wydawania oryginalnego narzędzia tego typu. Ten rozdział zawiera też opis napotkanych wyzwań, podjętych decyzji oraz użytych w celu zapewnienia jakości rozwiązań. Kolejny rozdział poświęcony jest procesowi ewaluacji stworzonego narzędzia, rozpoczynając od przygotowania i weryfikacji a kończąc na walidacji i wnioskach końcowych. Wymieniona jest także lista pomysłów i usprawnień które stanowią dobry punkt wyjścia przy dalszej pracy nad stworzonym rozwiązaniem.
\end{chapter}
