\begin{chapter}{Wstęp}
	\newcommand{\chapterPath}{chapters/Introduction}
% problemy zagadnienia
% badania metody

	\section{Kontekst pracy}
	Praca poświęcona jest tematowi rozwiązań informatycznych których celem jest monitorowanie, zbieranie oraz przetwarzanie informacji~o~działaniach, aktywności oraz zachowaniu użytkowników. Rozwój tej dziedziny jest spowodowany stale zwiększającym się rynkiem potencjalnych użytkowników oprogramowania i wiążącą się z nim rosnącą wartością tego typu danych. Wiedza użyciu produktu i występujących problemach jest podstawą przy podejmowaniu wielu decyzji biznesowych. Jeśli zostanie dobrze zastosowana może mieć duży wpływ na jakość, popularność produktu oraz związane w nim wyniki finansowe. Jednocześnie postępujący spadek cen wielu komponentów elektronicznych, takich jak czujniki,~a~także stale zwiększająca się moc obliczeniowa, która jest głównym ogranicznikiem  przy przetwarzaniu zebranych z nich informacji umożliwia rozwój nowej kategorii inteligentnych rozwiązań monitorujących swoje otoczenie. Czynnikiem mającym szczególny wpływ na wiele nowych, innowacyjnych rozwiązań w tej kategorii jest rozwój w dziedzinie uczenia maszynowego i sztucznych sieci neuronowych, które dobrze nadają się do tego typu zastosowań.
	
	\section{Cele pracy}
	Głównym celem pracy jest przeanalizowanie~i~porównanie różnych mechanizmów monitorowania aktywności użytkowników aplikacji mobilnych oraz implementacja wybranego mechanizmu~w~aplikacji terapeutycznej. Analiza rozwiązań~i~osiągnięć, zarówno~w~sektorze otwartego oprogramowania, komercyjnym jak~i~akademickim,~ma~na celu stworzenie możliwie szerokiego przeglądu stanu aktualnie istniejącego oprogramowania monitorującego aktywności użytkowników. Celem tworzonego narzędzia jest ułatwienie i automatyzacja procesu zbierania i analizy danych o interakcjach użytkowników z interfejsami aplikacji mobilnych. Jednocześnie zastosowanie odpowiedniego sposobu wizualizacji zebranych danych ma zwiększyć ich przydatność mierzoną w ilości uzyskanych z analizy wniosków. Dodatkowym zamierzeniem projektu jest innowacja w obszarze oferowanych funkcji i możliwości wyróżniających stworzone narzędzie na tle innych rozwiązań tej samej kategorii.
	
	\section{Organizacja dokumentu}
	Na początku przedstawione~i~opisane zostały przykłady istniejących rozwiązań monitorujących, aplikacji przetwarzających dane~z~czujników, obserwujących użycie urządzeń~na~których działają oraz platform mających~za~zadanie monitorowanie aplikacji~i~ich użytkowników. Kolejny rozdział zawiera sprawozdanie~z~przeprowadzonego~w~ramach pracy systematycznego przeglądu literatury naukowej dotyczącej tematu wykorzystania danych~z~sensorów~do~przewidywania zachowań~i~aktywności. Pytanie bazowe na którego podstawie formułowane było zapytanie brzmiało: {\it ``Jakie metryki~są~używane~do~opisu ludzkiej aktywności?''}. 
	
	Dalsza część pracy skupia się~na~wybranym zagadnieniu~z~obszaru monitorowania, czyli wizualizacji interakcji użytkowników~z~urządzeniami~w~postaci map cieplnych. Po~przedstawieniu tej techniki~i~jej najczęstszych zastosowań wymienione zostały przykłady istniejących produktów oferujących usługi~z~zakresu zbierania~i~prezentacji interakcji~w~formie map cieplnych. Następnie szczegółowo opisany jest proces projektowania, tworzenia~i~wydawania oryginalnego narzędzia tego typu. Przedstawione zostały też aspekty testów, dokumentacji i innych użytych~w~celu zapewnienia jakości rozwiązań. Napotkane w czasie implementacji wyzwania są zawarte w kolejnym rozdziale, po którym następuje szczegółowy opis procesu ewaluacji stworzonego narzędzia, rozpoczynający się~od~przygotowania~i~weryfikacji~a~kończący~na~walidacji~i~wnioskach końcowych. Wymieniona jest także lista pomysłów~i~usprawnień które stanowią dobry punkt wyjścia przy dalszej pracy~nad~projektem.
\end{chapter}
