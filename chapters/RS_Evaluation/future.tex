\section{Dalszy rozwój}
\label{sec:future_work}

\subsection{Typy interakcji}
Aktualnie jedyną wykrywaną formą interakcji użytkownika jest naciśnięcie ekranu palcem~na~urządzeniach dotykowych~lub~kliknięcie lewym przyciskiem myszy~na~komputerach. Obie formy interakcji pozwalają też jednak~na~wykonywanie gestów oraz akcji typu przeciągnij --- upuść \ang{drag \& drop}. Tego typu działania użytkownika mogłyby być także kolekcjonowane~i~przetwarzane~na~mapy cieplne,~w~których rysowane kształty~są~podłużne~i~odpowiadają trasom wybranym~przez~użytkownika.

\paragraph{Mapy gestów} Na~urządzeniach dotykowych~na~niektóre gesty składa się więcej niż jedno dotknięcie ekranu naraz. Najpowszechniejszym przykładem jest gest powiększenia~i~pomniejszenia, który jest wykonywany odpowiednio poprzez zbliżenie~i~oddalenie~od~siebie dwóch dotykających ekranu palców. Tego typu gesty ciężej byłoby zwizualizować~w~przejrzysty~i~zrozumiały sposób jako mapy cieplne. Mimo~to~zawarte~w~nich informacje~są~nie~mniej użyteczne~z~punktu widzenia analizy akcji wykonywanych~przez~użytkownika oraz interfejsu~i~jego dostępności.

\paragraph{Mapy ruchów myszy} Urządzenia obsługiwane~za~pomocą myszy prezentują możliwość dostępu~do~dodatkowego źródła informacji~o~działaniach użytkownika niedostępnego~na~ekranach dotykowych. W~przeciwieństwie~do~palców kursor myszy cały czas znajduje się~na~ekranie urządzenia. Aby wykonać dowolną interakcję, musi najpierw zostać przesunięty~w~odpowiednie miejsce. Co~więcej, wielu użytkowników wodzi kursorem myszy~po~obszarze ekranu,~na~którym aktualnie się skupiają,~na~przykład podczas czytania tekstu. Te~działania mogłyby zostać zawarte~w~mapie cieplnej ruchów kursora, dostarczając ważnych informacji~o~obszarach~i~elementach interfejsu,~na~których skupiają się użytkownicy.

\paragraph{Mapy przewijane} Ekrany przewijane mogą być źródłem dodatkowego rodzaju map cieplnych wizualizujących procent użytkowników, którzy przewinęli stronę~do~danego momentu jako zmianę koloru~w~tej osi. Ten rodzaj mapy mierzy zmianę zainteresowania treścią~w~miarę jej przeglądania~i~jest przydatną metryką,~na~podstawie której można optymalizować atrakcyjność zawartości wizualnej~i~zyskiwać nowych użytkowników.

\subsection{Pozostałe platformy}
Oprócz aplikacji mobilnych Flutter pozwala też~na~tworzenie stron internetowych oraz programów~na~systemy Linux, MacOS oraz Windows. Aby zapewnić wsparcie tych platform, wymagane byłoby przeprowadzenie testów kompatybilności, ewentualne wprowadzenie zmian oraz rozszerzenie załączonej dokumentacji~i~przykładów. W~tych środowiskach często używana jest mysz, dlatego~w~ramach zapewniania ich oficjalnego wsparcia warto byłoby też dodać specyficzne dla tego urządzenia mapy cieplne.

\subsection{Zawartość dynamiczna}
Tak jak zostało~to~już zaznaczone, elementy interfejsu, które dynamicznie zmieniają swój rozmiar~po~załadowaniu strony,~nie~są~aktualnie wspierane. Miejsca interakcji~są~rejestrowane jako przesunięcie~w~odniesieniu~do~lewego górnego rogu obszaru detektora. W~przypadku zmiany wielkości elementu~na~ekranie zawartość znajdująca się~po~nim zostanie przesunięta. W~efekcie zapisane już interakcje przestaną pokrywać się~z~elementami, których oryginalnie dotyczyły. 

Wsparcie dynamicznej zawartości znacznie rozszerzyłoby możliwości narzędzia, jednocześnie eliminując aktualnie występujące błędy wizualne. Po~wykryciu zmiany~w~drzewie elementów interfejsu zarejestrowane wcześniej interakcje musiałby zostać zaktualizowane poprzez dostosowanie ich miejsca~do~aktualnej pozycji odpowiadających~im~elementów. Alternatywnie przesunięcia interakcji mogłyby być zapisywane względem elementów interfejsu, których dotyczą zamiast całych obszarów detektorów. W~takim przypadku nawet~po~zmianie ich pozycji dane pozostawałyby prawidłowe.

\paragraph{Niezgodność zawartości}
Dane~do~map cieplnych~są~zbierane~z~wielu urządzeń. Jeśli ekran zawiera dynamiczną zawartość, każdy~z~użytkowników może mieć wyświetloną różną ilość tych samych,~lub~też zupełnie innych elementów. W~takim przypadku suma interakcji~z~różnych urządzeń~nie~będzie odpowiadała żadnemu pojedynczemu tłu. Ten fakt jest aktualnie akceptowany~a~tło służy~tylko~jako przybliżona zawartość mogąca niezupełnie zgadzać się~z~tym~co~widzą użytkownicy. W~przypadkach,~w~których dokładność tła jest ważniejsza niż ilość interakcji zgromadzonych~na~jednej mapie cieplnej, mogą one zostać podzielone~ze~względu~na~zawartość~tak,~aby lepiej odpowiadały wybranemu tłu.

\subsection{Automatyczna instrumentacja}
\label{sec:auto_instrumentation}
W celu tworzenia map cieplnych obszarów innych niż całe ekrany aplikacji,~na~przykład przestrzeni przewijanych, twórcy aplikacji~są~zmuszeni dodawać detektory~w~odpowiednich miejscach drzewa elementów interfejsu. To~znacznie zwiększa ilość pracy, która musi zostać włożona podczas integracji narzędzia, zwłaszcza~w~przypadku dużych aplikacji. Jednym~z~celów~na~przyszłość jest zmiana aktualnej implementacji~tak,~aby twórcy aplikacji~nie~musieli wykonywać tej dodatkowej pracy. Specyfika działania platformy Flutter utrudnia jednak zebranie wszystkich potrzebnych~do~przetwarzania interakcji danych~w~przypadku,~w~którym każdy obserwowany obszar~nie~posiada własnego detektora. Z~tego powodu ten cel został odłożony~na~przyszłość.

\subsection{Zwiększenie pokrycia testami}
\label{sec:future_coverage}
Aktualne \hyperref[par:test_coverage]{pokrycie testami}~nie~jest pełne, czego głównym powodem był ograniczony czas~na~wykonanie projektu. W~planach jest pokrycie testami całego projektu, włączając~w~to~problematyczne komponenty odpowiedzialne~za~przetwarzanie obrazów~i~łączenie części tła~w~całość. Oprócz testów jednostkowych aktualnie brakuje także testów integracyjnych weryfikujących poprawne współdziałanie wielu komponentów narzędzia.

\subsection{Prywatność}
Prywatność użytkowników jest ważnym aspektem każdego oprogramowania. Aktualnie wszystkie dane znajdujące się~na~ekranie aplikacji~w~momencie zapisywania tła~do~mapy cieplnej będą~na~niej umieszczone. Pojawia się zatem potrzeba dodania mechanizmu pozwalającego~na~usunięcie poufnych~i~wrażliwych informacji~ze~zdjęć~na~telefonie użytkownika przed ich wysłaniem. Aktualną propozycją jest dodanie komponentu analogicznego~do~detektora, który~po~umieszczeniu~w~drzewie elementów interfejsu~przez~twórcę aplikacji będzie odpowiedzialny~za~rozmazanie~na~zdjęciu obszaru, który obejmuje.

\subsection{Usługa}
Aby wdrożyć aktualną wersję narzędzia~w~aplikacji produkcyjnej, jej twórcy~są~zmuszeni sami zapewnić przestrzeń dyskową~do~magazynowania zebranych danych oraz dostarczyć sposób ich przesyłania. Aby umożliwić zmianę aktualnej konfiguracji~bez~aktualizacji wersji aplikacji, muszą też zapewnić sposób jej pobierania~na~urządzenia użytkowników przy starcie aplikacji. Te~ograniczenia wynikają~z~aktualnej charakterystyki architektury dostarczonego rozwiązania, które jest darmowe~i~działa~tylko~na~urządzeniu użytkownika \ang{client only}.

Zapewnienie miejsca składowania danych~i~ich automatycznego przesyłania,~a~także dostarczania konfiguracji wymagałoby przekształcenia prostego narzędzia~w~pełnoprawną płatną usługę. Klienci mieliby dostęp~do~stworzonych map cieplnych poprzez serwis internetowy, gdzie mogliby przeglądać, filtrować~i~eksportować zebrane dane. Wybrana konfiguracja byłaby automatycznie pobierana~przez~wszystkie urządzenia użytkowników monitorowanych aplikacji, znacznie upraszczając proces integracji wykonywany~przez~twórców aplikacji.
