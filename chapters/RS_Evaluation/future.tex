\section{Dalszy rozwój}
\label{sec:future_work}

\subsection{Typy interakcji}
Aktualnie jedyną wykrywaną formą interakcji użytkownika jest naciśnięcie ekranu palcem na urządzeniach dotykowych lub kliknięcie lewym przyciskiem myszy na komputerach. Obie formy interakcji pozwalają też dodatkowo na wykonywanie gestów oraz akcji typu przeciągnij-upuść \ang{drag \& drop}. Tego typu działania użytkownika mogłyby być także kolekcjonowane i przetwarzane na mapy cieplne w których rysowane kształty są podłużne i odpowiadają trasom wybranym przez użytkownika.

Na urządzeniach dotykowych na niektóre gesty składa się więcej niż jedno dotknięcie ekranu naraz. Najpowszechniejszym przykładem jest gest powiększenia i pomniejszenia, który jest wykonywany odpowiednio poprzez zbliżenie i oddalenie od siebie dwóch dotykających ekranu palców. Tego typu gesty ciężej byłoby zwizualizować w przejrzysty i zrozumiały sposób jako mapy cieplne. Mimo to zawarte w nich informacje są nie mniej użyteczne z punktu widzenia analizy interfejsu użytkownika i jego dostępności.

Urządzenia obsługiwane za pomocą myszy prezentują możliwość dostępu do dodatkowego źródła informacji o działaniach użytkownika niedostępnego na ekranach dotykowych. W przeciwieństwie do palców, kursor myszy cały czas znajduje się na ekranie urządzenia. Aby wykonać dowolną interakcję musi najpierw zostać przesunięty w odpowiednie miejsce. Co więcej wielu użytkowników wodzi kursorem myszy po obszarze ekranu na którym aktualnie się skupiają, na przykład podczas czytania tekstu. Te działania mogłyby zostać zawarte w mapie cieplnej ruchów kursora dostarczając ważnych informacji o obszarach i elementach interfejsu na których skupiają się użytkownicy.

Ekrany przewijane mogą być źródłem dodatkowego rodzaju map cieplnych wizualizujących procent użytkowników którzy przewinęli stronę do danego momentu jako zmianę koloru w tej osi. Ten rodzaj mapy mierzy zmianę zainteresowania treścią w miarę jej przeglądania i jest przydatną metryką na podstawie której można optymalizować atrakcyjność zawartości wizualnej i zyskiwać nowych użytkowników.

\subsection{Platformy}
Web, Win, Mac, Lin 

\subsection{Zawartość}
Dynamic content

\subsection{Automatyczna instrumentacja}
\label{sec:auto_instrumentation}

\subsection{Prywatność}
Blur

\subsection{Przetwarzanie lokalne}
serwer, mapy całego ruchu bez podziału na urządzenia

\subsection{Zwiększenie pokrycia testami}
\label{sec:future_coverage}
