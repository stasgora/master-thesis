\section{Przygotowanie}
Do przeprowadzenia testów oraz ewaluacji potrzebna była odpowiednia aplikacja zbudowana~na~platformie Flutter. Wymaganiem był swobodny dostęp~do~jej kodu źródłowego, możliwość jego modyfikacji oraz ponownej kompilacji. Aplikacja powinna być umiarkowanie rozbudowana,~z~minimum ośmioma ekranami, przestrzeniami przewijanymi oraz elementami nawigacji. Podczas jej używania powinien być wymagany dostęp~do~internetu, aby zapewnić możliwość zapisu zebranych danych. 

\subsection{Fokus}
Aplikacja mobilna Fokus została stworzona~w~ciągu ostatniego roku~w~ramach projektu grupowego~przez~zespół prowadzony~przez~autora tej pracy. Jej celem jest pomaganie dzieciom~w~wykonywaniu codziennych zadań~pod~kontrolą opiekuna. Tworzy~on~plany~i~zadania~do~zrobienia,~za~które dziecko dostaje punkty wydawane~na~dodane~przez~opiekuna nagrody. Aplikacja dobrze nadaje się~do~wykorzystania~do~przeprowadzenia~na~niej ewaluacji. Jest wystarczająco rozbudowana, dzieląc się~na~sekcje dostępne dla opiekuna~i~dziecka,~z~których każda zawiera około ośmiu ekranów. Zawiera różnorodne elementy interfejsu użytkownika, takie jak przewijane karty, zakładki, wyskakujące okna, paski nawigacji oraz pola~do~wprowadzania danych. Jest też aktywnie testowana~przez~grupę użytkowników.

\subsection{Integracja}
Proces integracji narzędzia~w~aplikacji Fokus składał się~z~konfiguracji oraz instrumentacji interfejsu użytkownika. Detektory interakcji zostały umieszczone~naokoło każdego obszaru przewijanego oraz tych wyskakujących okien, które~na~to~pozwalały. Do~konfiguracji posłużyły dwie~z~usług oferowanych~w~ramach używanej~przez~aplikację platformy \nameref{sec:firebase}. Moduł zdalnej konfiguracji (Remote Config) jest odpowiedzialny~za~pobieranie aktualnej \hyperref[sec:rs_config]{konfiguracji}~w~formie pliku {\it JSON}. Dzięki temu mechanizmowi parametry działania związane~ze~specyfiką zbierania danych mogą zostać zmienione~w~dowolnym momencie~bez~potrzeby instalacji nowej wersji aplikacji~przez~testerów. Po~zakończeniu ewaluacji kolekcja danych może też zostać zdalnie wyłączona. Firebase oferuje też usługę dysku~w~chmurze (Firebase Storage), która została użyta~w~celu zebrania danych~ze~wszystkich urządzeń użytkowników biorących udział~w~ewaluacji. Z~tej centralnej lokalizacji pliki mogą zostać łatwo pobrane~i~przetworzone~na~mapy cieplne.
