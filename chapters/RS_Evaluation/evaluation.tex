\section{Przygotowanie}

\subsection{Fokus}
przygotowanie, integracja 

\section{Weryfikacja}

\paragraph{Wskazanie popularności elementów}
Przegląd map cieplnych pozwala na łatwą identyfikację częściej i rzadziej używanych funkcji, ekranów i elementów interfejsu.

\paragraph{Rodzielenie ekranu na obszary}


\paragraph{Niewidoczne elementy}


\paragraph{Zmienne elementy}
W przypadku przewijanych obszarów zawierających animowane (zmieniające wielkość, kształt lub zawartość) elementy, tło stworzonych map może nie być spójne. Wynika to z opisanego już mechanizmu stopniowego tworzenia tła z części wyświetlanych na ekranie w danym momencie.


\section{Walidacja}


\section{Wnioski}

\subsection{Korzyści}
Narzędzie pozwala na automatyczne zbieranie cennych danych których pozyskanie w innym przypadku wymagałyby zorganizowania czasochłonnych testów. Dzięki intuicyjnej, graficznej reprezentacji interakcji połączonej z dobrą znajomością interfejsu jego twórca jest w stanie znacznie szybciej diagnozować problemy które mają użytkownicy w trakcie używania aplikacji. Nagrania zebrane ze zwykle przeprowadzanych testów są zazwyczaj cięższe i wolniejsze w analizie z powodu konieczności manualnego spisywania interakcji, braku informacji o dokładnych miejscach dotknięć ekranu oraz częściowym zasłanianiu obrazu przez rękę użytkownika. 
