\section{Sformułowane wnioski}

\subsection{Istniejące aplikacje~i~platformy}
Platformy monitorujące serwisy internetowe~i~aplikacje mobilne rozwinęły się~w~złożone rozwiązania zawierające szeroki zbiór funkcji~i~narzędzi mających dostarczać różnorodne informacje~o~produktach~i~ich użytkownikach. Ta~strategia~ma~zwiększyć potencjalne grono klientów którzy mogą być zainteresowani mniejszym podzbiorem funkcjonalności platformy. Porównanie konkurencyjnych produktów wykazało duże podobieństwo oferowanego zestawu narzędzi, które często różnią się głównie nazewnictwem. Choć dzięki temu łatwiej jest zmienić dostawcę używanych usług monitorowania muszą oni znaleźć inne pole~do~wyróżnienia się~na~tle konkurencji. Rynek aplikacji jest~w~porównaniu duży~i~różnorodny. Także~w~obszarze monitorowania zastosowania zebranych danych~są~mocno zróżnicowane. Popularne systemy mobilne oferują konkurencyjne rozwiązania podsumowujące wykorzystanie urządzenia~o~podobnych funkcjach. W~kategorii użycia zawartych~w~telefonach czujników najczęstszym skojarzeniem jest monitorowanie aktywności fizycznych jednak~to~nie~ich jedyne zastosowanie. Opisana~w~pracy aplikacja \nameref{sec:sleep_cycle} jest świetnym przykładem nieszablonowego, innowacyjnego wykorzystania powszechnie dostępnych sensorów, mającym realny, pozytywny wpływ na codzienne życie użytkowników.

\subsection{Literatura naukowa}
Największym zaskoczeniem~po~zapoznaniu się~ze~statystykami~i~zawartością wybranego zbioru artykułów było ich wzajemne podobieństwo. Większość publikacji używała takiego samego~lub~prawie identycznego zestawu sensorów (najczęściej zawierających akcelerometr, magnetometr, żyroskop oraz mikrofon), które następnie były wykorzystywane~do~wykrywania powtarzającego się~z~małymi wariacjami zestawu prostych aktywności ruchowych (chodzenia, stania, siedzenia, biegania~i~leżenia). Tylko~ niewielka część~z~wybranego zestawu publikacji opisywała oryginalny,~w~znaczeniu wykrywanych modalności~lub~wykorzystywanego zestawu sensorów, eksperyment. Zakładając reprezentatywność analizowanego zbioru artykułów ich powtarzalność~i~skupienie~na~prostych, niskopoziomowych modalnościach wskazuje~na~wczesne stadium rozwoju tej dziedziny~i~problemy które stoją~na~drodze~do~stworzenia dobrze działającego~w~warunkach świata rzeczywistego systemu wykrywającego aktywności ludzkie.

\subsection{Narzędzie monitorujące}

\paragraph{Komplikacje i wyzwania}
Wybrany temat okazał się zdecydowanie bardziej skomplikowany niż~na~początku się wydawało. Stworzenie użytecznych map cieplnych wymaga dużej ilości czasowo~i~wzajemnie powiązanych danych, których zebranie~nie~jest trywialne. Dalsza część procesu ich przetwarzania także zaprezentowała szereg trudności dotyczących rysowania map, tworzenia przewijanych obrazów tła oraz wyboru odpowiedniej architektury dającej swobodę grupowania~i~obróbki danych. Złożoność implementacyjna problemu, szczególnie jeśli~nie~został wcześniej dogłębnie zbadany~a~jego dziedzina~nie~jest dobrze znana,~nie~powinna być nigdy bagatelizowana. W~takim przypadku dobrym podejściem jest prototypowanie~i~iteracja, która pozwala~bez~większych kosztów~i~opóźnień zgłębić temat~i~zweryfikować podejście~do~niego, zyskując dodatkową wiedzę pozwalającą na dopasowanie podejścia i stworzenie lepszego, bardziej optymalnego rozwiązania.

\paragraph{Funkcje i kierunki rozwoju}
W trakcie zgłębiania tematu~na~powierzchnię wypłynęło wiele dodatkowych aspektów, które zwiększyłyby użyteczność tego typu narzędzia. Parę~z~tych~zidentyfikowanych tematów udało się zaimplementować (jak mapy przewijane), jednak duża część~z~nich pozostaje~w~tym momencie poza możliwościami stworzonego rozwiązania (jak dynamiczna zawartość monitorowanych stron). Pomimo początkowego niedoidentyfikowania zbioru potencjalnych funkcjonalności narzędzia były one zbierane~w~miarę implementacji~i~zapoznawania się~ze~specyfiką platformy. Ich lista została zamieszczone~w~sekcji \nameref{sec:future_work}. Horyzont czasowy pisania pracy znacznie ograniczył możliwości rozwoju stworzonego rozwiązania, jednak dostępny czas został rozdzielony~na~najważniejsze funkcje~i~mechanizmy~w~celu uzyskania możliwie spójnego, kompletnego~i~dobrze działającego narzędzia które może być dalej rozwijane~w~przyszłości.

\paragraph{Przewaga w testach interfejsu}
Przeprowadzona ewaluacja wskazuje słabe punkty działania narzędzia, jednak jednoznacznie ujawnia przydatność~i~wartość którą wnosi podczas przeprowadzanych testów~i~badań aplikacji mobilnych. Stworzone narzędzie pozwala~na~automatyczne zbieranie cennych danych których pozyskanie~w~innym przypadku wymagałyby zorganizowania czasochłonnych testów. Dzięki intuicyjnej, graficznej reprezentacji interakcji połączonej~z~dobrą znajomością interfejsu jego twórca jest~w~stanie znacznie szybciej diagnozować problemy które mają użytkownicy~w~trakcie używania aplikacji. Nagrania kamerą zbierane~z~klasycznych testów~są~zazwyczaj cięższe~i~wolniejsze~w~analizie~z~powodu konieczności manualnego spisywania interakcji, braku informacji~o~dokładnych miejscach dotknięć ekranu oraz częściowym zasłanianiu obrazu~przez~rękę użytkownika. 

\paragraph{Alternatywne rozwiązania}
Zestaw oferowanych funkcji~i~sposób działania stworzonego rozwiązania jest znacząco inny~od~przeanalizowanych istniejących produktów tej kategorii. Integracja~z~aplikacją jest mniej zautomatyzowana, wymagając więcej wkładu pracy~od~jej twórców. Ta~decyzja projektowa była wynikiem kompromisu pomiędzy łatwością implementacji~a~użycia narzędzia biorąc~pod~uwagę okres czasu dostępny~na~jego stworzenie. Przesunięcie części odpowiedzialności~na~twórcę aplikacji pozwoliło~na~poświęcenie większej uwagi oferowanym funkcjom. Dzięki temu opublikowany projekt wyróżnia się wsparciem przewijanych map cieplnych zajmujących fragment ekranu, podczas gdy najpopularniejszy komercyjny produkt, HotJar~nie~oferuje tej możliwości \cite{Hotjar_limitations}. Jednocześnie należy zaznaczyć~że~ma~on~inne zalety, takie jak wymazywanie wrażliwych informacji~z~robionych zdjęć ekranu,~co~jednak~nie~powinno dziwić biorąc~pod~uwagę~że~stoi~za~nim duża firma.

\paragraph{Inne typy rozwiązań}
W porównaniu~do~klasycznych platform monitorujących takich jak Google Analytics, przedstawione rozwiązanie, chociaż prostsze~i~posiadające mniej funkcji, umożliwia zebranie zestawu danych cechującego się dużą szczegółowością~i~przystępną, ułatwiającą analizę prezentacją. Najbliższą oferującą podobne możliwości~w~tym względzie usługą często oferowaną~przez~platformy monitorujące~są~opisane~w~sekcji \ref{par:ga-funnels} lejki. Cechują się one dosyć  ubogim zestawem danych których prezentacja zwykle sprowadza się~do~widoku zdefiniowanych ręcznie kroków~i~ilości użytkowników którzy~do~nich dotarli. Mapy cieplne oprócz prezentacji podobnej statystyki (poprzez policzenie interakcji~w~kolejnych obszarach) pomagają zrozumieć jak~i~dlaczego użytkownicy wykonują takie~a~nie~inne akcje.
