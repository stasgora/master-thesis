Zdefiniowany cel pracy magisterskiej składa się~z~dwóch części. Pierwszą~z~nich jest analiza~i~porównanie~z~sobą różnych mechanizmów monitorowania aplikacji i ich użytkowników, głównie~w~kontekście aplikacji mobilnych. Analiza miała być możliwie przekrojowa~i~obejmować przykłady~z~wielu kategorii rozwiązań monitorujących. Drugie wskazane~do~realizacji zadanie dotyczyło skupienia się~na~jednym~z~analizowanych mechanizmów~i~zaimplementowania~go~w~istniejącej aplikacji terapeutycznej. Stworzone rozwiązanie powinno automatyzować i upraszczać proces którego dotyczy, jednocześnie oferując wartość dodaną wynikającą z zebranych danych.
	
\section{Osiągnięte rezultaty}

\subsection{Przegląd istniejących rozwiązań}
Wykonana analiza istniejących mechanizmów monitorujących zajmująca rozdziały \ref{cha:existing_solutions}~do~\ref{cha:heat_maps} rozpatruje ten temat~z~wielu perspektyw. Opisane~w~niej zostały przykłady gotowych rozwiązań, takich jak aplikacje, mechanizmy systemów mobilnych~i~komercyjne platformy oferujące usługi~z~zakresu monitorowania. Wykonany systematyczny przegląd literatury naukowej skupia się~na~aspekcie wykorzystania czujników~w~urządzeniach takich jak smartfony~w~celu wykrywania~i~opisu aktywności ludzkich. Przedstawione zostały także przykłady popularnych produktów~z~wybranego~do~implementacji~w~dalszej części pracy typu rozwiązań tworzących mapy cieplne interakcji użytkowników. Wymienione przykłady zostały ze~sobą porównane~pod~względem funkcji, podobieństw~i~zaawansowania.

\subsection{Przegląd literatury}
Jako część analizy istniejących mechanizmów monitorowania przeprowadzony został systematyczny przegląd literatury naukowej poświęconej nowym metodom wykorzystania czujników. Zadane pytanie dotyczyło modalności, które wykorzystywane są do opisu działalności człowieka. W ramach przeglądu zidentyfikowanych zostało 648 publikacji naukowych które następnie zostały zawężone do trzydziestu dwóch podczas trzystopniowego procesu selekcji. Wybrane artykuły zostały przeanalizowane w analizie jakościowej oraz ilościowej, w ramach której stworzone zostały statystyki typów artykułów, poruszanych zastosowań monitorowania, używanych sensorów, wykrywanych modalności oraz wykorzystywanych metryk.

\paragraph{Ilość uczestników}
Głównym ograniczeniem zakresu przeprowadzonego przeglądu była liczba zaangażowanych uczestników. Oprócz trzeciego etapu selekcji artykułów który został wykonany~przez~dr hab. inż. Agnieszkę Landowską całość została wykonana~przez~autora pracy. Ten fakt znacząco ograniczył ilość publikacji które mogły zostać przestudiowane~po~etapie selekcji. Z~648 artykułów które zostały wstępnie zidentyfikowane jako pasujące~do~zadanego pytania~tylko~ 32 zostało przeczytanych~lub~przeglądniętych~a~co~za~tym idzie włączonych~do~wykonanej analizy ilościowej i jakościowej.

\subsection{Stworzone rozwiązanie monitorujące}
Wybrany~do~implementacji aspekt monitorowania polega~na~obserwacji interakcji użytkowników~z~interfejsem aplikacji~i~ich późniejszym przetworzeniu~do~graficznej formy map cieplnych poszczególnych ekranów. Zamiast bezpośredniej implementacji~w~wybranej aplikacji terapeutycznej powstało uniwersalne narzędzie które może zostać wykorzystane~w~dowolnej, opartej~o~platformę Flutter aplikacji,~co~znacznie zwiększa jego potencjalny obszar wpływu. Praca~nad~tworzeniem rozwiązania była przeprowadzona według najlepszych standardów jakości kodu oraz jego testowania. Wykonana ewaluacja składająca się~ze~wstępnej weryfikacji oraz walidacji~z~udziałem testerów wykazała słabe~i~mocne strony narzędzia potwierdzając jednocześnie jego przydatność~w~analizie dostępności interfejsu~i~zachowań użytkowników. 
	
\section{Napotkane wyzwania}
Podczas procesu implementacji~i~ewaluacji tworzonego narzędzia wystąpiło wiele problemów stanowiących poważne wyzwania stojące~na~drodze~do~poprawnie działającego, użytecznego rozwiązania. Część~z~potrzebnych danych, takich jak obrazy części ekranu, ich pozycje~lub~informacje~o~obszarach przewijanych~są~dosyć nietypowe~z~punktu widzenia użycia platformy Flutter~co~utrudniło ich pozyskanie~i~użycie. Wyzwaniem okazało się też rysowanie punktów~na~mapie cieplnej~w~atrakcyjny wizualnie~i~przejrzysty~w~analizie sposób uwzględniając jednocześnie parametry takie jak wielkość, kształt oraz gradient koloru. Wybór właściwej architektury oraz przepływu~i~przetwarzania danych dający odpowiednią swobodę ich manipulacji także początkowo sprawił problemy. Podobnie sprawa wyglądała~z~opisanymi~w~pracy obrazami map przewijanych które musiały być składane~z~części oraz końcowym procesem łączenia danych zebranych~z~urządzeń~o~różniących się rozmiarach ekranu. Wszystkie napotkane problemy udało się rozwiązać zmieniając podejście gdzie tam gdzie było~to~konieczne tworząc ostatecznie dobrze działające, użyteczne narzędzie spełniające założenia~i~postawione przed nim oczekiwania.
