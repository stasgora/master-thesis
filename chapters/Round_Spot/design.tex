W ramach praktycznej części pracy zostało stworzone narzędzie służące~do~tworzenia map cieplnych~na~podstawie interakcji użytkowników~z~programami, aplikacjami oraz stronami internetowymi. Jego celem jest umożliwienie wygodnej analizy dostępności interfejsów oraz wzorców interakcji użytkowników poprzez ich rejestrację, przetworzenie~i~zapis. Round Spot umożliwia twórcom łatwe przeglądanie~i~analizę zapisanych interakcji~pod~kątem identyfikacji problemów~i~ulepszania interfejsu użytkownika.

\section{Decyzje projektowe}

\paragraph{Wymagania} Przed rozpoczęciem implementacji został zdefiniowany zestaw wymagań projektowych dostępny~w~\annexref[u]{tool_requirements}. Określają podstawowe charakterystyki działania oraz dostępne opcje konfiguracji.

\subsection{Wybór platformy}
Narzędzie zostało zaimplementowane dla nowej, aktywnie rozwijanej platformy Flutter. Została ona wybrana~z~powodu swojej innowacyjności, oferowanych możliwości oraz rosnącej popularności wśród twórców aplikacji. Dodatkowo, ponieważ platforma jest stosunkowo młoda,~w~momencie pisania pracy~nie~istnieje otwarte, darmowe rozwiązanie tego typu. Istniejące narzędzia oferujące podobne możliwości~są~przeznaczone~w~dużej większości~na~platformę webową.

Ponieważ Round Spot został pomyślany jako wtyczka \ang{plugin}, dołączana~i~integrowana~z~kodem aplikacji opartej~o~Flutter,~do~wstępnych testów~i~ewaluacji potrzebna była gotowa aplikacja, znajomość jej budowy, dostęp~do~kodu źródłowego oraz możliwość zbudowania nowej wersji. W~ramach projektu grupowego odbywającego się~przez~pierwsze dwa semestry studiów magisterskich autor tej pracy kierował zespołem tworzącym aplikację Fokus, która stanowiła idealną bazę~do~przeprowadzenia testów~i~ewaluacji tworzonego narzędzia.

\subsection{Flutter}
Zestaw narzędzi stworzony~przez~Google, służący~do~tworzenia aplikacji~na~wiele platform. Pozwala~na~napisanie jednej bazy kodu \ang{codebase},~z~której budowane~są~aplikacje mobilne~na~systemy Android~i~iOS, programy komputerowe działające~na~Windowsie, Linuxie~i~MacOS oraz strony internetowe. Flutter osiąga ten efekt dzięki stworzeniu własnego, niezależnego~od~platformy docelowej mechanizmu budowania drzewa interfejsu definiowanego~w~języku Dart. 

Zależnie~od~specyfiki~i~możliwości oferowanych~przez~platformę docelową drzewo interfejsu jest przetwarzane~i~prezentowane~w~oknie aplikacji~w~inny sposób. W~przypadku budowania strony internetowej elementy~są~rysowane~na~płótnie (znacznik~\inline{<canvas>}). Pozostałe platformy oferują analogiczne rozwiązania. Dzięki użyciu wspomagania sprzętowego Flutter osiąga wydajność~i~płynność działania porównywalną~z~programami budowanymi natywnie. Dodatkowymi zaletami tego podejścia jest przekazanie deweloperowi kontroli~nad~każdym wyświetlanym~na~ekranie pikselem oraz identyczny wygląd interfejsu niezależnie~od~platformy.

Flutter~nie~używa~do~budowy interfejsu gotowych komponentów oferowanych~przez~każdą~ze~wspieranych~przez~siebie platform. Zamiast tego implementuje swoje mechanizmy zapewniające między innymi możliwość nawigowania między stronami aplikacji oraz przewijania treści. Z~tego powodu wszystkie biblioteki napisane~na~platformy takie jak Android~i~iOS, korzystające~z~danych~o~zawartości ekranu~i~jego zmianach,~nie~będą~w~stanie działać (lub ich działanie będzie znacznie ograniczone)~w~aplikacjach zbudowanych przy pomocy Fluttera. Opisywane~w~tej pracy narzędzie jest,~o~ile wiadomo autorowi, pierwszym tego typu otwartym rozwiązaniem zbudowanym~w~całości~na~platformie Flutter.

\subsection{Architektura}
Narzędzie składa się~z~dwóch części. Do~aplikacji mobilnej dołączana jest wtyczka odpowiedzialna~za~zbieranie~i~wstępne przetwarzanie danych. Pakiety~od~poszczególnych użytkowników~są~serializowane~i~zapisywane~w~chmurze. Drugim komponentem jest procesor danych który~po~pobraniu zebranych danych deserializuje je, grupuje~ze~względu~na~ekran którego dotyczą~i~przetwarza~na~mapy cieplne. Diagram architektury został przedstawiony~na~rysunku \ref{fig:round_spot_diagram}.
\bigskip
\img{\chapterPath/RoundSpot_Architecture.png}{Round Spot - Diagram architektury}{round_spot_diagram}{.9}

\paragraph{Aplikacja}
Twórca aplikacji dokonuje integracji narzędzia poprzez jego inicjalizację,~w~ramach której podawane~są~ opcjonalne parametry konfiguracyjne. Ponadto~z~uwagi~na~aktualne ograniczenia wymagane jest ręczne umieszczenie detektorów~w~odpowiednich miejscach~w~interfejsie aplikacji.

\paragraph{Flutter} Platforma będąca podstawą działania aplikacji obsługuje zdarzenia takie jak zmiana ekranu, wyjście~z~aplikacji~a~także interakcje użytkownika które~są~przekazywane odpowiednio~do~obserwatorów stanu oraz detektorów.

\paragraph{Detektory} 
\label{par:rs_detectors}
Są odpowiedzialne~za~zbieranie przetwarzanych później danych, takich jak interakcje użytkownika~i~częściowe obrazy tła. Każdy~z~detektorów zbiera dane~z~fragmentu ekranu którego jest przodkiem~w~drzewie elementów interfejsu.

\paragraph{Obserwatory stanu} 
\label{par:rs_observers}
Ich zadaniem jest informowanie menedżera sesji~o~wydarzeniach takich jak zmiana ekranu~lub~wyjście~z~aplikacji.

\paragraph{Sesja}
\label{par:rs_session}
Ograniczony czasowo zbiór danych, odpowiadający konkretnemu obszarowi interfejsu, zawierający wykonane~w~nim~przez~użytkownika interakcje oraz zrzut ekranu. Każda sesja posiada tekstowy identyfikator jednoznacznie identyfikujący obszar ekranu którego dotyczy.

\paragraph{Menedżer sesji} 
\label{par:rs_session_manager}
Centralny komponent odpowiedzialny zbieranie danych~do~aktualnie tworzonych map cieplnych oraz koordynację działania pozostałych komponentów. Decyduje,~w~oparciu~o~otrzymaną konfigurację, kiedy każda~z~sesji powinna się zakończyć~i~przekazuje zebrane~w~ramach niej dane~do~serializacji. Informuje też menedżera tła~o~zachodzących zdarzeniach, takich jak przewijanie ekranu.

\paragraph{Menedżer tła} 
\label{par:rs_bg_manager}
Tworzy obrazy interfejsu aplikacji będące tłem dla tworzonych map cieplnych polegając~na~sygnałach otrzymanych~od~menedżera sesji. W~przypadku obszaru przewijanego wynikowy obraz jest efektem złączenia wielu  obrazów cząstkowych zbieranych~w~miarę wyświetlania się~na~ekranie kolejnych fragmentów interfejsu.

\paragraph{Serializacja danych}
Przygotowuje aktualnie zebrany zestaw danych~do~wysłania poprzez przetworzenie~go~do~postaci pliku binarnego. Jako protokół serializacji została użyta biblioteka {\it Protocol Buffers} stworzona~przez~Google.

\paragraph{Google Cloud Storage}
Pakiety danych~ze~wszystkich urządzeń~są~wysyłane~do~jednej lokalizacji umożliwiając ich późniejsze pobranie~i~dalsze przetwarzanie. Do~składowania danych została użyta usługa {\it Firebase Storage} zbudowana~na~{\it Google Cloud Storage}.

\paragraph{Deserializacja~i~agregacja}
Po pobraniu, paczki~są~deserializowane~z~użyciem biblioteki {\it Protocol Buffers}. Następnie otrzymane dane podlegają agregacji służącej połączeniu informacji zebranych~ze~wszystkich urządzeń~i~pogrupowaniu ich~ze~względu~na~ekran którego dotyczą.

\paragraph{Procesor graficzny} 
\label{par:rs_graphical_processor}
Przetwarza zgrupowane interakcje~do~graficznej postaci mapy cieplnej nałożonej~na~obraz zawartości ekranu. Interakcje położone blisko siebie~są~grupowane~za~pomocą algorytmu DBSCAN \cite{DBSCAN_Wiki}~a~ich ilość~i~rozłożenie~w~grupie definiuje jej intensywność~i~kształt~na~wynikowej mapie.

