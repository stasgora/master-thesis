\section{Zapewnianie jakości}
Twórcy języka Dart,~w~którym pisane~są~aplikacje~i~narzędzia działające~na~platformie Flutter, opublikowali szereg wytycznych~i~wskazówek mających podnieść jakość, zwięzłość, czytelność~i~niezawodność kodu \cite{Effective_Dart}. Stworzyli też narzędzia pomagające deweloperom~w~trzymaniu się tych wytycznych~i~automatyzacji niektórych powtarzalnych czynności \cite{Dart_SDK}.

\subsection{Używane narzędzia}

\paragraph{Analiza statyczna}
\label{par:static_analysis}
Linter jest kolekcją stale rozwijanych reguł języka Dart określających błędy oraz dobre praktyki konstrukcji~i~stylu kodu \cite{Dart_Lints}. Każdy projekt  może dostosować listę reguł, które zostaną zaaplikowane~do~znajdującego się~w~nim kodu. Narzędzie \codeinline{dart analyze} służy~do~przeprowadzania statycznej analizy kodu, opierając się~o~wybrane reguły \cite{Dart_Analyze}. Wskazuje ono programiście potencjalne błędy~i~problemy~w~kodzie~bez~uruchamiania go. Duży wybór~i~różnorodność reguł (aktualnie jest ich prawie 200) pomaga uniknąć części problemów~i~utrzymać jakość pisanego programu~w~trakcie jego powstawania.

\paragraph{Formatowanie}
\label{par:dart_format}
Narzędzie \codeinline{dart format} zmienia ułożenie znaków białych~w~plikach programu, dostosowując~go~do~wytycznych twórców języka \cite{Dart_Format}. Dzięki określeniu oficjalnego stylu formatowania kodu trzymające się niego projekty pozostają wizualnie spójne, rozwiązując problem niejednolitego formatowania treści napisanych~przez~różnych deweloperów.

\subsection{Testy oprogramowania}
Testy jednostkowe~są~podstawową metodą weryfikacji działania poszczególnych komponentów programu komputerowego. Dzięki narzędziu \codeinline{flutter_test} możliwe jest pisanie testów~nie~tylko~dla zaimplementowanej logiki,~ale~także dla komponentów bezpośrednio wchodzących~w~interakcje~z~platformą Flutter. Pozwala ona~na~symulowanie zdarzeń takich jak interakcje użytkownika~i~weryfikację poprawnej reakcji programu. 
 
\paragraph{Pokrycie testami}
\label{par:test_coverage}
Narzędzie Round Spot zostało pokryte testami jednostkowymi weryfikującymi jego poprawne działanie. Do~mierzenia pokrycia testami, czyli części kodu źródłowego programu wykonywanego podczas działania testów, używany jest serwis \href{https://codecov.io/}{codecov.io}. Jedną~z~jego zalet jest intuicyjna wizualizacja pokrycia~w~formie wykresu pierścieniowego, zamieszczonego~na~rysunku \ref{fig:rs_coverage}, który przedstawia radialną reprezentację struktury folderów~i~plików~w~projekcie. 

\bigskip
\img[\footnotemark]{\chapterPath/coverage_sunburst.png}{Wykres pierścieniowy \ang{sunburst} pokrycia projektu Round Spot testami}{rs_coverage}{.5}
\footnotetext{\url{https://app.codecov.io/gh/stasgora/round-spot}}

Środkowe koło wykresu symbolizuje główny folder projektu zawierający podfoldery~i~pliki reprezentowane jako kolejne warstwy łuków (patrząc~od~centrum~na~zewnątrz). Łuki tworzące obwód wykresu reprezentują pliki~z~kodem źródłowym projektu. Długość każdego łuku jest proporcjonalna~do~liczby linii kodu zawartych~w~reprezentowanym~przez~niego elemencie (pliku~lub~folderze)~w~stosunku~do~sumy linii kodu~w~całym projekcie. Kolor każdego łuku odpowiada procentowi odpowiadających~mu~linii kodu które~są~pokryte testami jednostkowymi. Rozkład kolorów jest nierównomierny, 90\% pokrycia odpowiada kolorowi żółtemu,~a~już 80\%~to~kolor pomarańczowy. 

Pokrycie projektu~w~aktualnie używanej wersji wynosi ponad 50\% \cite{RS_Coverage}. Duża część nieprzetestowanego jeszcze kodu znajduje się~w~menedżerze tła, który sprawia~pod~tym względem trudności, ponieważ operuje~na~binarnych danych obrazów, przetwarzając~i~łącząc je. W~planach~na~przyszłość jest zwiększenie liczby testów~i~pokrycie nimi całego projektu \ref{sec:future_coverage}.

\subsection{Ciągła integracja}
\label{sec:rs_ci}
Praktyka stosowana~w~trakcie rozwoju oprogramowania, polegająca~na~częstym, regularnym włączaniu (integracji) bieżących zmian~w~kodzie~do~głównego repozytorium~i~każdorazowej weryfikacji zmian, poprzez zbudowanie projektu (jeśli jest taka potrzeba) oraz wykonanie testów jednostkowych \cite{CI_definition}.

Repozytorium GitHub używane~do~przechowywania projektu oferuje wbudowaną platformę~do~ciągłej integracji, GitHub Actions, dostępną~za~darmo dla projektów~o~otwartym kodzie źródłowym \cite{RoundSpot_Actions}. W~tym przypadku została ona skonfigurowana~tak,~aby przy każdej opublikowanej~w~repozytorium zmianie weryfikować brak problemów zgłaszanych~przez~\hyperref[par:static_analysis]{analizę statyczną}~i~poprawne \hyperref[par:dart_format]{sformatowanie} kodu. Uruchamia też testy jednostkowe, weryfikując poprawne działanie poszczególnych komponentów,~po~czym aktualizuje informacje zawarte~na~portalu \href{https://codecov.io/}{codecov.io} \cite{RS_Coverage} nowymi danymi dotyczącymi pokrycia testami. Jeśli którakolwiek~z~wykonywanych czynności~nie~powiedzie się (wykryte zostaną problemy, kod jest niepoprawnie sformatowany~lub~testy kończą się niepowodzeniem), akcja kończy się niepowodzeniem,~a~informacja~o~tym jest wyświetlana~na~stronie głównej repozytorium.
