\section{Specyfika działania}

\subsection{Konfiguracja}
\label{sec:rs_config}
Round Spot posiada wiele parametrów konfiguracyjnych pozwalających na dostosowanie jego działania do potrzeb twórcy aplikacji \cite{RoundSpot_Config_Docs}:
\begin{itemize}
	\item {\bf enabled}: Określa, czy biblioteka jest aktualnie aktywna i zbiera dane.
	\item {\bf uiElementSize}: Umożliwia dostosowanie mechanizmu grupowania interakcji i rysowania map cieplnych do średniego rozmiaru elementów interfejsu użytkownika.
	\item {\bf disabledRoutes}: Zawiera nazwy stron aplikacji, na których nie mają być tworzone mapy cieplne.
	\item {\bf outputTypes}: Określa rodzaje danych wyjściowych które powinny być generowane. Dostępne typy to {\it mapy cieplne} oraz {\it surowe dane}.
	\item {\bf maxSessionIdleTime}: Ustawia czas bezczynności aplikacji, po którym wszystkie bieżące sesje zostaną zamknięte.
	\item {\bf minSessionEventCount}: Ustawia minimalną liczbę zdarzeń potrzebnych do zamknięcia sesji.
	\item {\bf heatMapStyle}: Określa styl generowanych map cieplnych. Dostępne wartości to {\it gładki} oraz {\it warstwowy}.
	\item {\bf heatMapTransparency}: Ustawia przezroczystość warstwy mapy cieplnej rysowanej na tle zdjęcia ekranu. 
	\item {\bf heatMapPixelRatio}: Określa rozdzielczość tworzonych obrazów map cieplnych.
\end{itemize}

Aby umożliwić prostą zdalną zmianę konfiguracji, która może być potrzebna w środowisku produkcyjnym, umożliwiony został mechanizm pozwalający na wczytanie konfiguracji z pliku json, który może być łatwo przesłany na urządzenia użytkowników. Stworzony został także schemat pliku konfiguracyjnego \ang{JSON Schema} pozwalający zweryfikować jego poprawność \cite{RoundSpot_Config_Schema}.

\begin{lstlisting}[language=json,caption={Przykładowy plik konfiguracyjny w formacie json},label=lst:rs_config_json]
{
  "enabled": true,
  "uiElementSize": 10,
  "disabledRoutes": [
    "some-unimportant-route"
  ],
  "outputTypes": [
    "graphicalRender"
  ],
  "session": {
    "minEventCount": 1,
    "maxIdleTime": 60
  },
  "heatMap": {
    "style": "smooth",
    "transparency": 0.6,
    "pixelRatio": 2
  }
}
\end{lstlisting}

\subsection{Sesje}

\img{\chapterPath/banner.png}{Infografika przedstawiająca typy tworzonych sesji}{rs_infographic}{.82}

\paragraph{Mapy ekranów} Każdy detektor umieszczony w interfejsie aplikacji odpowiada jednemu obszarowi z którego zbiera dane. Są one dzielone na okresy czasowe tworząc sesji z których powstają, po ich zakończeniu, pojedyncze mapy cieplne. Istnieją trzy typy sesji. Domyślnie, bez umieszczania dodatkowych detektorów, każdy ekran aplikacji który nie został wskazany w konfiguracji jako ignorowany jest nagrywany w celu tworzenia map cieplnych.

\img{\chapterPath/screen-map.png}{Mapa cieplna ekranu aplikacji}{rs_screen_map}{.3}

\paragraph{Mapy obszarów} Dodatkowo używający biblioteki twórca aplikacji może dodawać detektory monitorujące wybrane części ekranów w celu izolacji zachodzących na nich interakcji i umieszczenia ich na osobnej mapie cieplnej. Przypadkiem w którym jest to niezbędne do otrzymania poprawnego, zgodnego z rzeczywistością zapisu działań użytkownika są przewijane części interfejsu. Bez dodatkowego detektora dotknięcia zarejestrowane na przewijanym fragmencie zostałyby narysowane na mapie cieplnej obejmującej cały ekran. Ponieważ na statycznym, dwuwymiarowym obrazie nie jest możliwe przestawienie niemieszczącej się na ekranie zawartości bez zasłonienia jego części, względne odległości tych interakcji i ich pozycja względem tła byłaby nieprawidłowa. Osobna mapa cieplna obejmująca tylko i wyłącznie obszar przewijany może objąć całą jego powierzchnię i poprawnie narysować wszystkie interakcje użytkownika.

\img{\chapterPath/area-map.png}{Mapa cieplna przewijanych kart}{rs_area_map}{.95}

\paragraph{Mapy kumulowane} Trzeci typ map cieplnych może zawierać interakcje z więcej niż jednego ekranu aplikacji. Jest on użyteczny w sytuacjach, w których celem jest przeanalizowanie skumulowanego użycia elementu interfejsu występującego w tej samej postaci na wielu stronach aplikacji, takiego jak pasek nawigacji lub menu. Dzięki tej możliwości oprócz rozbitych na ekrany interakcji dotyczących wspólnego elementu stworzona zostanie dodatkowa mapa cieplna zbierająca je wszystkie.

\img{\chapterPath/global-map.png}{Mapa cieplna paska nawigacji aplikacji}{rs_global_map}{.7}

\paragraph{Kontrola sesji}
Istnieją trzy mechanizmy powodujące zakończenie sesji i wygenerowanie z nich map cieplnych. Są to wyjście z aplikacji, przekroczenie wyznaczonego czasu bezczynności oraz bezpośrednie całkowite wyłączenia działania. Pierwsze dwa z nich tworzą naturalne przerwy między kolejnymi okresami użycia aplikacji przez użytkowników co czyni je dobrym kryterium podziału interakcji na grupy znajdujące się na pojedynczej mapie cieplnej.

\newcommand*\circled[1]{\tikz[baseline=(char.base)]{\node[shape=circle,draw,inner sep=1.2pt] (char) {#1};}}

\subsection{Tworzenie map cieplnych}
Mapy cieplne są tworzone ze zbioru dwuwymiarowych punktów znajdujących się w układzie współrzędnych ekranu urządzenia (proces pokazano na rusunku \ref{fig:rs_point_steps}) \circled{1}. Pierwszą operacją która jest na nich wykonywana jest uruchomienie algorytmu klasteryzacji DBSCAN \cite{DBSCAN_Wiki}. Jego wynikiem są zbiory pogrupowanych w zależności od odległości punktów \circled{2}. Punkty w grupie ulegają rozmyciu \circled{3} i po połączeniu tworzą podstawowy kształt graficznej reprezentacji grupy \circled{4}. Kształty grup są rysowane wielokrotnie z użyciem zmieniających się parametrów (opisanych dokładnie w kolejnym akapicie) \circled{5} i po nałożeniu na siebie tworzą końcowy wygląd punktów cieplnych umieszczanych na mapie \circled{6}.

\img{\chapterPath/point_steps.png}{Proces tworzenia punktów cieplnych z interakcji użytkowników}{rs_point_steps}{.7}

Ilość punktów znajdujących się w przetwarzanej grupie w stosunku do najliczniejszej grupy jest głównym parametrem odpowiedzialnym za jej wygląd. Im liczniejsza jest grupa tym więcej razy zostanie powtórzony jej kształt, osiągając coraz większy rozmiar. Kolory kolejnych rysowanych warstw są wybierane ze spektrum pomiędzy barwą czerwoną a niebieską z uwzględnieniem liczności grupy. Na przykładzie zamieszczonym na rysunku \ref{fig:rs_point_colors} najliczniejsza grupa ma $6$ punktów i będzie używać całego dostępnego spektrum podczas gdy grupa posiadająca $3$ punkty tylko połowy z niego. Dzięki temu na mapach cieplnych oddana jest względna liczność każdego z punktów cieplnych. Wszystkie warstwy są rysowane z efektem rozmycia i nakładane na siebie. 

\bigskip
\img{\chapterPath/point_colors.png}{Spektrum kolorów używane w zależności od liczności grupy}{rs_point_colors}{.7}

\section{Wyzwanie trybu przetwarzania danych}

\subsection{Przetwarzanie lokalne}
Pierwszy zaimplementowany system wykonywał całe przetwarzanie zebranych danych na mapy cieplne na telefonie użytkownika. Gotowe obrazki były wysyłane i składowane w chmurze. To podejście zostało wybrane ze względu na swoją prostotę, jednak po pierwszych testach okazało się nieodpowiednie. Podczas jednej wizyty w aplikacji  użytkownik odwiedza średnio parę stron, na każdej wykonując parę (często tylko jedną) interakcję. Z każdej wizyty tworzone było więc parę prawie pustych map cieplnych. Podczas pierwszych testów stało się jasne że to podejście charakteryzuje się stanowczo zbyt dużym rozdrobnieniem danych. Przeprowadzenie ewaluacji używając przetwarzania lokalnego spowodowałoby powstanie setek mało interesujących obrazków których nie sposób byłoby przeanalizować.

\subsection{Przetwarzanie zdalne}
Rozwiązaniem tego problemu było oddzielenie części przetwarzającej dane na mapy cieplne (procesora graficznego) do osobnego programu tworząc aktualną architekturę narzędzia \ref{fig:round_spot_diagram}. Zamiast gotowych obrazków do chmury są wysyłane zserializowane surowe dane o ekranie i działaniach użytkownika. Dzięki temu wszystkie interakcje na danym ekranie zebrane podczas ewaluacji mogą być umieszczone na pojedynczej mapie cieplnej. Takie podejście daje też dużo większą swobodę i kontrolę nad postacią danych wyjściowych. Mogą być one dowolnie grupowane i dzielone w celu uzyskania jak największej ilości informacji i wniosków.
