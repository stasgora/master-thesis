\subsection{Architektura}
Narzędzie składa się~z~dwóch części. Do~aplikacji mobilnej dołączana jest wtyczka odpowiedzialna~za~zbieranie~i~wstępne przetwarzanie danych. Pakiety~od~poszczególnych użytkowników~są~serializowane~i~zapisywane~w~chmurze. Drugim komponentem jest procesor danych, który~po~pobraniu zebranych danych deserializuje je, grupuje~ze~względu~na~ekran, którego dotyczą~i~przetwarza~na~mapy cieplne. Diagram architektury został przedstawiony~na~rysunku \ref{fig:round_spot_diagram}.
\bigskip
\img{\chapterPath/RoundSpot_Architecture.png}{Round Spot - Diagram architektury}{round_spot_diagram}{.9}

\paragraph{Aplikacja}
Twórca aplikacji dokonuje integracji narzędzia poprzez jego inicjalizację,~w~ramach której podawane~są~opcjonalne parametry konfiguracyjne. Ponadto~z~uwagi~na~aktualne ograniczenia wymagane jest ręczne umieszczenie detektorów~w~odpowiednich miejscach~w~interfejsie aplikacji.

\paragraph{Flutter} Platforma,~na~której budowane~są~wspierane aplikacje. Obsługuje ona zdarzenia takie jak zmiana ekranu, wyjście~z~aplikacji,~a~także interakcje użytkownika, które~są~przekazywane odpowiednio~do~obserwatorów stanu oraz detektorów.

\paragraph{Detektory} 
\label{par:rs_detectors}
Niewidoczne dla użytkownika elementy interfejsu. Są~odpowiedzialne~za~zbieranie przetwarzanych później danych, takich jak interakcje użytkownika~i~częściowe obrazy tła. Każdy~z~detektorów zbiera dane~z~fragmentu ekranu, którego jest przodkiem~w~drzewie elementów interfejsu.

\paragraph{Obserwatory stanu} 
\label{par:rs_observers}
Ich zadaniem jest informowanie menedżera sesji~o~wydarzeniach takich jak zmiana ekranu~lub~wyjście~z~aplikacji.

\paragraph{Sesja}
\label{par:rs_session}
Ograniczony czasowo zbiór danych, odpowiadający konkretnemu obszarowi interfejsu, zawierający wykonane~w~nim~przez~użytkownika interakcje oraz zrzut ekranu. Każda sesja posiada tekstowy identyfikator jednoznacznie identyfikujący obszar ekranu, którego dotyczy.

\paragraph{Menedżer sesji} 
\label{par:rs_session_manager}
Centralny komponent odpowiedzialny zbieranie danych~do~aktualnie tworzonych map cieplnych oraz koordynację działania pozostałych komponentów. Decyduje,~na~podstawie otrzymanej konfiguracji, kiedy każda~z~sesji powinna się zakończyć~i~przekazuje zebrane~w~ramach niej dane~do~serializacji. Informuje też menedżera tła~o~zachodzących zdarzeniach, takich jak przewijanie ekranu.

\paragraph{Menedżer tła} 
\label{par:rs_bg_manager}
Tworzy obrazy interfejsu aplikacji będące tłem dla tworzonych map cieplnych, polegając~na~sygnałach otrzymanych~od~menedżera sesji. W~przypadku obszaru przewijanego wynikowy obraz jest efektem złączenia wielu  obrazów cząstkowych zbieranych~w~miarę wyświetlania się~na~ekranie kolejnych fragmentów interfejsu.

\paragraph{Serializacja danych}
Przygotowuje aktualnie zebrany zestaw danych~do~wysłania poprzez przetworzenie~go~do~postaci pliku binarnego. Jako protokół serializacji została użyta biblioteka {\it Protocol Buffers} stworzona~przez~Google.

\paragraph{Google Cloud Storage}
Pakiety danych~ze~wszystkich urządzeń~są~wysyłane~do~jednej lokalizacji, umożliwiając ich późniejsze pobranie~i~dalsze przetwarzanie. Do~składowania danych została użyta usługa {\it Firebase Storage} zbudowana~na~{\it Google Cloud Storage}.

\paragraph{Deserializacja~i~agregacja}
Po pobraniu paczki~są~deserializowane~z~użyciem biblioteki {\it Protocol Buffers}. Następnie otrzymane dane podlegają agregacji służącej połączeniu informacji zebranych~ze~wszystkich urządzeń~i~pogrupowaniu ich~ze~względu~na~ekran, którego dotyczą.

\paragraph{Procesor graficzny} 
\label{par:rs_graphical_processor}
Przetwarza zgrupowane interakcje~do~graficznej postaci mapy cieplnej nałożonej~na~obraz zawartości ekranu. Interakcje położone blisko siebie~są~grupowane~za~pomocą algorytmu DBSCAN \cite{DBSCAN_Wiki}~a~ich liczba~i~rozłożenie~w~grupie definiuje jej intensywność~i~kształt~na~wynikowej mapie.
