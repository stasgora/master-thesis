\section{Przygotowanie~do~wydania}

\subsection{Dokumentacja}
Każdy opublikowany pakiet powinien posiadać dokumentację ułatwiającą jego użycie. Twórcy języka Dart udostępnili wytyczne standaryzujące sposób pisania dokumentacji~w~plikach źródłowych \cite{Dart_Doc_Guidelines}. Komentarze zaczynające się~od~potrójnego znaku \codeinline{///} zawierają opisy elementów~i~struktur, które poprzedzają. Istnieje też możliwość formatowania ich zawartości (np. \codeinline{_kursywa_}), odwoływania się~do~innych fragmentów kodu (\codeinline{[Klasa.pole]}) oraz zamieszczania przykładów (\codeinline{`kod`} oraz \codeinline{```wielolinijkowy fragment kodu```}).

\bigskip
\begin{lstlisting}[language=dartcomment,caption={Fragment dokumentacji zawartej~w~kodzie źródłowym pakietu},label=lst:rs_docs]
/// Initializes the _Round Spot_ library.
///
/// Takes a [child] widget, an optional [config],
/// a [loggingLevel] which defaults to [LogLevel.off]
/// and output callbacks ([localRenderCallback] and [dataCallback]) 
/// that must be set depending on the [Config.outputType] requested.
///
/// Should be invoked in `main()` or otherwise wrap the [MaterialApp] widget:
/// ```dart
/// void main() {
///   runApp(initialize(
///     child: Application()
///   ));
/// }
/// ```
\end{lstlisting} 


Aby umożliwić korzystanie~z~dokumentacji~w~wygodnej~i~interaktywnej formie stworzone zostało narzędzie \codeinline{dartdoc} \cite{Dart_Doc} przetwarzające komentarze~z~plików kodu źródłowego projektu~do~postaci strony internetowej. Przetwarzana treść jest formatowana,~a~w~miejsce odwołań~są~wstawiane linki. Wynik przetwarzania listingu \ref{lst:rs_docs} znajduje się~na~rysunku \ref{fig:rs_html_docs}. Powtarzające się fragmenty mogą zostać wstawione~w~szablony, których treść zostanie skopiowana~we~wskazane miejsca podczas przetwarzania. Podczas publikacji pakietu~w~repozytorium \href{https://pub.dev/}{pub.dev} strony~z~dokumentacją~są~automatycznie tworzone~i~umieszczane~w~serwisie, zapewniając łatwy dostęp każdemu, kto chce użyć pakietu \cite{RS_Documentation}.

\bigskip
\img{\chapterPath/rs_docs.png}{Fragment przetworzonej~na~stronę internetową dokumentacji}{rs_html_docs}{.9}

\subsection{Przykład użycia}
\label{sec:rs_example}
Sugeruje się, żeby każdy opublikowany pakiet posiadał dołączony przykład ilustrujący sugerowany sposób jego użycia. Powinien być jak najprostszy, przy czym jednocześnie wyraźnie pokazywać funkcjonalność oferowaną~przez~zawierający~go~pakiet. W~tym przypadku przykład zawiera prostą aplikację~z~dwoma ekranami~i~przewijaną listą wraz~z~instrukcją jak uruchomić kod~i~uzyskać pliki map cieplnych używając lokalnego przetwarzania danych \cite{RS_Example}.

\subsection{Repozytorium kodu}
Kod źródłowy projektu jest publicznie dostępny~w~serwisie GitHub \cite{RoundSpot_GitHub}. Serwis został wybrany~z~powodu popularności~i~liczby oferowanych funkcji. Akcje umożliwiają stworzenie procesu \hyperref[sec:rs_ci]{ciągłej integracji} opartego~o~konteneryzację. Zakładka problemów \ang{Issues} pozwala każdemu~na~zadawanie pytań~i~zgłaszanie uwag podczas gdy projekty oferują prostą tablicę typu Kanban ułatwiającą organizację zadań. Na~stronie głównej projektu wyświetlany jest plik informacyjny \codeinline{README.md} zawierający podstawowe informacje~i~pierwsze kroki. Kod udostępniony jest~na~otwartej licencji MIT.

\subsection{Pub.dev}
Oficjalne repozytorium otwartych~i~darmowych pakietów stworzonych~w~języku Dart oraz~na~platformie Flutter. Na~głównej stronie pakietu (rysunek \ref{fig:rs_pub_dev}) wyświetlony jest jego plik \codeinline{README.md}. Pozostałe zakładki zawierają między innymi historię zmian (braną~z~pliku \codeinline{CHANGELOG.md}), dodany \hyperref[sec:rs_example]{przykład} oraz instrukcje instalacji. Po~prawej stronie znajduje się kolumna zawierająca dodatkowe informacje~i~linki~do~repozytorium, dokumentacji~i~licencji \cite{RS_Pub_dev}.

\bigskip
\img{\chapterPath/rs_pub_dev.png}{Strona główna pakietu~w~repozytorium pub.dev}{rs_pub_dev}{1}

\paragraph{Ocena} Każdy pakiet jest automatycznie oceniany~w~skali 0-130. Na~ocenę wpływ~ma~między innymi wynik działania \hyperref[par:static_analysis]{analizy statycznej}, obecność dokumentacji, wsparcie dostępnych platform oraz najnowszych funkcji języka.
