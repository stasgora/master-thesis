Wykrywanie aktywności~i~zachowań ludzkich~z~pomocą technologii jest zagadnieniem szerokim~i~wielowarstwowym. W~celu pogłębienia swojej wiedzy, zapoznania się~z~osiągnięciami naukowymi~w~tej dziedzinie oraz zebrania materiałów~do~pracy został przeprowadzony \textit{Systematyczny Przegląd Literatury}. Bazował~on~na~publikacjach naukowych zebranych~z~pomocą trzech wyszukiwarek: \textit{Web of Science}, \textit{IEEE Explore} oraz \textit{PubMed}. 

Trzystopniowa selekcja pozwoliła~na~wyizolowanie najciekawszych~i~najlepiej pasujących~do~postawionego pytania artykułów. Wybrane publikacje zostały, zależnie~od~ich oceny, przeczytane~lub~przejrzane. Jednocześnie tworzone były interesujące~z~punktu widzenia tego przeglądu statystyki. 

Dużym zaskoczeniem~po~zapoznaniu się zebranymi materiałami był fakt~że~pomimo dużej powierzchownej  różnorodności większość~z~opisywanych eksperymentów skupiało się~na~wykrywaniu tego samego,~lub~bardzo podobnego zbioru modalności zawierającego zwykle chodzenie, siedzenie, stanie oraz leżenie. Ze~względu~na~ich charakterystykę najodpowiedniejszy~do~ich wykrywania jest akcelerometr, który~z~tego powodu okazał się najczęściej wykorzystywanym sensorem.

Mimo~to~dzięki przeprowadzeniu przeglądu udało się zidentyfikować parę oryginalnych publikacji które przyczyniły się~do~wzbogacenia tej pracy poprzez swoje nowatorskie podejście~do~tematu monitorowania aktywności~i~zachowań ludzkich.

Główną limitacją zakresu przeprowadzonego przeglądu była liczba zaangażowanych uczestników. Oprócz trzeciego etapu selekcji artykułów który został wykonany~przez~dr hab. inż. Agnieszkę Landowską całość została wykonana~przez~autora pracy. Ten fakt znacząco ograniczył ilość publikacji które mogły zostać przestudiowane~po~etapie selekcji. Z~648 artykułów które zostały wstępnie zidentyfikowane jako pasujące~do~zadanego pytania~tylko~ 32 zostało przeczytanych~lub~przeglądniętych~a~co~za~tym idzie włączonych~do~analizy przedstawionej~w~tym rozdziale.
