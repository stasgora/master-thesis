\section{Analiza ilościowa}
Z pośród wyselekcjonowanych artykułów 12 zostało sklasyfikowanych jako \textit{Przegląd teoretyczny} lub \textit{Meta-analiza} podczas gdy pozostałe 20 zawierało opis przeprowadzonego przez autorów eksperymentu. Niektóre z artykułów zostały zaliczone do więcej niż jednej kategorii. Połowa przeglądów zawierała jednocześnie meta-analizę zbioru opublikowanych eksperymentów.
\centertable{
	\hline \textbf{Klasyfikacja} & \textbf{Przegląd teoretyczny} & \textbf{Meta-analiza} & \textbf{Eksperyment} \\
	\hline {\bf 2} & 12 & 8 & 7 \\
	\hline {\bf 1} & - & - & 13 \\
	\hline\hline {\bf Suma} & 12 & 8 & 20 \\
	\hline
}{|c | c | c | c|}{Podział na typy artykułów}{area_clasification}

Pośród artykułów o charakterze teoretycznym często wymieniane są zastosowania z kategorii opieki medycznej takie jak monitorowania stanu zdrowia \cite{22_HAR_Survey_Ultrasonic} lub wspomaganie rehabilitacji \cite{35_HAR_Wearable_Review}. Popularym tematem jest też asysta osób starszych, szczególnie tych mieszkających samotnie \cite{21_HAR_Smartphone}. Proponowane rozwiązania, określane najczęściej jako \textit{``Ambient Assisted Living''} mogą pomóc w przypadku problemu lub wypadku poprzez zawiadomienie odpowiednich osób lub służb.

Analiza zachowania użytkowników jest tematem większości publikacji zawierających eksperymenty \cite{2_Real_Time_HAR, 59_Air_Pressure_HAR, 33_Inertial_Study}. Jej popularność jest spowodowana wysoką dostępnością telefonów zawierających nadające się do tego celu sensory oraz dużą ilością opublikowanych już na ten temat prac i materiałów obniżających barierę wejściową.

Spośród pozostałych zastosowań wyróżniają się te współtworzące inteligentny dom, jak na przykład sterowanie oświetleniem zależnie od wykrytej aktywności \cite{36_Smart_Home_HAR} a także te związane z wykrywaniem aktywności sportowych \cite{29_Daily_Sport_HAR} które można by zintegrować z istniejącymi platformami takimi jak \textit{Strava}.

\centertable{
	\hline \textbf{Zastosowanie} & \textbf{Przegląd/Analiza} & \textbf{Eksperyment} & \textbf{Suma} \\
	\hline Analiza zachowania & 8 & 12 & 20 \\
	\hline Opieka zdrowotna & 8 & 4 & 12 \\
	\hline Asysta osób starszych & 7 & 3 & 10 \\
	\hline Sport & 5 & 4 & 9 \\
	\hline Bezpieczeństwo & 7 & 0 & 7 \\
	\hline Monitorowanie pacjentów & 4 & 2 & 6 \\
	\hline Inteligentny dom & 4 & 2 & 6 \\
	\hline Rozrywka & 3 & 0 & 3 \\
	\hline
}{|l | c | c || c|}{Najczęściej pojawiające się zastosowania}{uses_analysis}

Akcelerometr, magnetometr oraz żyroskop, występujące często razem jako tak zwany \textit{``Inertial measurement unit''} są najpopularniejszym zestawem wybieranym przez autorów artykułów \cite{30_Context_Awareness, 32_Accel_Phone_HAR}. Nadają się one dobrze do wykrywania aktywności ruchowych, zarówno przemieszczania jak i wykonywanych gestów i ruchów ciała pod warunkiem rozmieszczenia wielu sensorów na ciele osoby monitorowanej, jak w \cite{29_Daily_Sport_HAR}.

Mikrofon jest sensorem o szerokim zastosowaniu. Jest używany w środowiskach domowych do zbierania dźwięków z otoczenia i identyfikowania na ich podstawie czynności takich jak chodzenie, stukanie, otwieranie / zamykanie drzwi lub odkurzanie \cite{46_Indor_Audio_Rec}. Jednym z ciekawszych przykładów jego wykorzystania jest emitowanie ultradźwiękowych sygnałów i nasłuchiwanie echa głośnikiem \cite{22_HAR_Survey_Ultrasonic}. Ta technika która leży u podstawy działania sonaru pozwala na wykrycie położenia i względnego ruchu pobliskich przedmiotów poprzez wykorzystanie efektu Dopplera.

Przykłady wykorzystania pulsometra pojawiają się zarówno w obszarze medyczno-zdrowotnym \cite{62_The_Long_Review} przy monitorowaniu parametrów życiowych jak i rekreacyjno-sportowym \cite{51_Fixbit_tracker} przy zapisie aktywności fizycznej.
\centertable{
	\hline \textbf{Sensor} & \textbf{Przegląd/Analiza} & \textbf{Eksperyment} & \textbf{Suma} \\
	\hline Akcelerometr & 10 & 15 & 25 \\
	\hline Mikrofon & 8 & 7 & 15 \\
	\hline Magnetometr & 7 & 7 & 14 \\
	\hline Żyroskop & 7 & 7 & 14 \\
	\hline Kamera & 9 & 2 & 11 \\
	\hline GPS & 6 & 3 & 9 \\
	\hline Pulsometr & 6 & 2 & 8 \\
	\hline Termometr & 7 & 0 & 7 \\
	\hline Radio/RFID & 6 & 1 & 7 \\
	\hline Czujnik światła & 6 & 0 & 6 \\
	\hline WiFi & 4 & 1 & 5 \\
	\hline Bluetooth & 4 & 1 & 5 \\
	\hline Barometr & 4 & 1 & 5 \\
	\hline
}{|l | c | c || c|}{Najczęściej wykorzystywane sensory}{sensor_analysis}

Zdecydowanie najpopularniejszymi modalnościami są proste aktywności ruchowe. Zaliczają się do nich zwykle:
\begin{itemize}
    \item Chodzenie
    \item Stanie
    \item Bieganie
    \item Siedzenie
    \item Leżenie
    \item Wchodzenie, Schodzenie po schodach
    \label{base_modalities}
\end{itemize}

Powyższa lista może lekko różnić się od eksperymentu do eksperymentu jednak stanowi dobrą przekrojową reprezentację najczęściej wykrywanych modalności ruchowych. Aktywności te są łatwo wykrywalne za pomocą akcelerometru, czasami połączonego z żyroskopem i magnetometrem, co wpływa na popularność wykorzystania tych sensorów (\ref{tab:sensor_analysis}).

\subsection{Poziomy modalności}
Duża część eksperymentów skupia się na rozpoznawaniu prostych, ``niskopoziomowych'' modalności. Charakteryzują się one krótkim czasem trwania, niską złożonością i prostym mechanizmem wykrywania. Kolejnym poziomem abstrakcji, którego podejmuje się niewielka część przestudiowanych publikacji, jest złożenie tych prostych modalności i próba wywnioskowania na ich postawie bardziej złożonych modalności.

Przykładowo przy określaniu lokalizacji użytkownika za proste modalności można uznać przebywanie w jednym miejscu oraz przemieszczanie się. Mogą tu także zostać zebrane dane takie jak prędkość poruszania lub pokonana droga. Za wyższy poziom modalności może zostać uznane określenie celu podróży oraz przypisanie  nazwy miejscu przebywania odpowiedniej dla danej osoby.

Sensory zawarte w smartphonach dają duże możliwości monitorowania i analizy wzorców przemieszczania się ludzi w miastach. Na podstawie danych GPS, GSM oraz WiFi zebranych od ponad 1000 uczestników przez \textit{``Future Mobility Survey''} \cite{26_Mobility_Sensing} prezentuje kategoryzację miejsc przebywania i przemieszczania rozpoznając między innymi:
\begin{itemize}
    \item Dom
    \item Pracę
    \item Edukację
    \item Odbiór / transport
    \item Posiłek
    \item Zakupy
    \item Wizytę lekarską
    \item Spotkanie
    \item Rekreację
    \item Odwiedziny
    \item Rozrywkę
\end{itemize}

W \cite{38_High_Lvl_HAR} zaprezentowane jest wykrywanie modalności wyższego poziomu w środowisku domowym takich jak picie kawy, jedzenie kanapki oraz sprzątanie pokoju. Na każdą z tych czynności składa się seria ruchów, jak na przykład krojenie chleba lub otworzenie lodówki które muszą zostać zidentyfikowane. W tym celu śledzone są ruchy kończyn. Autorzy dochodzą do wniosku że łatwiej wykryć jest modalności które nie wymagają skomplikowanych interakcji z otoczeniem, takie jak relaks lub poruszanie po pomieszczeniu.

\centertable{
	\hline \textbf{Modalność} & \textbf{Przegląd/Analiza} & \textbf{Eksperyment} & \textbf{Suma} \\
	\hline Aktywność ruchowa & 11 & 16 & 27 \\
	\hline Ruch ciała & 7 & 11 & 18 \\
	\hline Poza ciała & 9 & 4 & 13 \\
	\hline Wykonywane gesty & 4 & 2 & 6 \\
	\hline Podnoszenie przedmiotów & 2 & 2 & 4 \\
	\hline Bicie serca & 1 & 3 & 4 \\
	\hline Jedzenie & 1 & 3 & 4 \\
	\hline Picie & 2 & 2 & 4 \\
	\hline Ruch rąk / ramion & 2 & 2 & 4 \\
	\hline
}{|l | c | c || c|}{Najczęściej obserwowane modalności}{modality_analysis}


\centertable{
	\hline \textbf{Metryka} & \textbf{Przegląd/Analiza} & \textbf{Eksperyment} & \textbf{Suma} \\
	\hline Standard Deviation & 5 & 10 & 15 \\
	\hline Min, Max & 4 & 7 & 11 \\
	\hline Mean, Median & 3 & 8 & 11 \\
	\hline Signal Magnitude Area & 2 & 5 & 7 \\
	\hline Mean Absolute Deviation & 3 & 3 & 6 \\
	\hline Variance & 2 & 3 & 5 \\
	\hline Corelation Coefficients & 2 & 3 & 5 \\
	\hline Interquartile Range & 2 & 3 & 5 \\
	\hline Signal Entropy & 2 & 3 & 5 \\
	\hline Root Mean Square & 2 & 2 & 4 \\
	\hline
}{|l | c | c || c|}{Najczęściej używane metryki - {\bf Akcelerometr}}{accl_features}

\centertable{
	\hline \textbf{Metryka} & \textbf{Eksperyment} & \textbf{Suma} \\
	\hline Step frequency & 2 & 2 \\
	\hline Power Spectral Density & 2 & 2 \\
	\hline Variance & 1 & 1 \\
	\hline FFT Peaks \& Energy & 1 & 1 \\
	\hline Trunk inclination & 1 & 1 \\
	\hline
}{|l | c || c|}{Najczęściej używane metryki - {\bf Pulsometr}}{heart_rate_features}
