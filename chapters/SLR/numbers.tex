\section{Analiza ilościowa}
Z pośród wyselekcjonowanych artykułów 12 zostało sklasyfikowanych jako \textit{Przegląd teoretyczny} lub \textit{Meta-analiza} podczas gdy pozostałe 20 zawierało opis przeprowadzonego przez autorów eksperymentu. Niektóre z artykułów zostały zaliczone do więcej niż jednej kategorii. Połowa przeglądów zawierała jednocześnie meta-analizę zbioru opublikowanych eksperymentów.
\centertable{
	\hline \textbf{Klasyfikacja} & \textbf{Przegląd teoretyczny} & \textbf{Meta-analiza} & \textbf{Eksperyment} \\
	\hline {\bf 2} & 12 & 8 & 7 \\
	\hline {\bf 1} & - & - & 13 \\
	\hline\hline {\bf Suma} & 12 & 8 & 20 \\
	\hline
}{|c | c | c | c|}{Podział na typy artykułów}{area_clasification}


\centertable{
	\hline \textbf{Zastosowanie} & \textbf{Przegląd/Analiza} & \textbf{Eksperyment} & \textbf{Suma} \\
	\hline Analiza zachowania & 8 & 12 & 20 \\
	\hline Opieka zdrowotna & 8 & 4 & 12 \\
	\hline Asysta osób starszych & 7 & 3 & 10 \\
	\hline Sport & 5 & 4 & 9 \\
	\hline Bezpieczeństwo & 7 & 0 & 7 \\
	\hline Monitorowanie pacjentów & 4 & 2 & 6 \\
	\hline Inteligentny dom & 4 & 2 & 6 \\
	\hline Rozrywka & 3 & 0 & 3 \\
	\hline
}{|l | c | c || c|}{Najczęściej pojawiające się zastosowania}{uses_analysis}

\centertable{
	\hline \textbf{Sensor} & \textbf{Przegląd/Analiza} & \textbf{Eksperyment} & \textbf{Suma} \\
	\hline Akcelerometr & 10 & 15 & 25 \\
	\hline Mikrofon & 8 & 7 & 15 \\
	\hline Magnetometr & 7 & 7 & 14 \\
	\hline Żyroskop & 7 & 7 & 14 \\
	\hline Kamera & 9 & 2 & 11 \\
	\hline GPS & 6 & 3 & 9 \\
	\hline Pulsometr & 6 & 2 & 8 \\
	\hline Termometr & 7 & 0 & 7 \\
	\hline Radio/RFID & 6 & 1 & 7 \\
	\hline Czujnik światła & 6 & 0 & 6 \\
	\hline WiFi & 4 & 1 & 5 \\
	\hline Bluetooth & 4 & 1 & 5 \\
	\hline Barometr & 4 & 1 & 5 \\
	\hline
}{|l | c | c || c|}{Najczęściej wykorzystywane sensory}{sensor_analysis}

\centertable{
	\hline \textbf{Modalność} & \textbf{Przegląd/Analiza} & \textbf{Eksperyment} & \textbf{Suma} \\
	\hline Aktywność ruchowa & 11 & 16 & 27 \\
	\hline Ruch ciała & 7 & 11 & 18 \\
	\hline Poza ciała & 9 & 4 & 13 \\
	\hline Wykonywane gesty & 4 & 2 & 6 \\
	\hline Podnoszenie przedmiotów & 2 & 2 & 4 \\
	\hline Bicie serca & 1 & 3 & 4 \\
	\hline Jedzenie & 1 & 3 & 4 \\
	\hline Picie & 2 & 2 & 4 \\
	\hline Ruch rąk / ramion & 2 & 2 & 4 \\
	\hline
}{|l | c | c || c|}{Najczęściej obserwowane modalności}{modality_analysis}

Zdecydowanie najpopularniejszymi modalnościami są proste aktywności ruchowe. Zaliczają się do nich zwykle:
\begin{itemize}
    \item Chodzenie
    \item Stanie
    \item Bieganie
    \item Siedzenie
    \item Leżenie
    \item Wchodzenie, Schodzenie po schodach
\end{itemize}

Powyższa lista może lekko różnić się od eksperymentu do eksperymentu jednak stanowi dobrą przekrojową reprezentację najczęściej wykrywanych modalności ruchowych. Aktywności te są łatwo wykrywalne za pomocą akcelerometru, czasami połączonego z żyroskopem i magnetometrem, co wpływa na popularność wykorzystania tych sensorów (\ref{tab:sensor_analysis}).

\centertable{
	\hline \textbf{Metryka} & \textbf{Przegląd/Analiza} & \textbf{Eksperyment} & \textbf{Suma} \\
	\hline Standard Deviation & 5 & 10 & 15 \\
	\hline Min, Max & 4 & 7 & 11 \\
	\hline Mean, Median & 3 & 8 & 11 \\
	\hline Signal Magnitude Area & 2 & 5 & 7 \\
	\hline Mean Absolute Deviation & 3 & 3 & 6 \\
	\hline Variance & 2 & 3 & 5 \\
	\hline Corelation Coefficients & 2 & 3 & 5 \\
	\hline Interquartile Range & 2 & 3 & 5 \\
	\hline Signal Entropy & 2 & 3 & 5 \\
	\hline Root Mean Square & 2 & 2 & 4 \\
	\hline
}{|l | c | c || c|}{Najczęściej używane metryki - {\bf Akcelerometr}}{accl_features}

\centertable{
	\hline \textbf{Metryka} & \textbf{Eksperyment} & \textbf{Suma} \\
	\hline Step frequency & 2 & 2 \\
	\hline Power Spectral Density & 2 & 2 \\
	\hline Variance & 1 & 1 \\
	\hline FFT Peaks \& Energy & 1 & 1 \\
	\hline Trunk inclination & 1 & 1 \\
	\hline
}{|l | c || c|}{Najczęściej używane metryki - {\bf Pulsometr}}{heart_rate_features}
