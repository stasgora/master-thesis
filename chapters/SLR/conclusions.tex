\section{Raport}

\subsection{Poziomy modalności}
Duża część eksperymentów skupia się na rozpoznawaniu prostych, ``niskopoziomowych'' modalności. Charakteryzują się one krótkim czasem trwania, niską złożonością i prostym mechanizmem wykrywania. Kolejnym poziomem abstrakcji, którego podejmuje się niewielka część przestudiowanych publikacji, jest złożenie tych prostych modalności i próba wywnioskowania na ich postawie bardziej złożonych modalności.

Przykładowo przy określaniu lokalizacji użytkownika za proste modalności można uznać przebywanie w jednym miejscu oraz przemieszczanie się. Mogą tu także zostać zebrane dane takie jak prędkość poruszania lub pokonana droga. Za wyższy poziom modalności może zostać uznane określenie celu podróży oraz przypisanie  nazwy miejscu przebywania odpowiedniej dla danej osoby.

Sensory zawarte w smartphonach dają duże możliwości monitorowania i analizy wzorców przemieszczania się ludzi w miastach. Na podstawie danych GPS, GSM oraz WiFi zebranych od ponad 1000 uczestników przez \textit{``Future Mobility Survey''} \cite{26_Mobility_Sensing} prezentuje kategoryzację miejsc przebywania i przemieszczania rozpoznając między innymi:
\begin{itemize}
    \item Dom
    \item Pracę
    \item Edukację
    \item Odbiór / transport
    \item Posiłek
    \item Zakupy
    \item Wizytę lekarską
    \item Spotkanie
    \item Rekreację
    \item Odwiedziny
    \item Rozrywkę
\end{itemize}

W \cite{38_High_Lvl_HAR} zaprezentowane jest wykrywanie modalności wyższego poziomu w środowisku domowym takich jak picie kawy, jedzenie kanapki oraz sprzątanie pokoju. Na każdą z tych czynności składa się seria ruchów, jak na przykład krojenie chleba lub otworzenie lodówki które muszą zostać zidentyfikowane. W tym celu śledzone są ruchy kończyn. Autorzy dochodzą do wniosku że łatwiej wykryć jest modalności które nie wymagają skomplikowanych interakcji z otoczeniem, takie jak relaks lub poruszanie po pomieszczeniu.
