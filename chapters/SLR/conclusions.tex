\pagebreak
\section{Dyskusja}
Przeprowadzenie systematycznego przeglądu literatury pozwoliło na szczegółową analizę procesów wykorzystywanych do identyfikowania modalności. W jego trakcie udało się zidentyfikować popularne cele zastosowania technik analizy aktywności jak i najczęściej wykorzystywane do tego celu sensory, modalności i metryki. Jednocześnie uwidocznione zostały obszary w których nadal brakuje obszernych i szczegółowych badań i eksperymentów.

\subsection{Kompletność informacji}
Część opublikowanych opisów eksperymentów skupiała się nieinteresujących z punktu wiedzenia tego przeglądu aspektach wykrywania aktywności, takich jak używanych klasyfikatorach i algorytmach uczenia maszynowego. Równocześnie ponad 1/3 (34\%) z przeglądanych artykułów pomija zupełnie temat metryk używanych w procesie wykrywania aktywności pomimo ważnej roli którą w nim pełnią. Braki te mają dużego wpływu na wyniki i wnioski niniejszego przeglądu, jednak należy zaznaczyć że z ich względu zebrane i przedstawione tu informacje nie są w pełni kompletne i nie przedstawiają pełnego obrazu przestudiowanych eksperymentów. 

\subsection{Równowaga tematów}
\label{sec:article_balance}
Użycie akcelerometru do przeprowadzenia analizy zachowania poprzez wykrycie aktywności ruchowych jest podejściem które pojawia się w przeważającej części przeglądanej literatury, m.in. \cite{59_Air_Pressure_HAR, 33_Inertial_Study, 29_Daily_Sport_HAR, 30_Context_Awareness, 32_Accel_Phone_HAR, 42_Micro_AR}. Jest to z pewnością częściowo spowodowane wspomnianą już łatwością uzyskania efektów przy wybraniu tej popularnej kombinacji sensorów i modalności. Wiadomym jest że łatwiej budować na wcześniejszych osiągnięciach w danej dziedzinie niż być innowatorem stawiającym w niej pierwsze kroki. 

Mimo to tak duży zbiór publikacji na taki temat stanowi wyczerpujący zbiór informacji umożliwiający wybranie najefektywniejszego podejścia do danego problemu bez konieczności przeprowadzania czasochłonnej i kosztownej analizy. Stanowi też solidną bazę do zwykle postępującej za osiągnięciami naukowymi szerszej adopcji danego rozwiązania.

\subsection{Poziomy modalności}
\label{sec:modality_levels}
Duża część eksperymentów skupia się na rozpoznawaniu prostych, ``niskopoziomowych'' modalności. Charakteryzują się one krótkim czasem trwania, niską złożonością i prostym mechanizmem wykrywania. Kolejnym poziomem abstrakcji, którego podejmuje się niewielka część przestudiowanych publikacji, jest złożenie tych prostych modalności i próba wywnioskowania na ich postawie bardziej złożonych modalności.

Przykładowo przy określaniu lokalizacji użytkownika za proste modalności można uznać przebywanie w jednym miejscu oraz przemieszczanie się. Mogą tu także zostać zebrane dane takie jak prędkość poruszania lub pokonana droga. Za wyższy poziom modalności może zostać uznane określenie celu podróży oraz przypisanie nazwy miejscu przebywania, odpowiedniej dla danej osoby.Jak napisano we wstępie udało się zidentyfikować parę artykułów eksplorujących tematy wykrywania aktywności wyższego poziomu.

Sensory zawarte w smartphonach dają duże możliwości monitorowania i analizy wzorców przemieszczania się ludzi w miastach. Na podstawie danych GPS, GSM oraz WiFi zebranych od ponad 1000 uczestników przez \textit{``Future Mobility Survey''} \cite{26_Mobility_Sensing} prezentuje kategoryzację miejsc przebywania i przemieszczania się rozróżniając między innymi:
\begin{itemize}
    \item Dom
    \item Pracę
    \item Edukację
    \item Odbiór / transport
    \item Posiłek
    \item Zakupy
    \item Wizytę lekarską
    \item Spotkanie
    \item Rekreację
    \item Odwiedziny
    \item Rozrywkę
\end{itemize}

W \cite{38_High_Lvl_HAR} zaprezentowane jest wykrywanie modalności wyższego poziomu w środowisku domowym takich jak picie kawy, jedzenie kanapki oraz sprzątanie pokoju. Na każdą z tych czynności składa się seria ruchów, jak na przykład krojenie chleba lub otworzenie lodówki które muszą zostać zidentyfikowane. W tym celu śledzone są ruchy kończyn. Autorzy dochodzą do wniosku że łatwiej wykryć jest modalności które nie wymagają skomplikowanych interakcji z otoczeniem, takie jak relaks lub poruszanie po pomieszczeniu.

Ilość przestudiowanych artykułów zajmujących się aktywnościami wyższymi jest szczątkowa. Dodatkowo każdy skupia się na innym obszarze zastosowań. Z powodu braku punktu odniesienia z którym można by porównać wyniki przestawione w tych publikacjach występuje zwiększone ryzyko ich stronniczości. 

\subsection{Różnorodność podejść}
Biorąc pod uwagę niską liczbę publikacji poruszającą problemy związane z wykrywaniem aktywności wyższego poziomu (\ref{sec:modality_levels}) oraz wysoki odsetek nakładających się lub identycznych tematów pośród pozostałych artykułów (\ref{sec:article_balance}) należy zwrócić uwagę na brak różnorodności wśród wyselekcjonowanej grupy publikacji. Zostało zidentyfikowane parę czynników mogących mieć na to wpływ. 

Ograniczenia w ilości publikacji wybranych do przeprowadzenia przeglądu zwiększyły ilość potencjalnie interesujących artykułów które zostały odrzucone. Sama selekcja wyników wyszukiwania musiała zostać przeprowadzona przed zapoznaniem się z ich zawartością co prowadzi do paradoksu w którym niemożliwe jest wybranie najlepszych publikacji bazując na szczątkowych informacjach o ich wartości. W wyniku artykuły o wartościowej treści mogły zostać odrzucone na podstawie niereprezentatywnego tytułu lub abstraktu. Dodatkowym czynnikiem związanym z selekcją wykonywaną w większości przez jedną osobę jest ryzyko stronniczości i zwiększony wpływ subiektywizmu. Wreszcie nie należy wykluczać ewentualności że uzyskany zbiór jest mniej lub bardziej reprezentatywny i przedstawione tu wnioski odnoszą się w miarę dobrze do całości literatury naukowej poświęconej zagadnieniu rozpoznawania aktywności u ludzi.
