\section{Analiza ilościowa}
Publikacje oznaczone jako {\bf 2} podczas selekcji abstraktów zostały przeczytane~a~artykuły które otrzymały ocenę {\bf 2} podczas dodatkowej selekcji - przejrzane. Obie grupy stanowiły bazę~do~wykonanej analizy ilościowej. W~sumie liczba zakwalifikowanych~do~dalszej analizy artykułów wyniosła {\bf 32}.

\subsection{Szukane informacje}
Podczas zapoznawania się~z~wybranymi publikacjami aktywnie szukane~i~spisywane były informacje~z~pięciu kategorii:
\begin{itemize}
    \item {\bf Kategoria}: Typ artykułu. Występują~tu~trzy wartości:
    \begin{itemize}
		\item {\it Przegląd teoretyczny}: Opisuje teoretyczne zagadnienia związane~z~danym tematem. Podaje definicje~i~przykłady.
		\item {\it Meta-analiza}: Jest zbiorem wielu publikacji~na~określony temat. Dokonuje porównania ich założeń, podejścia oraz wyników.
		\item {\it Eksperyment}: Opisuje oryginalny proces wykrywania aktywności używający wybranych sensorów. Zwykle przeprowadza jego testy~na~grupie uczestników.
	\end{itemize}
    \item {\bf Zastosowania}: Ogólne obszary poruszane~w~artykule~do~których należą wykrywane modalności.
    \item {\bf Sensory}: Urządzenia rejestrujące sygnały~z~otoczenia używane~przez~przeprowadzających badanie.
    \item {\bf Modalności}: Podstawowe zdarzenia które~są~wykrywane~w~procesie rozpoznawania aktywności.
    \item {\bf Metryki}: Wzory matematyczne używane~do~transformacji sygnałów przy ich przetwarzaniu.
\end{itemize}

\subsection{Kategoria}
Z pośród wyselekcjonowanych artykułów 12 zostało sklasyfikowanych jako \textit{Przegląd teoretyczny}~lub~\textit{Meta-analiza} podczas gdy pozostałe 20 zawierało opis przeprowadzonego~przez~autorów eksperymentu. 

{\bf Uwaga:} Niektóre~z~artykułów zostały zaliczone~do~więcej niż jednej kategorii~przez~co~suma wartości poniższej tabeli \ref{tab:area_clasification} jest wyższa~od~rzeczywistej liczby analizowanych artykułów. Połowa przeglądów zawierała jednocześnie meta-analizę wybranego zbioru innych opublikowanych eksperymentów.

\centertable{\toprule
	\textbf{Klasyfikacja} & \textbf{Przegląd teoretyczny} & \textbf{Meta-analiza} & \textbf{Eksperyment} \\\toprule
	\drow{\bf 2} & 12 & 8 & 7 \\
	& \cite{S01,S53,S58,S35,S39,S56,S07,S22,S62,S57,S38,S06} & \cite{S01,S53,S19,S39,S48,S07,S22,S62} & \cite{S04,S50,S21,S38,S26,S02,S06} \\\midrule
	\drow{\bf 1} & \drow{-} & \drow{-} & 13 \\
	&&& \cite{S13,S16,S25,S29,S30,S32,S33,S36,S42,S46,S51,S54,S59} \\\bottomrule
	{\bf Suma} & 12 & 8 & 20 \\\bottomrule
}{c x{3cm} x{2.2cm} x{4cm}}{Podział~na~typy artykułów}{area_clasification}

\subsection{Zastosowania}
Pośród artykułów o charakterze teoretycznym często wymieniane są zastosowania z kategorii opieki medycznej takie jak monitorowania stanu zdrowia \cite{22_HAR_Survey_Ultrasonic} lub wspomaganie rehabilitacji \cite{35_HAR_Wearable_Review}. Popularnym tematem jest też asysta osób starszych, szczególnie tych mieszkających samotnie \cite{21_HAR_Smartphone}. Proponowane rozwiązania, określane najczęściej jako \textit{``Ambient Assisted Living''} mogą pomóc w przypadku problemu lub wypadku poprzez zawiadomienie odpowiednich osób lub służb.

Analiza zachowania użytkowników jest tematem większości publikacji zawierających eksperymenty \cite{2_Real_Time_HAR, 59_Air_Pressure_HAR, 33_Inertial_Study}. Jej popularność jest spowodowana między innymi wysoką dostępnością telefonów zawierających nadające się do tego celu sensory oraz dużą ilością opublikowanych w tym obszarze prac i materiałów obniżających barierę wejściową.

Spośród pozostałych zastosowań wyróżniają się te współtworzące inteligentny dom, jak na przykład sterowanie oświetleniem zależnie od wykrytej aktywności \cite{36_Smart_Home_HAR} a także te związane z wykrywaniem aktywności sportowych \cite{29_Daily_Sport_HAR} które można by zintegrować z istniejącymi platformami takimi jak \textit{Strava}.

\centertable{
	\hline \textbf{Zastosowanie} & \textbf{Przegląd/Analiza} & \textbf{Eksperyment} & \textbf{Suma} \\
	\hline Analiza zachowania & 8 & 12 & 20 \\
	\hline Opieka zdrowotna & 8 & 4 & 12 \\
	\hline Asysta osób starszych & 7 & 3 & 10 \\
	\hline Sport & 5 & 4 & 9 \\
	\hline Bezpieczeństwo & 7 & 0 & 7 \\
	\hline Monitorowanie pacjentów & 4 & 2 & 6 \\
	\hline Inteligentny dom & 4 & 2 & 6 \\
	\hline Rozrywka & 3 & 0 & 3 \\
	\hline
}{|l | c | c || c|}{Najczęściej pojawiające się zastosowania}{uses_analysis}

\subsection{Sensory}
Akcelerometr, magnetometr oraz żyroskop, występujące często razem jako tak zwany \textit{``Inertial measurement unit''} są najpopularniejszym zestawem wybieranym przez autorów artykułów \cite{30_Context_Awareness, 32_Accel_Phone_HAR}. Nadają się one dobrze do wykrywania aktywności ruchowych takich jak przemieszczanie się. Pod warunkiem rozmieszczenia wielu sensorów na ciele osoby monitorowanej, jak w \cite{29_Daily_Sport_HAR} możliwe jest także rozpoznawanie wykonywanych gestów i ruchów ciała.

Mikrofon jest sensorem o szerokim zastosowaniu. Jest używany w środowiskach domowych do zbierania dźwięków z otoczenia i identyfikowania na ich podstawie czynności takich jak chodzenie, stukanie, otwieranie / zamykanie drzwi lub odkurzanie \cite{46_Indor_Audio_Rec}. Jednym z ciekawszych przykładów jego wykorzystania zaproponowanym w jednej z publikacji jest emitowanie ultradźwiękowych sygnałów i nasłuchiwanie echa mikrofonem \cite{22_HAR_Survey_Ultrasonic}. Ta technika, leżąca u podstawy działania sonaru pozwala na wykrycie położenia i względnego ruchu pobliskich przedmiotów poprzez wykorzystanie efektu Dopplera.

Przykłady wykorzystania pulsometru pojawiają się zarówno w obszarze medyczno-zdrowotnym \cite{62_The_Long_Review} przy monitorowaniu parametrów życiowych jak i rekreacyjno-sportowym \cite{51_Fixbit_tracker} przy zapisie aktywności fizycznej użytkownika. Rytm i szybkość bicia serca dostarczają użyteczne i jednocześnie proste w przetwarzaniu i interpretacji dane o aktualnym stanie i poziomowi wysiłku monitorowanej osoby.
\centertable{
	\hline \textbf{Sensor} & \textbf{Przegląd/Analiza} & \textbf{Eksperyment} & \textbf{Suma} \\
	\hline Akcelerometr & 10 & 16 & 26 \\
	\hline Mikrofon & 8 & 6 & 14 \\
	\hline Magnetometr & 7 & 7 & 14 \\
	\hline Żyroskop & 7 & 7 & 14 \\
	\hline Kamera & 9 & 2 & 11 \\
	\hline GPS & 6 & 3 & 9 \\
	\hline Pulsometr & 6 & 2 & 8 \\
	\hline Termometr & 7 & 0 & 7 \\
	\hline Radio/RFID & 6 & 1 & 7 \\
	\hline Czujnik światła & 6 & 0 & 6 \\
	\hline WiFi & 4 & 1 & 5 \\
	\hline Bluetooth & 4 & 1 & 5 \\
	\hline Barometr & 4 & 1 & 5 \\
	\hline
}{|l | c | c || c|}{Najczęściej wykorzystywane sensory}{sensor_analysis}

\pagebreak
\subsection{Modalności}
Zdecydowanie najpopularniejszymi wśród zebranych publikacji modalnościami są proste aktywności ruchowe. Wymienia je, w przypadku przeglądu, lub wykrywa w przeprowadzanym eksperymencie ponad 84\% przeglądanych artykułów. Do aktywności ruchowych zaliczają się najczęściej:
\begin{itemize}
    \item Chodzenie
    \item Stanie
    \item Bieganie
    \item Siedzenie
    \item Leżenie
    \item Wchodzenie, Schodzenie po schodach
    \label{base_modalities}
\end{itemize}

Aktywności te są łatwo wykrywalne za pomocą akcelerometru, czasami połączonego z żyroskopem i magnetometrem, co wpływa na popularność wykorzystania tych sensorów \ref{tab:sensor_analysis} oraz częstą kategoryzację w grupie zastosowań ``Analiza zachowania'' \ref{tab:uses_analysis}. 

\centertable{
	\hline \textbf{Modalność} & \textbf{Przegląd/Analiza} & \textbf{Eksperyment} & \textbf{Suma} \\
	\hline Aktywność ruchowa & 11 & 16 & 27 \\
	\hline Ruch ciała & 7 & 11 & 18 \\
	\hline Poza ciała & 9 & 4 & 13 \\
	\hline Wykonywane gesty & 4 & 2 & 6 \\
	\hline Podnoszenie przedmiotów & 2 & 2 & 4 \\
	\hline Bicie serca & 1 & 3 & 4 \\
	\hline Jedzenie & 1 & 3 & 4 \\
	\hline Picie & 2 & 2 & 4 \\
	\hline Ruch rąk / ramion & 2 & 2 & 4 \\
	\hline
}{|l | c | c || c|}{Najczęściej obserwowane modalności}{modality_analysis}

\subsection{Metryki}
Metryki używane w procesie wykrywania aktywności są bardzo liczne. Z powodu już wspomnianej dużej ilości publikacji i eksperymentów używających akcelerometru większość używanych i wymienianych metryk dotyczy właśnie do tego sensora. Oprócz tego paru autorów wymienia wybrane przez siebie metryki odpowiednie dla pulsometru \cite{4_HAR_Features}, mikrofonu \cite{22_HAR_Survey_Ultrasonic, 46_Indor_Audio_Rec} oraz GPS \cite{26_Mobility_Sensing}. Warto zaznaczyć że część z autorów zupełnie pomija temat metryk w swoich publikacjach.

\centertable{
	\hline \textbf{Sensor} & \textbf{Przegląd/Analiza} & \textbf{Eksperyment} & \textbf{Suma} \\
	\hline Akcelerometr & 20 & 20 & 27 \\
	\hline Pulsometr & 0 & 5 & 5 \\
	\hline Mikrofon & 3 & 3 & 6 \\
	\hline GPS & 0 & 4 & 4 \\
	\hline Żyroskop & 1 & 0 & 1 \\
	\hline\hline {\bf Suma} & 24 & 32 & 43 \\
	\hline
}{|l | c | c || c|}{Ilość rodzajów metryk dotyczących danego sensora}{modality_stats}

W przypadku akcelerometru proste metryki, takie jak odchylenie standardowe, średnia, mediana oraz wartości minimalne i maksymalne są używane zdecydowanie najczęściej co świadczy o ich uniwersalności. Rzadziej pojawiają się metryki związane z entropią sygnału, rozstępem międzykwartylowym czy średnią kwadratową.
\centertable{
	\hline \textbf{Metryka} & \textbf{Przegląd/Analiza} & \textbf{Eksperyment} & \textbf{Suma} \\
	\hline Standard Deviation & 5 & 10 & 15 \\
	\hline Min, Max & 4 & 7 & 11 \\
	\hline Mean, Median & 3 & 8 & 11 \\
	\hline Signal Magnitude Area & 2 & 5 & 7 \\
	\hline Mean Absolute Deviation & 3 & 3 & 6 \\
	\hline Variance & 2 & 3 & 5 \\
	\hline Corelation Coefficients & 2 & 3 & 5 \\
	\hline Interquartile Range & 2 & 3 & 5 \\
	\hline Signal Entropy & 2 & 3 & 5 \\
	\hline Root Mean Square & 2 & 2 & 4 \\
	\hline
}{|l | c | c || c|}{Najczęściej używane metryki - {\bf Akcelerometr}}{accl_features}

\centertable{
	\hline \textbf{Metryka} & \textbf{Eksperyment} & \textbf{Suma} \\
	\hline Step frequency & 2 & 2 \\
	\hline Power Spectral Density & 2 & 2 \\
	\hline Variance & 1 & 1 \\
	\hline FFT Peaks \& Energy & 1 & 1 \\
	\hline Trunk inclination & 1 & 1 \\
	\hline
}{|l | c || c|}{Najczęściej używane metryki - {\bf Pulsometr}}{heart_rate_features}

