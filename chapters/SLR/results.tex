\section{Analiza ilościowa}
Publikacje oznaczone jako {\bf 2} podczas selekcji abstraktów zostały przeczytane a artykuły które otrzymały ocenę {\bf 2} podczas dodatkowej selekcji - przejrzane. Obie grupy stanowiły bazę do wykonanej analizy ilościowej. W sumie liczba zakwalifikowanych do dalszej analizy artykułów wyniosła {\bf 32}.

\subsection{Szukane informacje}
Podczas zapoznawania się z wybranymi publikacjami aktywnie szukane i spisywane były informacje z pięciu kategorii:
\begin{itemize}
    \item {\bf Kategoria}: Typ artykułu. Występują tu trzy wartości:
    \begin{itemize}
		\item {\it Przegląd teoretyczny}: Opisuje teoretyczne zagadnienia związane z danym tematem. Podaje definicje i przykłady.
		\item {\it Meta-analiza}: Jest zbiorem wielu publikacji na określony temat. Dokonuje porównania ich założeń, podejścia oraz wyników.
		\item {\it Eksperyment}: Opisuje oryginalny proces wykrywania aktywności używający wybranych sensorów. Zwykle przeprowadza jego testy na grupie uczestników.
	\end{itemize}
    \item {\bf Zastosowania}: Ogólne obszary poruszane w artykule do których należą wykrywane modalności.
    \item {\bf Sensory}: Urządzenia rejestrujące sygnały z otoczenia używane przez autorów.
    \item {\bf Modalności}: Podstawowe zdarzenia które są wykrywane w procesie rozpoznawania aktywności.
    \item {\bf Metryki}: Wzory matematyczne używane do transformacji sygnałów przy ich przetwarzaniu.
\end{itemize}

\subsection{Kategoria}
Z pośród wyselekcjonowanych artykułów 12 zostało sklasyfikowanych jako \textit{Przegląd teoretyczny} lub \textit{Meta-analiza} podczas gdy pozostałe 20 zawierało opis przeprowadzonego przez autorów eksperymentu. 

{\bf Uwaga:} Niektóre z artykułów zostały zaliczone do więcej niż jednej kategorii przez co suma wartości poniższej tabeli \ref{tab:area_clasification} jest wyższa od rzeczywistej liczby analizowanych artykułów. Połowa przeglądów zawierała jednocześnie meta-analizę zbioru opublikowanych eksperymentów.
\centertable{
	\hline \textbf{Klasyfikacja} & \textbf{Przegląd teoretyczny} & \textbf{Meta-analiza} & \textbf{Eksperyment} \\
	\hline {\bf 2} & 12 & 8 & 7 \\
	\hline {\bf 1} & - & - & 13 \\
	\hline\hline {\bf Suma} & 12 & 8 & 20 \\
	\hline
}{|c | c | c | c|}{Podział na typy artykułów}{area_clasification}

\subsection{Zastosowania}
Pośród artykułów~o~charakterze teoretycznym często wymieniane~są~zastosowania~z~kategorii opieki medycznej takie jak monitorowanie stanu zdrowia \cite{S22}~lub~wspomaganie rehabilitacji \cite{S35}. Popularnym tematem jest też asysta osób starszych, szczególnie tych mieszkających samotnie \cite{S21}. Proponowane rozwiązania, określane najczęściej jako \textit{``Ambient Assisted Living''} mogą pomóc~w~przypadku problemu~lub~wypadku poprzez zawiadomienie odpowiednich osób,~lub~służb.

Analiza zachowania użytkowników jest tematem większości publikacji zawierających eksperymenty \cite{S02, S59, S33}. Jej popularność jest spowodowana między innymi wysoką dostępnością telefonów zawierających nadające się~do~tego celu sensory oraz dużą ilością opublikowanych~w~tym obszarze prac~i~materiałów obniżających barierę wejściową.

Spośród pozostałych zastosowań wyróżniają się~te~współtworzące inteligentny dom, jak~na~przykład sterowanie oświetleniem zależnie~od~wykrytej aktywności \cite{S36}~a~także~te~związane~z~wykrywaniem aktywności sportowych \cite{S29} które można~by~z~powodzeniem zintegrować~z~istniejącymi platformami takimi jak \nameref{sec:strava} eliminując potrzebę ręcznego wybierania wykonywanej aktywności przez użytkownika.

\centertable{\toprule
	\textbf{Zastosowanie} & \textbf{Przegląd/Analiza} & \textbf{Eksperyment} & \textbf{Suma} \\\toprule
	\drow{Analiza zachowania} & 8 & 12 & \drow{20} \\
	& \cite{S01,S53,S58,S19,S35,S56,S22,S57} & \cite{S21,S38,S02,S06,S13,S16,S29,S30,S32,S33,S42,S59} & \\\midrule
	\drow{Opieka zdrowotna} & 8 & 4 & \drow{12} \\
	& \cite{S01,S19,S35,S39,S07,S22,S62,S57} & \cite{S04,S50,S21,S54} & \\\midrule
	\drow{Asysta osób starszych} & 7 & 3 & \drow{10} \\
	& \cite{S01,S19,S35,S39,S56,S07,S62} & \cite{S04,S21,S25} & \\\midrule
	\drow{Sport} & 5 & 4 & \drow{9} \\
	& \cite{S53,S58,S19,S39,S07} & \cite{S29,S30,S32,S51} & \\\midrule
	\drow{Bezpieczeństwo} & 7 & \drow{-} & \drow{7} \\
	& \cite{S01,S19,S35,S39,S48,S56,S22} && \\\midrule
	\drow{Monitorowanie pacjentów} & 4 & 2 & \drow{6} \\
	& \cite{S01,S35,S39,S62} & \cite{S50,S54} & \\\midrule
	\drow{Inteligentny dom} & 4 & 2 & \drow{6} \\
	& \cite{S01,S19,S56,S07} & \cite{S36,S51} & \\\midrule
	\drow{Rozrywka} & 3 & \drow{-} & \drow{3} \\
	& \cite{S01,S35,S48} && \\\bottomrule
}{l x{3.5cm} x{3.5cm} r}{Najczęściej pojawiające się zastosowania}{uses_analysis}

\subsection{Sensory}
Akcelerometr, magnetometr oraz żyroskop, występujące często razem jako~tak~zwany \textit{``Inertial measurement unit''}~są~najpopularniejszym zestawem wybieranym~przez~autorów artykułów \cite{S30, S32}. Nadają się one dobrze~do~wykrywania aktywności ruchowych takich jak przemieszczanie się. Pod~warunkiem rozmieszczenia wielu sensorów~na~ciele osoby monitorowanej, jak~w~\cite{S29} możliwe jest także rozpoznawanie wykonywanych gestów~i~ruchów ciała.

Mikrofon jest sensorem~o~szerokim zastosowaniu. Jest używany~w~środowiskach domowych~do~zbierania dźwięków~z~otoczenia~i~identyfikowania~na~ich podstawie czynności takich jak chodzenie, stukanie, otwieranie / zamykanie drzwi~lub~odkurzanie \cite{S46}. Jednym~z~ciekawszych przykładów jego wykorzystania zaproponowanym~w~jednej~z~publikacji jest emitowanie ultradźwiękowych sygnałów~i~nasłuchiwanie echa mikrofonem \cite{S22}. Ta~technika, leżąca~u~podstawy działania sonaru pozwala~na~wykrycie położenia~i~względnego ruchu pobliskich przedmiotów poprzez wykorzystanie efektu Dopplera.

Przykłady wykorzystania pulsometru pojawiają się zarówno~w~obszarze medyczno-zdrowotnym \cite{S62} przy monitorowaniu parametrów życiowych jak~i~rekreacyjno-sportowym \cite{S51} przy zapisie aktywności fizycznej użytkownika. Rytm~i~szybkość bicia serca dostarczają użyteczne~i~jednocześnie proste~w~przetwarzaniu~i~interpretacji dane~o~aktualnym stanie~i~poziomowi wysiłku monitorowanej osoby.
\centertable{\toprule
\textbf{Sensor} & \textbf{Przegląd/Analiza} & \textbf{Eksperyment} & \textbf{Suma} \\\toprule
\drow{Akcelerometr} & 10 & 16 & \drow{26} \\
& \cite{S01,S53,S58,S19,S35,S39,S56,S07,S62,S57} & \cite{S04,S50,S21,S38,S26,S02,S06,S13,S16,S29,S30,S32,S33,S42,S54,S59} & \\\midrule
\drow{Mikrofon} & 8 & 6 & \drow{14} \\
& \cite{S01,S58,S19,S39,S56,S07,S22,S62} & \cite{S38,S02,S06,S36,S46,S51} & \\\midrule
\drow{Magnetometr} & 7 & 7 & \drow{14} \\
& \cite{S01,S53,S19,S56,S07,S62,S57} & \cite{S21,S38,S02,S06,S13,S42,S59} & \\\midrule
\drow{Żyroskop} & 7 & 7 & \drow{14} \\
& \cite{S01,S53,S19,S56,S07,S62,S57} & \cite{S50,S21,S13,S29,S33,S42,S59} & \\\midrule
\drow{Kamera} & 9 & 2 & \drow{11} \\
& \cite{S01,S53,S58,S39,S48,S56,S07,S22,S62} & \cite{S38,S25} & \\\midrule
\drow{GPS} & 6 & 3 & \drow{9} \\
& \cite{S01,S53,S19,S39,S07,S62} & \cite{S26,S06,S54} & \\\midrule
\drow{Pulsometr} & 6 & 2 & \drow{8} \\
& \cite{S53,S58,S35,S39,S62,S57} & \cite{S04,S51} & \\\midrule
\drow{Termometr} & 7 & \drow{-} & \drow{7} \\
& \cite{S01,S58,S35,S39,S56,S07,S57} && \\\midrule
\drow{Radio/RFID} & 6 & 1 & \drow{7} \\
& \cite{S01,S53,S58,S19,S22,S62} & \cite{S38} & \\\midrule
\drow{Czujnik światła} & 6 & \drow{-} & \drow{6} \\
& \cite{S53,S19,S39,S56,S07,S22} && \\\midrule
\drow{WiFi} & 4 & 1 & \drow{5} \\
& \cite{S01,S53,S39,S56} & \cite{S26} & \\\midrule
\drow{Bluetooth} & 4 & 1 & \drow{5} \\
& \cite{S53,S35,S39,S56} & \cite{S38} & \\\midrule
\drow{Barometr} & 4 & 1 & \drow{5} \\
& \cite{S01,S19,S56,S62} & \cite{S59} & \\\bottomrule
}{l x{4cm} x{4.4cm} r}{Najczęściej wykorzystywane sensory}{sensor_analysis}

\subsection{Modalności}
Zdecydowanie najpopularniejszymi wśród zebranych publikacji modalnościami są proste aktywności ruchowe. Wymienia je, w przypadku przeglądu, lub wykrywa w przeprowadzanym eksperymencie ponad 84\% przeglądanych artykułów. Do aktywności ruchowych zaliczają się najczęściej:
\begin{itemize}
    \item Chodzenie
    \item Stanie
    \item Bieganie
    \item Siedzenie
    \item Leżenie
    \item Wchodzenie, Schodzenie po schodach
    \label{base_modalities}
\end{itemize}

Aktywności te są łatwo wykrywalne za pomocą akcelerometru, czasami połączonego z żyroskopem i magnetometrem, co wpływa na popularność wykorzystania tych sensorów \ref{tab:sensor_analysis} oraz częstą kategoryzację w grupie zastosowań ``Analiza zachowania'' \ref{tab:uses_analysis}. 

\centertable{\toprule
	\textbf{Modalność} & \textbf{Przegląd/Analiza} & \textbf{Eksperyment} & \textbf{Suma} \\\toprule
	\drow{Aktywność ruchowa} & 11 & 16 & \drow{27} \\
	& \cite{S01,S53,S58,S19,S39,S48,S56,S07,S22,S62,S57} & \cite{S04,S50,S21,S38,S26,S02,S06,S13,S16,S29,S30,S32,S33,S42,S54,S59} & \\\midrule
	\drow{Ruch ciała} & 7 & 11 & \drow{18} \\
	& \cite{S53,S58,S19,S35,S48,S07,S62} & \cite{S04,S50,S38,S02,S06,S29,S30,S32,S33,S42,S51} & \\\midrule
	\drow{Poza ciała} & 9 & 4 & \drow{13} \\
	& \cite{S53,S58,S19,S35,S39,S48,S56,S07,S62} & \cite{S04,S38,S02,S42} & \\\midrule
	\drow{Wykonywane gesty} & 4 & 2 & \drow{6} \\
	& \cite{S01,S58,S35,S22} & \cite{S50,S38} & \\\midrule
	\drow{Podnoszenie przedmiotów} & 2 & 2 & \drow{4} \\
	& \cite{S01,S62} & \cite{S38,S36} & \\\midrule
	\drow{Bicie serca} & 1 & 3 & \drow{4} \\
	& \cite{S22} & \cite{S04,S51,S54} & \\\midrule
	\drow{Jedzenie} & 1 & 3 & \drow{4} \\
	& \cite{S62} & \cite{S38,S13,S36} & \\\midrule
	\drow{Picie} & 2 & 2 & \drow{4} \\
	& \cite{S62,S57} & \cite{S38,S36} & \\\midrule
	\drow{Ruch rąk / ramion} & 2 & 2 & \drow{4} \\
	& \cite{S22,S62} & \cite{S38,S13} & \\\bottomrule
}{l x{3cm} x{4.4cm} r}{Najczęściej obserwowane modalności}{modality_analysis}

\subsection{Metryki}
Metryki używane w procesie wykrywania aktywności są bardzo liczne. Z powodu już wspomnianej dużej ilości publikacji i eksperymentów używających akcelerometru większość używanych i wymienianych metryk dotyczy właśnie do tego sensora. Oprócz tego paru autorów wymienia wybrane przez siebie metryki odpowiednie dla pulsometru \cite{S04}, mikrofonu \cite{S22, S46} oraz GPS \cite{S26}. Warto zaznaczyć że część z autorów zupełnie pomija temat metryk w swoich publikacjach.

\centertable{\toprule
	\textbf{Sensor} & \textbf{Przegląd/Analiza} & \textbf{Eksperyment} & \textbf{Suma} \\\toprule
	Akcelerometr & 20 & 20 & 27 \\\midrule
	Pulsometr & 0 & 5 & 5 \\\midrule
	Mikrofon & 3 & 3 & 6 \\\midrule
	GPS & 0 & 4 & 4 \\\midrule
	Żyroskop & 1 & 0 & 1 \\\bottomrule
	{\bf Suma} & 24 & 32 & 43 \\\bottomrule
}{l c c r}{Ilość rodzajów metryk dotyczących danego sensora}{modality_stats}

W przypadku akcelerometru proste metryki, takie jak odchylenie standardowe, średnia, mediana oraz wartości minimalne i maksymalne są używane zdecydowanie najczęściej co świadczy o ich uniwersalności. Rzadziej pojawiają się metryki związane z entropią sygnału, rozstępem międzykwartylowym czy średnią kwadratową.
\centertable{\toprule
	\textbf{Metryka} & \textbf{Przegląd/Analiza} & \textbf{Eksperyment} & \textbf{Suma} \\\toprule
	\drow{Standard Deviation} & 5 & 10 & \drow{15} \\
	& \cite{S58,S19,S39,S56,S62} & \cite{S50,S21,S02,S13,S16,S29,S30,S33,S42,S54} & \\\midrule
	\drow{Min, Max} & 4 & 7 & \drow{11} \\
	& \cite{S58,S19,S56,S62} & \cite{S50,S02,S16,S29,S30,S33,S54} & \\\midrule
	\drow{Mean, Median} & 3 & 8 & \drow{11} \\
	& \cite{S39,S56,S62} & \cite{S50,S13,S16,S29,S30,S33,S42,S54} & \\\midrule
	\drow{Signal Magnitude Area} & 2 & 5 & \drow{7} \\
	& \cite{S56,S62} & \cite{S04,S50,S30,S32,S33} & \\\midrule
	\drow{Mean Absolute Deviation} & 3 & 3 & \drow{6} \\
	& \cite{S19,S39,S56} & \cite{S04,S29,S33} & \\\midrule
	\drow{Variance} & 2 & 3 & \drow{5} \\
	& \cite{S39,S62} & \cite{S50,S16,S29} & \\\midrule
	\drow{Correlation Coefficients} & 2 & 3 & \drow{5} \\
	& \cite{S19,S62} & \cite{S04,S50,S54} & \\\midrule
	\drow{Interquartile Range} & 2 & 3 & \drow{5} \\
	& \cite{S39,S56} & \cite{S29,S30,S33} & \\\midrule
	\drow{Signal Entropy} & 2 & 3 & \drow{5} \\
	& \cite{S56,S62} & \cite{S50,S06,S33} & \\\midrule
	\drow{Root Mean Square} & 2 & 2 & \drow{4} \\
	& \cite{S58,S39} & \cite{S13,S29} & \\\midrule
}{l x{3cm} x{4.4cm} r}{Najczęściej używane metryki - {\bf Akcelerometr}}{accl_features}

Proces wykrywania aktywności zaproponowany w \cite{S04} wykorzystujący akcelerometr oraz pulsometr przykłada dużą wagę do metryk. Po zebraniu i wstępnym przetworzeniu danych z sensorów następuje faza wyodrębnienia cech, której celem jest wydobycie możliwie wielu informacji z uzyskanych sygnałów. Następnie na ich podstawie następuje selekcja najważniejszych metryk używanych przy uczeniu klasyfikatora aktywności. Pod uwagę wzięte zostały zarówno cechy z dziedziny czasu (np. {\it Mean Absolute Deviation}) jak i częstotliwości (np. {\it Power Spectral Density}) źródłowego sygnału.

\centertable{\toprule
	\textbf{Metryka} & \textbf{Eksperyment} & \textbf{Suma} \\\toprule
	\drow{Step frequency} & 2 & \drow{2} \\
	& \cite{S04,S29} & \\\midrule
	\drow{Power Spectral Density} & 2 & \drow{2} \\
	& \cite{S04,S13} & \\\midrule
	\drow{Variance} & 1 & \drow{1} \\
	& \cite{S04} & \\\midrule
	\drow{FFT Peaks \& Energy} & 1 & \drow{1} \\
	& \cite{S04} & \\\midrule
	\drow{Trunk inclination} & 1 & \drow{1} \\
	& \cite{S04} & \\\midrule
}{l x{3cm} r}{Najczęściej używane metryki - {\bf Pulsometr}}{heart_rate_features}

