\section{Przygotowanie}
W porozumieniu z promotorką pracy wstępne pytanie charakteryzujące temat poszukiwań zostało zdefiniowane jako \textbf{``What measures are used for describing human activity?''}. Na jego podstawie zostały wyszczególnione cztery obszary z których składać się będzie zapytanie skierowane do baz publikacji naukowych:
\begin{itemize}
    \item Użytkownik
    \item Aktywność
    \item Metryki
    \item Urządzenie
\end{itemize}

Po dodaniu synonimów mających duży wpływ na liczbę zwracanych wyników powstało końcowe zapytanie. 
\begin{center}
	\begin{minipage}{0.9\linewidth}
		\begin{verbatim}
		user OR human
		AND
		activity OR action OR behavior
		AND
		measures OR metrics OR indicators OR indexes OR monitoring OR recognition
		AND
		application OR system OR wearable OR smartphone OR sensor NOT video
		\end{verbatim}
	\end{minipage}
\end{center}

Tytuły publikacji zostały przeszukiwane pod względem słów z zapytania. Pod uwagę brane były tylko artykuły w języku angielskim, wybranym z powodu jego uniwersalności, opublikowane w ciągu ostatnich 10 lat. Jak powszechnie wiadomo sprzęt i możliwości technologiczne zwiększają się dosyć gwałtownie co prowadzi do szybkiej dezaktualizacji osiągnięć z przed tego okresu czasu. Słowo \textit{Video} zostało usunięte po pierwszych przeglądach wyników z powodu niedopasowania zwracanych wyników do tematu przeglądu.
