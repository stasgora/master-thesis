\section{Przegląd}
Zapytanie zostało wykonane na trzech wyszukiwarkach, a wybrane wyniki scalone po porzednim usunięciu duplikatów. W wyniku otrzymano 650 artykułów.
\centertable{
	\hline \textbf{Wyszukiwarka} & \textbf{Ilość zwróconych wyników} & \textbf{Ilość duplikatów} \\
	\hline Web of Science & 329 & N/A \\
	\hline IEEE & 378 & 76 \\
	\hline PubMed & 114 & 95 \\
	\hline
}{|c | c | c|}{Ilość zwróconych wyników}{article_count}

\subsection{Filtrowanie}
W celu wyodrębnienia najbardziej pasujących artykułów oraz zmiejszenia ich liczby została zastosowana wielostopniowa selekcja. Pierwszym etapem była ocena istotności wyników po ich tytułach w skali od 0 do 2.
\centertable{
	\hline \textbf{Źródło}& \textbf{Ilość} & \textbf{0} & \textbf{1} & \textbf{2} \\
	\hline Web of Science & 329 & 98 & 180 & 51 \\
	\hline IEEE & 302 & 79 & 193 & 30 \\
	\hline PubMed & 17 & 10 & 5 & 2 \\
	\hline\hline \textbf{Suma} & 648 & 187 & 378 & 83 \\
	\hline
}{|c | c | c | c | c|}{Selekcja po tytułach}{title_tagging}

Artykuły zklasyfikowane jako {\bf0} i {\bf1} zostały odrzucone pozostawiając tylko {\bf2}. Następnie analogiczna selekcja została wykonana na podstawie abstraktów.
\centertable{
	\hline \textbf{Ilość} & \textbf{0} & \textbf{1} & \textbf{2} \\
	\hline 82 & 20 & 45 & 17 \\
	\hline
}{|c | c | c | c|}{Selekcja po abstraktach}{abstract_tagging}

Analiza jakościowa została przeprowadzona na artykułach sklasyfikowanych jako \textbf{2}. W celu przeprowadzenia analizy ilościowej na większej liczbie próbek przeprowadzono dodatkową selekcję na wynikach oznaczonych jako \textbf{1} w poprzedniej klasyfikacji.

\centertable{
	\hline \textbf{Ilość} & \textbf{0} & \textbf{1} & \textbf{2} \\
	\hline 45 & 21 & 9 & 15 \\
	\hline
}{|c | c | c | c|}{Dodatkowa selekcja jedynek}{ones_tagging}

\subsection{Analiza ilościowa}
Z pośród wyselekcjonowanych artykułów 12 zostało sklasyfikowanych jako \textit{Przegląd teoretyczny} lub \textit{Meta-analiza} podczas gdy pozostałe 20 zawierało opis przeprowadzonego przez autorów eksperymentu. Niektóre z artykułów zostały zaliczone do więcej niż jednej kategorii. Połowa przeglądów zawierała jednocześnie meta-analizę zbioru opublikowanych eksperymentów.
\centertable{
	\hline \textbf{Klasyfikacja} & \textbf{Przegląd teoretyczny} & \textbf{Meta-analiza} & \textbf{Eksperyment} \\
	\hline {\bf 2} & 12 & 8 & 7 \\
	\hline {\bf 1} & - & - & 13 \\
	\hline\hline {\bf Suma} & 12 & 8 & 20 \\
	\hline
}{|c | c | c | c|}{Podział na typy artykułów}{area_clasification}

\centertable{
	\hline \textbf{Klasyfikacja} & \textbf{Przegląd/Analiza} & \textbf{Eksperyment} & \textbf{Suma} \\
	\hline Analiza zachowania & 8 & 12 & 20 \\
	\hline Opieka zdrowotna & 8 & 4 & 12 \\
	\hline Asysta osób starszych & 7 & 3 & 10 \\
	\hline Sport & 5 & 4 & 9 \\
	\hline Bezpieczeństwo & 7 & 0 & 7 \\
	\hline Monitorowanie pacjentów & 4 & 2 & 6 \\
	\hline Inteligentny dom & 4 & 2 & 6 \\
	\hline Rozrywka & 3 & 0 & 3 \\
	\hline
}{|l | c | c || c|}{Najczęściej pojawiające się zastosowania}{uses_analysis}

\centertable{
	\hline \textbf{Klasyfikacja} & \textbf{Przegląd/Analiza} & \textbf{Eksperyment} & \textbf{Suma} \\
	\hline Akcelerometr & 10 & 15 & 25 \\
	\hline Mikrofon & 8 & 7 & 15 \\
	\hline Magnetometr & 7 & 7 & 14 \\
	\hline Żyroskop & 7 & 7 & 14 \\
	\hline Kamera & 9 & 2 & 11 \\
	\hline GPS & 6 & 3 & 9 \\
	\hline Pulsometr & 6 & 2 & 8 \\
	\hline Termometr & 7 & 0 & 7 \\
	\hline Radio/RFID & 6 & 1 & 7 \\
	\hline Czujnik światła & 6 & 0 & 6 \\
	\hline WiFi & 4 & 1 & 5 \\
	\hline Bluetooth & 4 & 1 & 5 \\
	\hline Barometr & 4 & 1 & 5 \\
	\hline
}{|l | c | c || c|}{Najczęściej wykorzystywane sensory}{sensor_analysis}
