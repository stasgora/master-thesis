\section{Selekcja artykułów}

\subsection{Zbieranie publikacji}
Zapytanie zostało wykonane~na~trzech wyszukiwarkach: \textit{Web of Science}, \textit{IEEE Explore} oraz \textit{PubMed}. Z~powodu różnic~w~składni musiało ono zostać dostosowane~do~każdej~z~wyszukiwarek~z~osobna. Zależnie~od~możliwości wyszukiwarki końcówki słów~z~zapytania zostały zamienione~na~symbole wieloznaczne (*)~w~celu dopasowania ich~do~możliwie dużej liczby form danego słowa.

Dla przykładu \textit{Web of Science} stosuje dwuliterowe klucze dla każdego~z~pól, np. atrybutowi {\bf Title} odpowiada {\bf TI}. Zapytanie dostosowane~do~tej wyszukiwarki wygląda następująco:

\bigskip
\begin{center}
	\begin{minipage}{0.85\linewidth}
		\begin{verbatim}
		TI=((user* OR human*) 
		AND 
		(activit* OR action* OR behavior*)
		AND 
		(measure* OR metric* OR indicator* OR index* OR monitor* OR recogni*) 
		AND 
		(applica* OR system* OR wear* OR *phone OR sensor*) NOT video)
		\end{verbatim}
	\end{minipage}
\end{center}
\bigskip

\noindent Zastosowane zaawansowane ustawienia wyszukiwania:
\begin{itemize}
    \item Język: {\it Angielski};
    \item Typ dokumentów: {\it Artykuł};
    \item Okres czasu: {\it 2010} - {\it 2020};
    \item Kolekcja podstawowa WoS: {\it Science Citation Index Expanded (SCI-EXPANDED) --1900-present}.
\end{itemize}

\subsection{Łączenie wyników}
Każda~z~używanych wyszukiwarek oferuje zapis wyników wyszukiwania~w~formie pliku {\it .csv}, jednak zawarte~w~nich kolumny~są~różnie ułożone~i~podpisane. Z~tego powodu został stworzony skrypt łączący wyniki~z~trzech baz~w~całość, wybierający~tylko~potrzebne kolumny (Tytuł, Abstrakt oraz Listę autorów). Dodatkowo sprawdza~on~i~pomija duplikaty wyników występujące~w~przypadku gdy~ta~sama publikacja została znaleziona~przez~więcej niż jedną wyszukiwarkę. Wykaz liczby wyników wyszukiwania~i~duplikatów został zamieszczony~w~tabeli \ref{tab:article_count}. W~wyniku otrzymano 648 artykułów.
\centertable{\toprule
	\textbf{Wyszukiwarka} & \textbf{Liczba zwróconych wyników} & \textbf{Liczba duplikatów} \\\toprule
	Web of Science & 329 & nd. \\\midrule
	IEEE Explore & 378 & 77 \\\midrule
	PubMed & 114 & 96 \\\bottomrule
}{l c c}{Liczba zwróconych wyników}{article_count}

\subsection{Selekcja artykułów}
W celu wyodrębnienia najbardziej pasujących artykułów oraz zmniejszenia ich liczby została zastosowana wielostopniowa selekcja. W~każdym~z~etapów przyjęte zostały następujące kryteria oraz skala oceny pozycji:
\begin{itemize}
    \item {\bf 0} - Publikacja zupełnie~nie~pasuje~do~zadawanego pytania,
    \item {\bf 1} - Artykuł dotyczy rozważanego zagadnienia, jednak~nie~pasuje~do~kryteriów kategorii {\it 2},
    \item {\bf 2} - Pozycja porusza oryginalny~lub~interesujący temat mogący wnieść dużą wartość~do~przeglądu.
\end{itemize}

\noindent Pierwszym etapem była ocena istotności artykułów~na~podstawie ich tytułów, której wyniki widać~w~tabeli \ref{tab:title_tagging}.
\centertable{\toprule
	\textbf{Źródło}& \textbf{Liczba} & \textbf{0} & \textbf{1} & \textbf{2} \\\toprule
	Web of Science & 329 & 98 & 180 & 51 \\\midrule
	IEEE & 302 & 79 & 193 & 30 \\\midrule
	PubMed & 17 & 10 & 5 & 2 \\\midrule
	\textbf{Suma} & 648 & 187 & 378 & 83 \\\bottomrule
}{l c c c r}{Selekcja~po~tytułach}{title_tagging}

Artykuły zaklasyfikowane jako {\bf0}~i~{\bf1} zostały odrzucone pozostawiając~tylko~{\bf2}. Następnie analogiczna selekcja (tabela \ref{tab:abstract_tagging}) została wykonana~na~podstawie abstraktów.
\centertable{\toprule
	\textbf{Liczba} & \textbf{-\footnotemark} & \textbf{0} & \textbf{1} & \textbf{2} \\\midrule
	83 & 1 & 20 & 45 & 17 \\\bottomrule
}{l c c c r}{Selekcja~po~abstraktach}{abstract_tagging}
\footnotetext{Brak abstraktu}

Analiza jakościowa została przeprowadzona~na~artykułach sklasyfikowanych jako \textbf{2}. W~celu przeprowadzenia analizy ilościowej~na~większej liczbie próbek przeprowadzono dodatkową selekcję (tabela \ref{tab:ones_tagging})~na~wynikach oznaczonych jako \textbf{1}~w~poprzedniej klasyfikacji.

\centertable{\toprule
	\textbf{Liczba} & \textbf{0} & \textbf{1} & \textbf{2} \\\midrule
	45 & 21 & 9 & 15 \\\bottomrule
}{l c c r}{Dodatkowa selekcja jedynek}{ones_tagging}

\subsection{Diagram przepływu PRISMA}
\begin{figure}[H]
	\centering
	\includestandalone[width=\textwidth]{\chapterPath/prisma_diagram}
	\label{fig:prisma_diagram}
\end{figure}
