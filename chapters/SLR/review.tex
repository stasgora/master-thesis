\section{Przegląd}
Zapytanie zostało wykonane na trzech wyszukiwarkach, a wybrane wyniki scalone po poprzednim usunięciu duplikatów. W wyniku otrzymano 650 artykułów.
\centertable{
	\hline \textbf{Wyszukiwarka} & \textbf{Ilość zwróconych wyników} & \textbf{Ilość duplikatów} \\
	\hline Web of Science & 329 & N/A \\
	\hline IEEE Explore & 378 & 76 \\
	\hline PubMed & 114 & 95 \\
	\hline
}{|c | c | c|}{Ilość zwróconych wyników}{article_count}

\subsection{Filtrowanie}
W celu wyodrębnienia najbardziej pasujących artykułów oraz zmniejszenia ich liczby została zastosowana wielostopniowa selekcja. Pierwszym etapem była ocena istotności wyników po ich tytułach w skali od 0 do 2.
\centertable{
	\hline \textbf{Źródło}& \textbf{Ilość} & \textbf{0} & \textbf{1} & \textbf{2} \\
	\hline Web of Science & 329 & 98 & 180 & 51 \\
	\hline IEEE & 302 & 79 & 193 & 30 \\
	\hline PubMed & 17 & 10 & 5 & 2 \\
	\hline\hline \textbf{Suma} & 648 & 187 & 378 & 83 \\
	\hline
}{|c | c | c | c | c|}{Selekcja po tytułach}{title_tagging}

Artykuły zaklasyfikowane jako {\bf0} i {\bf1} zostały odrzucone pozostawiając tylko {\bf2}. Następnie analogiczna selekcja została wykonana na podstawie abstraktów.
\centertable{
	\hline \textbf{Ilość} & \textbf{0} & \textbf{1} & \textbf{2} \\
	\hline 82 & 20 & 45 & 17 \\
	\hline
}{|c | c | c | c|}{Selekcja po abstraktach}{abstract_tagging}

Analiza jakościowa została przeprowadzona na artykułach sklasyfikowanych jako \textbf{2}. W celu przeprowadzenia analizy ilościowej na większej liczbie próbek przeprowadzono dodatkową selekcję na wynikach oznaczonych jako \textbf{1} w poprzedniej klasyfikacji.

\centertable{
	\hline \textbf{Ilość} & \textbf{0} & \textbf{1} & \textbf{2} \\
	\hline 45 & 21 & 9 & 15 \\
	\hline
}{|c | c | c | c|}{Dodatkowa selekcja jedynek}{ones_tagging}
