\subsection{Zastosowania}
Pośród artykułów o charakterze teoretycznym często wymieniane są zastosowania z kategorii opieki medycznej takie jak monitorowania stanu zdrowia \cite{S22} lub wspomaganie rehabilitacji \cite{S35}. Popularnym tematem jest też asysta osób starszych, szczególnie tych mieszkających samotnie \cite{S21}. Proponowane rozwiązania, określane najczęściej jako \textit{``Ambient Assisted Living''} mogą pomóc w przypadku problemu lub wypadku poprzez zawiadomienie odpowiednich osób lub służb.

Analiza zachowania użytkowników jest tematem większości publikacji zawierających eksperymenty \cite{S02, S59, S33}. Jej popularność jest spowodowana między innymi wysoką dostępnością telefonów zawierających nadające się do tego celu sensory oraz dużą ilością opublikowanych w tym obszarze prac i materiałów obniżających barierę wejściową.

Spośród pozostałych zastosowań wyróżniają się te współtworzące inteligentny dom, jak na przykład sterowanie oświetleniem zależnie od wykrytej aktywności \cite{S36} a także te związane z wykrywaniem aktywności sportowych \cite{S29} które można by zintegrować z istniejącymi platformami takimi jak \textit{Strava}.

\centertable{\hline
	\textbf{Zastosowanie} & \textbf{Przegląd/Analiza} & \textbf{Eksperyment} & \textbf{Suma} \\\hline
	\drow{Analiza zachowania} & 8 & 12 & \drow{20} \\\cline{2-3}
	& \cite{S01,S53,S58,S19,S35,S56,S22,S57} & \cite{S21,S38,S02,S06,S13,S16,S29,S30,S32,S33,S42,S59} & \\\hline
	\drow{Opieka zdrowotna} & 8 & 4 & \drow{12} \\\cline{2-3}
	& \cite{S01,S19,S35,S39,S07,S22,S62,S57} & \cite{S04,S50,S21,S54} & \\\hline
	\drow{Asysta osób starszych} & 7 & 3 & \drow{10} \\\cline{2-3}
	& \cite{S01,S19,S35,S39,S56,S07,S62} & \cite{S04,S21,S25} & \\\hline
	\drow{Sport} & 5 & 4 & \drow{9} \\\cline{2-3}
	& \cite{S53,S58,S19,S39,S07} & \cite{S29,S30,S32,S51} & \\\hline
	\drow{Bezpieczeństwo} & 7 & \drow{-} & \drow{7} \\\cline{2-2}
	& \cite{S01,S19,S35,S39,S48,S56,S22} && \\\hline
	\drow{Monitorowanie pacjentów} & 4 & 2 & \drow{6} \\\cline{2-3}
	& \cite{S01,S35,S39,S62} & \cite{S50,S54} & \\\hline
	\drow{Inteligentny dom} & 4 & 2 & \drow{6} \\\cline{2-3}
	& \cite{S01,S19,S56,S07} & \cite{S36,S51} & \\\hline
	\drow{Rozrywka} & 3 & \drow{-} & \drow{3} \\\cline{2-2}
	& \cite{S01,S35,S48} && \\\hline
}{|l | x{3.5cm} | x{3.5cm} || c|}{Najczęściej pojawiające się zastosowania}{uses_analysis}

\subsection{Sensory}
Akcelerometr, magnetometr oraz żyroskop, występujące często razem jako tak zwany \textit{``Inertial measurement unit''} są najpopularniejszym zestawem wybieranym przez autorów artykułów \cite{S30, S32}. Nadają się one dobrze do wykrywania aktywności ruchowych takich jak przemieszczanie się. Pod warunkiem rozmieszczenia wielu sensorów na ciele osoby monitorowanej, jak w \cite{S29} możliwe jest także rozpoznawanie wykonywanych gestów i ruchów ciała.

Mikrofon jest sensorem o szerokim zastosowaniu. Jest używany w środowiskach domowych do zbierania dźwięków z otoczenia i identyfikowania na ich podstawie czynności takich jak chodzenie, stukanie, otwieranie / zamykanie drzwi lub odkurzanie \cite{S46}. Jednym z ciekawszych przykładów jego wykorzystania zaproponowanym w jednej z publikacji jest emitowanie ultradźwiękowych sygnałów i nasłuchiwanie echa mikrofonem \cite{S22}. Ta technika, leżąca u podstawy działania sonaru pozwala na wykrycie położenia i względnego ruchu pobliskich przedmiotów poprzez wykorzystanie efektu Dopplera.

Przykłady wykorzystania pulsometru pojawiają się zarówno w obszarze medyczno-zdrowotnym \cite{S62} przy monitorowaniu parametrów życiowych jak i rekreacyjno-sportowym \cite{S51} przy zapisie aktywności fizycznej użytkownika. Rytm i szybkość bicia serca dostarczają użyteczne i jednocześnie proste w przetwarzaniu i interpretacji dane o aktualnym stanie i poziomowi wysiłku monitorowanej osoby.
\centertable{\hline
\textbf{Sensor} & \textbf{Przegląd/Analiza} & \textbf{Eksperyment} & \textbf{Suma} \\\hline
\drow{Akcelerometr} & 10 & 16 & \drow{26} \\\cline{2-3}
& \cite{S01,S53,S58,S19,S35,S39,S56,S07,S62,S57} & \cite{S04,S50,S21,S38,S26,S02,S06,S13,S16,S29,S30,S32,S33,S42,S54,S59} & \\\hline
\drow{Mikrofon} & 8 & 6 & \drow{14} \\\cline{2-3}
& \cite{S01,S58,S19,S39,S56,S07,S22,S62} & \cite{S38,S02,S06,S36,S46,S51} & \\\hline
\drow{Magnetometr} & 7 & 7 & \drow{14} \\\cline{2-3}
& \cite{S01,S53,S19,S56,S07,S62,S57} & \cite{S21,S38,S02,S06,S13,S42,S59} & \\\hline
\drow{Żyroskop} & 7 & 7 & \drow{14} \\\cline{2-3}
& \cite{S01,S53,S19,S56,S07,S62,S57} & \cite{S50,S21,S13,S29,S33,S42,S59} & \\\hline
\drow{Kamera} & 9 & 2 & \drow{11} \\\cline{2-3}
& \cite{S01,S53,S58,S39,S48,S56,S07,S22,S62} & \cite{S38,S25} & \\\hline
\drow{GPS} & 6 & 3 & \drow{9} \\\cline{2-3}
& \cite{S01,S53,S19,S39,S07,S62} & \cite{S26,S06,S54} & \\\hline
\drow{Pulsometr} & 6 & 2 & \drow{8} \\\cline{2-3}
& \cite{S53,S58,S35,S39,S62,S57} & \cite{S04,S51} & \\\hline
\drow{Termometr} & 7 & \drow{-} & \drow{7} \\\cline{2-2}
& \cite{S01,S58,S35,S39,S56,S07,S57} && \\\hline
\drow{Radio/RFID} & 6 & 1 & \drow{7} \\\cline{2-3}
& \cite{S01,S53,S58,S19,S22,S62} & \cite{S38} & \\\hline
\drow{Czujnik światła} & 6 & \drow{-} & \drow{6} \\\cline{2-2}
& \cite{S53,S19,S39,S56,S07,S22} && \\\hline
\drow{WiFi} & 4 & 1 & \drow{5} \\\cline{2-3}
& \cite{S01,S53,S39,S56} & \cite{S26} & \\\hline
\drow{Bluetooth} & 4 & 1 & \drow{5} \\\cline{2-3}
& \cite{S53,S35,S39,S56} & \cite{S38} & \\\hline
\drow{Barometr} & 4 & 1 & \drow{5} \\\cline{2-3}
& \cite{S01,S19,S56,S62} & \cite{S59} & \\\hline
}{|l | x{4cm} | x{4.4cm} || c|}{Najczęściej wykorzystywane sensory}{sensor_analysis}
