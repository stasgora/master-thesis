\subsection{Zastosowania}
Pośród artykułów o charakterze teoretycznym często wymieniane są zastosowania z kategorii opieki medycznej takie jak monitorowania stanu zdrowia \cite{22_HAR_Survey_Ultrasonic} lub wspomaganie rehabilitacji \cite{35_HAR_Wearable_Review}. Popularnym tematem jest też asysta osób starszych, szczególnie tych mieszkających samotnie \cite{21_HAR_Smartphone}. Proponowane rozwiązania, określane najczęściej jako \textit{``Ambient Assisted Living''} mogą pomóc w przypadku problemu lub wypadku poprzez zawiadomienie odpowiednich osób lub służb.

Analiza zachowania użytkowników jest tematem większości publikacji zawierających eksperymenty \cite{2_Real_Time_HAR, 59_Air_Pressure_HAR, 33_Inertial_Study}. Jej popularność jest spowodowana między innymi wysoką dostępnością telefonów zawierających nadające się do tego celu sensory oraz dużą ilością opublikowanych w tym obszarze prac i materiałów obniżających barierę wejściową.

Spośród pozostałych zastosowań wyróżniają się te współtworzące inteligentny dom, jak na przykład sterowanie oświetleniem zależnie od wykrytej aktywności \cite{36_Smart_Home_HAR} a także te związane z wykrywaniem aktywności sportowych \cite{29_Daily_Sport_HAR} które można by zintegrować z istniejącymi platformami takimi jak \textit{Strava}.

\centertable{
	\hline \textbf{Zastosowanie} & \textbf{Przegląd/Analiza} & \textbf{Eksperyment} & \textbf{Suma} \\
	\hline Analiza zachowania & 8 & 12 & 20 \\
	\hline Opieka zdrowotna & 8 & 4 & 12 \\
	\hline Asysta osób starszych & 7 & 3 & 10 \\
	\hline Sport & 5 & 4 & 9 \\
	\hline Bezpieczeństwo & 7 & 0 & 7 \\
	\hline Monitorowanie pacjentów & 4 & 2 & 6 \\
	\hline Inteligentny dom & 4 & 2 & 6 \\
	\hline Rozrywka & 3 & 0 & 3 \\
	\hline
}{|l | c | c || c|}{Najczęściej pojawiające się zastosowania}{uses_analysis}

\subsection{Sensory}
Akcelerometr, magnetometr oraz żyroskop, występujące często razem jako tak zwany \textit{``Inertial measurement unit''} są najpopularniejszym zestawem wybieranym przez autorów artykułów \cite{30_Context_Awareness, 32_Accel_Phone_HAR}. Nadają się one dobrze do wykrywania aktywności ruchowych takich jak przemieszczanie się. Pod warunkiem rozmieszczenia wielu sensorów na ciele osoby monitorowanej, jak w \cite{29_Daily_Sport_HAR} możliwe jest także rozpoznawanie wykonywanych gestów i ruchów ciała.

Mikrofon jest sensorem o szerokim zastosowaniu. Jest używany w środowiskach domowych do zbierania dźwięków z otoczenia i identyfikowania na ich podstawie czynności takich jak chodzenie, stukanie, otwieranie / zamykanie drzwi lub odkurzanie \cite{46_Indor_Audio_Rec}. Jednym z ciekawszych przykładów jego wykorzystania zaproponowanym w jednej z publikacji jest emitowanie ultradźwiękowych sygnałów i nasłuchiwanie echa mikrofonem \cite{22_HAR_Survey_Ultrasonic}. Ta technika, leżąca u podstawy działania sonaru pozwala na wykrycie położenia i względnego ruchu pobliskich przedmiotów poprzez wykorzystanie efektu Dopplera.

Przykłady wykorzystania pulsometru pojawiają się zarówno w obszarze medyczno-zdrowotnym \cite{62_The_Long_Review} przy monitorowaniu parametrów życiowych jak i rekreacyjno-sportowym \cite{51_Fixbit_tracker} przy zapisie aktywności fizycznej użytkownika. Rytm i szybkość bicia serca dostarczają użyteczne i jednocześnie proste w przetwarzaniu i interpretacji dane o aktualnym stanie i poziomowi wysiłku monitorowanej osoby.
\centertable{
	\hline \textbf{Sensor} & \textbf{Przegląd/Analiza} & \textbf{Eksperyment} & \textbf{Suma} \\
	\hline Akcelerometr & 10 & 16 & 26 \\
	\hline Mikrofon & 8 & 6 & 14 \\
	\hline Magnetometr & 7 & 7 & 14 \\
	\hline Żyroskop & 7 & 7 & 14 \\
	\hline Kamera & 9 & 2 & 11 \\
	\hline GPS & 6 & 3 & 9 \\
	\hline Pulsometr & 6 & 2 & 8 \\
	\hline Termometr & 7 & 0 & 7 \\
	\hline Radio/RFID & 6 & 1 & 7 \\
	\hline Czujnik światła & 6 & 0 & 6 \\
	\hline WiFi & 4 & 1 & 5 \\
	\hline Bluetooth & 4 & 1 & 5 \\
	\hline Barometr & 4 & 1 & 5 \\
	\hline
}{|l | c | c || c|}{Najczęściej wykorzystywane sensory}{sensor_analysis}
