\subsection{Zastosowania}
Wykaz zastosowań przytaczanych~w~artykułach został umieszczony~w~tabeli \ref{tab:uses_analysis}. Pośród artykułów~o~charakterze teoretycznym często wymieniane~są~zastosowania~z~kategorii opieki medycznej takie jak monitorowanie stanu zdrowia \cite{S22}~lub~wspomaganie rehabilitacji \cite{S35}. Popularnym tematem jest też asysta osób starszych, szczególnie tych mieszkających samotnie \cite{S21}. Proponowane rozwiązania, określane najczęściej jako \textit{``Ambient Assisted Living''} mogą pomóc~w~przypadku problemu~lub~wypadku poprzez zawiadomienie odpowiednich osób,~lub~służb.

Analiza zachowania użytkowników jest tematem większości publikacji zawierających eksperymenty \cite{S02, S59, S33}. Jej popularność jest spowodowana między innymi wysoką dostępnością telefonów zawierających nadające się~do~tego celu sensory oraz dużą liczbą opublikowanych~w~tym obszarze prac~i~materiałów obniżających barierę wejściową.

Spośród pozostałych zastosowań wyróżniają się~te~współtworzące inteligentny dom, jak~na~przykład sterowanie oświetleniem zależnie~od~wykrytej aktywności \cite{S36}~a~także~te~związane~z~wykrywaniem aktywności sportowych \cite{S29} które można~by~z~powodzeniem zintegrować~z~istniejącymi platformami takimi jak \nameref{sec:strava} eliminując potrzebę ręcznego wybierania wykonywanej aktywności~przez~użytkownika.

\centertable{\toprule
	\textbf{Zastosowanie} & \textbf{Przegląd/Analiza} & \textbf{Eksperyment} & \textbf{Suma} \\\toprule
	\drow{Analiza zachowania} & 8 & 12 & \drow{20} \\
	& \cite{S01,S53,S58,S19,S35,S56,S22,S57} & \cite{S21,S38,S02,S06,S13,S16,S29,S30,S32,S33,S42,S59} & \\\midrule
	\drow{Opieka zdrowotna} & 8 & 4 & \drow{12} \\
	& \cite{S01,S19,S35,S39,S07,S22,S62,S57} & \cite{S04,S50,S21,S54} & \\\midrule
	\drow{Asysta osób starszych} & 7 & 3 & \drow{10} \\
	& \cite{S01,S19,S35,S39,S56,S07,S62} & \cite{S04,S21,S25} & \\\midrule
	\drow{Sport} & 5 & 4 & \drow{9} \\
	& \cite{S53,S58,S19,S39,S07} & \cite{S29,S30,S32,S51} & \\\midrule
	\drow{Bezpieczeństwo} & 7 & \drow{-} & \drow{7} \\
	& \cite{S01,S19,S35,S39,S48,S56,S22} && \\\midrule
	\drow{Monitorowanie pacjentów} & 4 & 2 & \drow{6} \\
	& \cite{S01,S35,S39,S62} & \cite{S50,S54} & \\\midrule
	\drow{Inteligentny dom} & 4 & 2 & \drow{6} \\
	& \cite{S01,S19,S56,S07} & \cite{S36,S51} & \\\midrule
	\drow{Rozrywka} & 3 & \drow{-} & \drow{3} \\
	& \cite{S01,S35,S48} && \\\bottomrule
}{l x{3.5cm} x{3.5cm} r}{Najczęściej pojawiające się zastosowania}{uses_analysis}

\subsection{Sensory}
Wykaz sensorów używanych~w~artykułach został umieszczony~w~tabeli \ref{tab:sensor_analysis}. Akcelerometr, magnetometr oraz żyroskop, występujące często razem jako~tak~zwany \textit{``Inertial measurement unit''}~są~najpopularniejszym zestawem wybieranym~przez~autorów artykułów \cite{S30, S32}. Nadają się one dobrze~do~wykrywania aktywności ruchowych takich jak przemieszczanie się. Pod~warunkiem rozmieszczenia wielu sensorów~na~ciele osoby monitorowanej, jak~w~\cite{S29} możliwe jest także rozpoznawanie wykonywanych gestów~i~ruchów ciała.

Mikrofon jest sensorem~o~szerokim zastosowaniu. Jest używany~w~środowiskach domowych~do~zbierania dźwięków~z~otoczenia~i~identyfikowania~na~ich podstawie czynności takich jak chodzenie, stukanie, otwieranie / zamykanie drzwi~lub~odkurzanie \cite{S46}. Jednym~z~ciekawszych przykładów jego wykorzystania zaproponowanym~w~jednej~z~publikacji jest emitowanie ultradźwiękowych sygnałów~i~nasłuchiwanie echa mikrofonem \cite{S22}. Ta~technika, leżąca~u~podstawy działania sonaru pozwala~na~wykrycie położenia~i~względnego ruchu pobliskich przedmiotów poprzez wykorzystanie efektu Dopplera.

Przykłady wykorzystania pulsometru pojawiają się zarówno~w~obszarze medyczno-zdrowotnym \cite{S62} przy monitorowaniu parametrów życiowych jak~i~rekreacyjno-sportowym \cite{S51} przy zapisie aktywności fizycznej użytkownika. Rytm~i~szybkość bicia serca dostarczają użyteczne~i~jednocześnie proste~w~przetwarzaniu~i~interpretacji dane~o~aktualnym stanie~i~poziomowi wysiłku monitorowanej osoby.
\centertable{\toprule
\textbf{Sensor} & \textbf{Przegląd/Analiza} & \textbf{Eksperyment} & \textbf{Suma} \\\toprule
\drow{Akcelerometr} & 10 & 16 & \drow{26} \\
& \cite{S01,S53,S58,S19,S35,S39,S56,S07,S62,S57} & \cite{S04,S50,S21,S38,S26,S02,S06,S13,S16,S29,S30,S32,S33,S42,S54,S59} & \\\midrule
\drow{Mikrofon} & 8 & 6 & \drow{14} \\
& \cite{S01,S58,S19,S39,S56,S07,S22,S62} & \cite{S38,S02,S06,S36,S46,S51} & \\\midrule
\drow{Magnetometr} & 7 & 7 & \drow{14} \\
& \cite{S01,S53,S19,S56,S07,S62,S57} & \cite{S21,S38,S02,S06,S13,S42,S59} & \\\midrule
\drow{Żyroskop} & 7 & 7 & \drow{14} \\
& \cite{S01,S53,S19,S56,S07,S62,S57} & \cite{S50,S21,S13,S29,S33,S42,S59} & \\\midrule
\drow{Kamera} & 9 & 2 & \drow{11} \\
& \cite{S01,S53,S58,S39,S48,S56,S07,S22,S62} & \cite{S38,S25} & \\\midrule
\drow{GPS} & 6 & 3 & \drow{9} \\
& \cite{S01,S53,S19,S39,S07,S62} & \cite{S26,S06,S54} & \\\midrule
\drow{Pulsometr} & 6 & 2 & \drow{8} \\
& \cite{S53,S58,S35,S39,S62,S57} & \cite{S04,S51} & \\\midrule
\drow{Termometr} & 7 & \drow{-} & \drow{7} \\
& \cite{S01,S58,S35,S39,S56,S07,S57} && \\\midrule
\drow{Radio/RFID} & 6 & 1 & \drow{7} \\
& \cite{S01,S53,S58,S19,S22,S62} & \cite{S38} & \\\midrule
\drow{Czujnik światła} & 6 & \drow{-} & \drow{6} \\
& \cite{S53,S19,S39,S56,S07,S22} && \\\midrule
\drow{WiFi} & 4 & 1 & \drow{5} \\
& \cite{S01,S53,S39,S56} & \cite{S26} & \\\midrule
\drow{Bluetooth} & 4 & 1 & \drow{5} \\
& \cite{S53,S35,S39,S56} & \cite{S38} & \\\midrule
\drow{Barometr} & 4 & 1 & \drow{5} \\
& \cite{S01,S19,S56,S62} & \cite{S59} & \\\bottomrule
}{l x{4cm} x{4.4cm} r}{Najczęściej wykorzystywane sensory}{sensor_analysis}
