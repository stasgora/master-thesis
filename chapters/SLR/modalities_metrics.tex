\subsection{Modalności}
Zdecydowanie najpopularniejszymi wśród zebranych publikacji modalnościami są proste aktywności ruchowe. Wymienia je, w przypadku przeglądu, lub wykrywa w przeprowadzanym eksperymencie ponad 84\% przeglądanych artykułów. Do aktywności ruchowych zaliczają się najczęściej:
\begin{itemize}
    \item Chodzenie
    \item Stanie
    \item Bieganie
    \item Siedzenie
    \item Leżenie
    \item Wchodzenie, Schodzenie po schodach
    \label{base_modalities}
\end{itemize}

Aktywności te są łatwo wykrywalne za pomocą akcelerometru, czasami połączonego z żyroskopem i magnetometrem, co wpływa na popularność wykorzystania tych sensorów \ref{tab:sensor_analysis} oraz częstą kategoryzację w grupie zastosowań ``Analiza zachowania'' \ref{tab:uses_analysis}. 

\centertable{
	\hline \textbf{Modalność} & \textbf{Przegląd/Analiza} & \textbf{Eksperyment} & \textbf{Suma} \\
	\hline Aktywność ruchowa & 11 & 16 & 27 \\
	\hline Ruch ciała & 7 & 11 & 18 \\
	\hline Poza ciała & 9 & 4 & 13 \\
	\hline Wykonywane gesty & 4 & 2 & 6 \\
	\hline Podnoszenie przedmiotów & 2 & 2 & 4 \\
	\hline Bicie serca & 1 & 3 & 4 \\
	\hline Jedzenie & 1 & 3 & 4 \\
	\hline Picie & 2 & 2 & 4 \\
	\hline Ruch rąk / ramion & 2 & 2 & 4 \\
	\hline
}{|l | c | c || c|}{Najczęściej obserwowane modalności}{modality_analysis}

\subsection{Metryki}
Metryki używane w procesie wykrywania aktywności są bardzo liczne. Z powodu już wspomnianej dużej ilości publikacji i eksperymentów używających akcelerometru większość używanych i wymienianych metryk dotyczy właśnie do tego sensora. Oprócz tego paru autorów wymienia wybrane przez siebie metryki odpowiednie dla pulsometru \cite{4_HAR_Features}, mikrofonu \cite{22_HAR_Survey_Ultrasonic, 46_Indor_Audio_Rec} oraz GPS \cite{26_Mobility_Sensing}. Warto zaznaczyć że część z autorów zupełnie pomija temat metryk w swoich publikacjach.

\centertable{
	\hline \textbf{Sensor} & \textbf{Przegląd/Analiza} & \textbf{Eksperyment} & \textbf{Suma} \\
	\hline Akcelerometr & 20 & 20 & 27 \\
	\hline Pulsometr & 0 & 5 & 5 \\
	\hline Mikrofon & 3 & 3 & 6 \\
	\hline GPS & 0 & 4 & 4 \\
	\hline Żyroskop & 1 & 0 & 1 \\
	\hline\hline {\bf Suma} & 24 & 32 & 43 \\
	\hline
}{|l | c | c || c|}{Ilość rodzajów metryk dotyczących danego sensora}{modality_stats}

W przypadku akcelerometru proste metryki, takie jak odchylenie standardowe, średnia, mediana oraz wartości minimalne i maksymalne są używane zdecydowanie najczęściej co świadczy o ich uniwersalności. Rzadziej pojawiają się metryki związane z entropią sygnału, rozstępem międzykwartylowym czy średnią kwadratową.
\centertable{
	\hline \textbf{Metryka} & \textbf{Przegląd/Analiza} & \textbf{Eksperyment} & \textbf{Suma} \\
	\hline Standard Deviation & 5 & 10 & 15 \\
	\hline Min, Max & 4 & 7 & 11 \\
	\hline Mean, Median & 3 & 8 & 11 \\
	\hline Signal Magnitude Area & 2 & 5 & 7 \\
	\hline Mean Absolute Deviation & 3 & 3 & 6 \\
	\hline Variance & 2 & 3 & 5 \\
	\hline Corelation Coefficients & 2 & 3 & 5 \\
	\hline Interquartile Range & 2 & 3 & 5 \\
	\hline Signal Entropy & 2 & 3 & 5 \\
	\hline Root Mean Square & 2 & 2 & 4 \\
	\hline
}{|l | c | c || c|}{Najczęściej używane metryki - {\bf Akcelerometr}}{accl_features}

\centertable{
	\hline \textbf{Metryka} & \textbf{Eksperyment} & \textbf{Suma} \\
	\hline Step frequency & 2 & 2 \\
	\hline Power Spectral Density & 2 & 2 \\
	\hline Variance & 1 & 1 \\
	\hline FFT Peaks \& Energy & 1 & 1 \\
	\hline Trunk inclination & 1 & 1 \\
	\hline
}{|l | c || c|}{Najczęściej używane metryki - {\bf Pulsometr}}{heart_rate_features}
