\begin{chapter}{Przegląd istniejących rozwiązań monitorujących}
	\label{cha:existing_solutions}
	\newcommand{\chapterPath}{chapters/Existing_solutions}

	Wykrywanie aktywności~i~zachowań ludzkich~z~pomocą technologii jest zagadnieniem szerokim~i~wielowarstwowym. W celu pogłębienia swojej wiedzy, zapoznania się~z~osiągnięciami naukowymi~w~tej dziedzinie oraz zebrania materiałów~do~pracy został przeprowadzony \textit{Systematyczny Przegląd Literatury}. Bazował~on~na~publikacjach naukowych zebranych~z~pomocą trzech wyszukiwarek: \textit{Web of Science}, \textit{IEEE Explore} oraz \textit{PubMed}. 

Trzystopniowa selekcja pozwoliła~na~wyizolowanie najciekawszych~i~najlepiej pasujących~do~postawionego pytania artykułów. Wybrane publikacje zostały, zależnie~od~ich oceny, przeczytane~lub~przejrzane. Jednocześnie tworzone były interesujące~z~punktu widzenia tego przeglądu statystyki. 

Dużym zaskoczeniem~po~zapoznaniu się zebranymi materiałami był fakt~że~pomimo dużej powierzchownej  różnorodności większość~z~opisywanych eksperymentów skupiało się~na~wykrywaniu tego samego,~lub~bardzo podobnego zbioru modalności zawierającego zwykle chodzenie, siedzenie, stanie oraz leżenie. Ze względu~na~ich charakterystykę najodpowiedniejszy~do~ich wykrywania jest akcelerometr, który~z~tego powodu okazał się najczęściej wykorzystywanym sensorem.

Mimo~to~dzięki przeprowadzeniu przeglądu udało się zidentyfikować parę oryginalnych publikacji które przyczyniły się~do~wzbogacenia tej pracy poprzez swoje nowatorskie podejście~do~tematu monitorowania aktywności~i~zachowań ludzkich.

Główną limitacją zakresu przeprowadzonego przeglądu była liczba zaangażowanych uczestników. Oprócz trzeciego etapu selekcji artykułów który został wykonany~przez~dr hab. inż. Agnieszkę Landowską całość została wykonana~przez~autora pracy. Ten fakt znacząco ograniczył ilość publikacji które mogły zostać przestudiowane~po~etapie selekcji. Z 648 artykułów które zostały wstępnie zidentyfikowane jako pasujące~do~zadanego pytania~tylko~ 32 zostało przeczytanych~lub~przeglądniętych~a~co~za~tym idzie włączonych~do~analizy przedstawionej~w~tym rozdziale.


	\section{Platformy monitorujące dla aplikacji} 
	\subsection{Google Analytics}
\label{sec:ga}
Google Analytics jest wiodącą usługą z zakresu monitorowania ruchu w aplikacjach i na stronach internetowych. To kompleksowe rozwiązanie oferujące wiele narzędzi wspomagających analizę i wykorzystanie zbieranych danych. Dostarcza wiedzy o wielu aspektach wykorzystania monitorowanego produktu, zaczynając od sposobów w jaki użytkownicy go odkrywają, poprzez charakterystykę grup odbiorców i szczegółów użycia przez nich systemu do analizy zaangażowania i powodów dla których podejmują takie a nie inne decyzje.

Pierwszym środowiskiem które wspierał Google Analytics były strony internetowe jednak od tej pory wsparcie zostało rozszerzone na platformy mobilne z wydaniem ``Google Analytics for Mobile Apps''. Oprócz tego Google Analytics jest częścią platformy \nameref{sec:firebase} omówionej w kolejnym podrozdziale.

\img{\chapterPath/analytics-dashboard.png}{Panel główny Google Analytics}{ga-dashboard}{.7}

\paragraph{Wydarzenia}
\label{par:ga-events}
Definiowane przez twórcę aplikacji wydarzenia powiązane z konkretnymi akcjami wykonywanymi przez użytkownika. Istnieje możliwość ich kategoryzowania oraz dołączenia do nich prostych atrybutów które dostarczą więcej informacji o kontekście w którym zdarzenie zostało powstało. Przykładami akcji które mogą być warte zaraportowania jako wydarzenia są: założenie konta, logowanie, pobranie oferowanej treści, dokonanie zakupu oraz wypełnienie formularza \cite{GA_Events}.

\paragraph{Wyświetlenia ekranów}
Szczególnym typem wydarzenia jest przejście na nowy ekran aplikacji. Poprzez dodanie do niego parametrów takich jak sygnatura czasowa oraz identyfikator użytkownika, Google Analytics jest w stanie stworzyć statystyki czasu przebywania na poszczególnych ekranach aplikacji. Umożliwiają one łatwe zidentyfikowanie popularnych i nieużywanych części aplikacji \cite{GA_Pages}.

\paragraph{Sesje}
Są zdefiniowane jako grupy interakcji złożone z wyświetleń ekranu, wydarzeń i transakcji wykonanych przez jednego użytkownika w ograniczonym czasie. Sesje są automatycznie kończone po 30 minutach braku aktywności oraz o północy. Ilość sesji w kolejnych okresach czasu jest intuicyjną miarą ruchu w aplikacji. Możliwe jest także odniesienie innych metryk do sesji, jak na przykład sprawdzenie średniej ilości stron wyświetlonych w pojedynczej sesji \cite{GA_Sessions}.

\paragraph{Odbiorcy}
Ta sekcja zawiera wszystkie informacje zebrane na temat użytkowników aplikacji. Umożliwia kategoryzowanie odbiorców w grupy docelowe na podstawie zebranych o nich danych i metryk, takich jak wygenerowane wydarzenia, wiek, płeć, lokalizacja, wykorzystywane urządzenie, używana wersja aplikacji, dane demograficzne oraz zainteresowania. Powyższe metryki są zbierane automatycznie przez Google Analytics, w zależności od potrzeb istnieje też możliwość dodawania własnych, lepiej dostosowanych do profilu działalności firmy i zawartości monitorowanego produktu \cite{GA_Audiences}.

\paragraph{Lejki}
\label{par:ga-funnels}
Służą do analizy i wizualizacji serii kroków wykonywanych przez użytkowników składających się na ważną z biznesowego punktu widzenia akcję, taką jak dokonanie zakupu lub założenie konta. Umożliwiają łatwe zidentyfikowanie problematycznych miejsc w których użytkownicy mają problemy, wahają się lub co gorsza  zupełnie rezygnują z wykonywanej czynności \cite{GA_Funnels}.

\paragraph{Retencja i zaangażowanie} 
Retencja to ilość powracających do aplikacji  użytkowników w stosunku do nowych klientów odwiedzających ją po raz pierwszy. Z kolei zaangażowanie jest mierzone jako zmiana długości czasu który użytkownicy spędzają w aplikacji oraz ilości akcji które wykonują. Obie informacje dają bardzo ważny sygnał na temat jakości, atrakcyjności i użyteczności aplikacji z punktu widzenia klientów \cite{GA_Retention}.

\paragraph{Zdobywanie użytkowników}
Informacja w jaki sposób użytkownicy dowiadują się o aplikacji jest kluczowa przy planowaniu jej reklamy i rozwoju. Google Analytics pozwala na śledzenie źródła z którego użytkownik pobrał aplikację. Jeśli dostępna jest ona na wielu platformach i sklepach pokazywane są też statystyki popularności w każdym z nich \cite{GA_Aquisition}.

	\subsection{Firebase}
\label{sec:firebase}
Firebase jest platformą firmy Google oferującą szereg usług skierowanych~do~twórców aplikacji mobilnych~i~webowych. Początkowo dostępne było uwierzytelnianie użytkowników oraz hosting, jednak~od~tego czasu platforma została rozszerzona~o~szereg innych usług,~z~których część jest związana~z~monitorowaniem aplikacji~i~urządzeń. \nameref{sec:ga}, które zostało opisane~w~osobnym podrozdziale, zostało dołączone jako jedna~z~usług oferowana~w~ramach Firebase (jak widać~w~menu~na~rysunku \ref{fig:firebase-crashlytics}).

\bigskip
\img{\chapterPath/firebase-crashlytics.png}{Panel Crashlytics~w~konsoli Firebase}{firebase-crashlytics}{.9}

\paragraph{Crashlytics}
Usługa wykrywająca występujące~w~aplikacji błędy natury programistycznej. Zebrane informacje związane~z~błędami~są~wysyłane~i~prezentowane~w~użytecznej dla twórcy aplikacji formie. Narzędzie oferuje statystyki liczby dotkniętych błędem użytkowników oraz możliwość powiadomienia twórców aplikacji~o~pojawieniu się nowego problemu \cite{Fb_Crashlytics}.

Powtarzające się błędy~są~grupowane,~a~ich wystąpienia zostają oznaczone~na~osi czasu. Do~każdego z nich dołączany jest pomocny w znajdowaniu przyczyny błędu zrzut stosu wywołań funkcji. Istnieje też możliwość przeglądnięcia zebranych~przez~Google Analytics \hyperref[par:ga-events]{wydarzeń} które miały miejsce bezpośrednio przed wystąpieniem błędu, takich jak otwierane strony~i~wykonywane akcje. 

\paragraph{Monitoring wydajności}
Narzędzie automatyczne zbierające metryki dotyczące wydajności działania aplikacji takie jak:
\begin{itemize}
	\item Czas uruchamiania aplikacji mierzony~od~momentu jej otwarcia~do~pełnego załadowania,
	\item Długości działania~na~ekranie użytkownika oraz~w~tle,
	\item Dla każdego ekranu procent wolno ładujących~i~zawieszających się klatek,
	\item Czas trwania zapytań sieciowych,
	\item Rozmiar wysyłanych~i~odbieranych~w~trakcie zapytań danych,
	\item Procent zapytań zakończonych sukcesem.
\end{itemize}
\bigskip

Istnieje możliwość definiowania i mierzenia dodatkowych wykonywanych~przez~aplikację zadań. Domyślną metryką jest czas, można też jednak dodawać inne. Do~wszystkich pomiarów dodawane~są~atrybuty dotyczące urządzenia~i~systemu,~na~którym były wykonywane, używanej wersji aplikacji oraz przybliżonej lokalizacji geograficznej. Możliwe jest też dodawanie własnych parametrów. Wszystkie zebrane metadane służą~do~kategoryzacji zebranych pomiarów umożliwiając ich  filtrowanie~i~ułatwiając identyfikację źródeł problemów~z~wydajnością \cite{Fb_Pref_Monitor}.

\paragraph{Testy A/B}
Rodzaj testów polegających~na~porównaniu dwóch~lub~więcej wersji elementu interfejsu. Każda~z~nich jest wyświetlana innej grupie odbiorców, których reakcje, zebrane~w~formie metryk, wskazują~na~efektywność każdej~z~wersji. 

Firebase oferuje rozwiązanie znacznie ułatwiające proces przygotowywania, przeprowadzania~i~analizy testów A/B. Dzięki integracji~z~inną usługą oferowaną~w~ramach pakietu Firebase --- zdalną konfiguracją, twórcy aplikacji~są~w~stanie łatwo definiować grupy odbiorców, którym zostanie wyświetlony dany wariant interfejsu będący przedmiotem testu. Dostęp~do~szerokiej gamy narzędzi Google Analytics,~w~szczególności \hyperref[par:ga-funnels]{lejków} oraz  \hyperref[par:ga-events]{wydarzeń}, pozwala~na~zebranie danych,~na~których podstawie możliwe jest wybranie lepszej~z~testowanych wersji. Firebase pozwala także~na~testowanie wariantów powiadomień mobilnych, które aplikacja wysyła~do~użytkowników~w~celu znalezienia ich najefektywniejszej zawartości \cite{Fb_AB_Testing}.

	\subsection{Matomo}
\label{sec:matomo}
Matomo jest kompleksową platformą monitorującą konkurującą~z~Google Analytics. Podobnie jak ona oferuje duży zestaw narzędzi~i~wspiera zarówno środowisko webowe jak~i~mobilne. Matomo oferuje wiele funkcji analogicznych~do~opisanych usług produktu firmy Google. Moduł Odwiedzający~to~odpowiednik \hyperref[par:ga-audiences]{odbiorców}, podczas gdy Kohorty~są~analogiczne~do~wskaźnika \hyperref[par:ga-retention]{retencji}. Pod~tą~samą nazwą dostępne~są~lejki, wydarzenia oraz metryki. Matomo oferuje też narzędzie pozwalające~na~automatyczną migrację danych~z~konta Google Analytics.

\paragraph{Zachowanie}
Zbiorczy moduł zbierający informacje dotyczące działań użytkowników. Zawiera statystyki dotyczące ekranów serwisu takie jak liczba wyświetleń, średni czas przeglądania oraz listę miejsc,~z~których użytkownicy trafili~na~daną stronę. Pozwala~na~śledzenie zdefiniowanych wydarzeń, popularnych ścieżek wybieranych~przez~użytkowników oraz wskaźników retencji.

\paragraph{Kanały marketingowe}
Pozwalają~na~analizę czynników wpływających~na~zwrot~z~inwestycji \ang{return~on~investment - ROI}. Zaliczają się~do~nich wyniki wyszukiwania~w~przeglądarkach internetowych, odnośniki~na~innych stronach oraz kampanie reklamowe. Każde~z~tych źródeł~ma~przypisaną liczbę konwersji oraz uzyskany dochód. Dodatkowo możliwe jest wyświetlenie zmiany wybranych wskaźników~w~czasie.

\paragraph{E-commerce}
Zbiera informacje dotyczące zakupów. Prezentuje historię zysków~z~podziałem~na~kraje oraz sprzedawane produkty. Zapisuje też profile odwiedzających~i~klientów pozwalając~na~lepsze dopasowanie prezentowanej~im~oferty. Moduł współpracuje~z~wieloma popularnymi platformami typu e-commerce, znacznie przyśpieszając jego konfigurację.

\paragraph{Nagrania sesji}
Zapisane~w~formie filmu nagrania wizyt klientów. Wszystkie wykonane~przez~nich akcje, takie jak kliknięcia~i~ruchy kursora, przewijanie~i~zmienianie rozmiaru ekranu oraz interakcje~z~interfejsem~są~nagrywane~i~zaznaczane~na~filmie~w~momencie ich występowania. Ruch myszy pozostawia ślad, rysując linię~na~ekranie.

\paragraph{Treści wideo}
Moduł zbiera szczegółowe informacje~na~temat odtworzeń zamieszczonych treści multimedialnych,~w~szczególności filmów. Zbierane dane~to~między innymi liczba odtworzeń, czas oglądania,~a~także wykres procentowy liczby widzów oglądających dany fragment filmu. Ponadto wyświetlenia~są~nanoszone~na~mapę, pozwalając~na~analizę popularności geograficznej.

	
	\subsection{Porównanie rozwiązań}
	Usługa firmy Google jest niekwestionowanym liderem~w~kategorii rozwiązań~i~platform monitorujących ruch~i~zbierających statystyki użytkowania stron internetowych. Według badania przeprowadzonego~przez~serwis {\it W3Techs} jest ona używana~przez~56.6\% stron internetowych~a~jej udział~w~rynku rozwiązań monitorujących~to~aż~85.9\% \cite{Analytics_Stats}. Taka przewaga wywiera~na~konkurencji dużą presję oferowania analogicznych narzędzi,~do~których klienci mogą być przyzwyczajeni. Przeanalizowane narzędzie \nameref{sec:matomo}~z~pewnością potwierdza~tą~tezę próbując przekonać~do~siebie klientów transparentnością sposobu przechowywania danych~i~dodatkowymi funkcjami których~nie~oferuje produkt firmy Google. Rozwiązanie \nameref{sec:firebase} jest kategoryzowane jako {\it ``back-end jako usługa''} \ang{backend~as~a~service - BaaS},~i~jako takie~nie~powinno być bezpośrednio porównywane~do~pozostałych dwóch przytoczonych usług. Mimo~że~niniejsza praca opisuje~tylko~i~wyłącznie wchodzące~w~skład produktu \nameref{sec:firebase} narzędzia mające związek~z~monitorowaniem,~są~one~w~większym stopniu skierowane~do~programistów niż analityków biznesowych~a~spełniane~przez~nie~zadania~są~z~natury techniczne.
	
	\section{Aplikacje przetwarzające dane z czujników}

\subsection{Strava}
\label{sec:strava}
Popularna usługa pozwalająca na śledzenie aktywności fizycznej. Działa na podstawie modułu GPS, dzięki któremu zapisuje trasę przebytą przez użytkownika. Biorąc pod uwagę typ aktywności wybrany przez użytkownika, jego wagę, szybkość przemieszczania oraz różnicę elewacji aplikacja szacuje ilość spalonych podczas treningu kalorii. Strava posiada wersję na inteligentne zegarki, które dzięki wyposażeniu w dodatkowe czujniki dostarczają aplikacji dodatkowych metryk takich jak tętno. 

Historia przebytych tras jest zapisywana i może być udostępniana innym użytkownikom. Publiczne aktywności są automatycznie grupowane ze względu na lokalizację, typ i czas. W 2017 roku została udostępniona globalna mapa cieplna zawierająca publicznie udostępnione dane użytkowników z dwóch poprzednich lat \cite{Strava_Heatmap}. Według autorów to największy tego typu zbiór danych, zawierający 700 milionów zapisanych aktywności o całkowitej długości 16 miliardów kilometrów.

\img[\footnotemark]{\chapterPath/strava-heatmap.jpeg}{Strava - Mapa cieplna zarejestrowanych aktywności fizycznych w Moskwie}{strava-heatmap}{.9}
\footnotetext{\url{https://medium.com/strava-engineering/the-global-heatmap-now-6x-hotter-23fc01d301de}}

\subsection{Samsung Health}
Aplikacja firmy Samsung towarzysząca produktom takim jak zegarek Galaxy Watch oraz opaska Galaxy Fit. Przetwarza i magazynuje dane zbierane przez zawarte w nich czujniki. Podobnie jak \nameref{sec:strava} pozwala na zapisywanie przebytych tras i aktywności sportowych. Z powodu kontroli nad urządzeniami na których działa aplikacja Samsung był w stanie dodać do aplikacji szereg innowacyjnych funkcji:

\paragraph{Puls} Obecność pulsometra w nowszych modelach linii Galaxy umożliwia mierzenie pulsu, pulsoksymetrii (wysycenia krwi tlenem) oraz poziomu stresu. 

\paragraph{Kroki} Wbudowany w smartfon lub opaskę pedometr pozwala na mierzenie ilości kroków wykonanych każdego dnia porównując ją do ustawionego celu (domyślnie 10~000 kroków). Na podstawie liczby kroków obliczany jest także przebyty dystans oraz ilość spalonych kalorii. Dane historyczne są widoczne na grafie.

\paragraph{Sen} Jeśli użytkownik nie zdejmuje zegarka lub opaski na noc aplikacja może monitorować jego jakość następujące po sobie fazy snu na podstawie wykonywanych ruchów. Zebrane dane pomagają dostosować harmonogram snu w celu lepszego wypoczęcia.

\paragraph{Inne} Oprócz danych automatycznie zbieranych z czujników aplikacja służy jako dziennik poprzez umożliwienie ręcznego wprowadzania danych. Monitorowanie diety wymaga wpisywania spożytych kalorii i składników odżywczych. Możliwe jest też wprowadzanie aktualnej wagi oraz ilości wypitej wody. 

\subsection{Google Fit}

\subsection{Sleep Cycle}

	
	\subsection{Porównanie rozwiązań}
	Aplikacja \nameref{sec:strava} jest popularnym rozwiązaniem monitorującym aktywności fizyczne zawierającym wyróżniające~je~na~tle konkurencji elementy sieci społecznościowej takie jak dzielenie się~ze~znajomymi szczegółami ostatnich wycieczek. Aplikacja \nameref{sec:samsung_health} podobnie jak \nameref{sec:strava} umożliwia śledzenie treningów, oferując jednak szerszy zakres monitorowanych aktywności wykraczających poza domenę sportu~w~życie codzienne. Podczas gdy \nameref{sec:strava} skupia się~na~ćwiczeniach fizycznych~na~świeżym powietrzu umożliwiających ich rejestrację~przez~system GPS, \nameref{sec:samsung_health} pozwala między innymi~na~zapis wagi, pulsu, spożywanych posiłków, zrobionych kroków oraz ilości wypitej wody. Aplikacja \nameref{sec:sleep_cycle} jest jedną~z~ciekawszych~i~bardziej innowacyjnych propozycji zastosowania czujników zawartych~w~telefonach, które zostały przytoczone~w~tej pracy.  Wykorzystanie nietypowych~i~zaawansowanych jak dla kategorii budzików technik skutkuje powstaniem wyróżniającej aplikację funkcji prezentującej potencjał różnorodności zastosowań danych pochodzących~z~powszechnie dostępnych czujników.
	
	\section{Aplikacje monitorujące użycie telefonu}

\subsection{Digital Wellbeing}
\label{sec:digital_wellbeing}
Aplikacja stworzona~przez~Google~na~system Android, której celem jest monitorowanie~i~raportowanie nawyków użycia telefonu~przez~użytkownika,~a~także pomoc~w~ich kontroli~i~zmianie. Narzędzie zbiera informacje~o~codziennym wykorzystaniu urządzenia~z~podziałem~na~zainstalowane aplikacje. Te~statystyki~są~wizualizowane~w~formie wykresu pokazanego~na~rysunku \ref{fig:digital-wellbeing}. Monitorowane~są~też inne statystyki takie jak liczba odblokowań telefonu~i~otrzymanych powiadomień. 
\bigskip
\img[\footnotemark]{\chapterPath/digital-wellbeing.png}{Panel aplikacji Digital Wellbeing}{digital-wellbeing}{.3}
\footnotetext{\url{https://wellbeing.google/}}

Aplikacja posiada też szereg funkcji mających dać użytkownikowi większą kontrolę~nad~swoimi nawykami. Wyczerpanie dziennych limitów czasu użycia nakładanych~na~aplikacje powoduje~ich~dezaktywację~do~końca dnia. Tryb skupienia pozwala~na~czasowe wyciszenie wybranych aplikacji~a~tryb czarno-biały wyszarza cały ekran neutralizując znajdujące się~na~nim rozpraszające elementy.

\subsection{Family Link}
\label{sec:family_link}
Rozwiązanie komplementarne~do~aplikacji \nameref{sec:digital_wellbeing} pozwalające~na~monitorowanie~i~kontrolę urządzeń dzieci. Po~instalacji~i~konfiguracji rodzic jest~w~stanie przeglądać aktywność dzieci~z~podziałem~na~czas spędzony~w~zainstalowanych aplikacjach. Ma~też możliwość weryfikacji~i~zatwierdzenia~lub~odrzucenia prób pobrania nowych aplikacji~przez~dziecko. Wbudowane funkcje kontroli pozwalają~na~ustawianie dziennych limitów korzystania~z~telefonu oraz ręcznego blokowania dostępu~do~niego. Dostępna jest też funkcja dająca wgląd~w~aktualną lokalizację urządzenia dziecka~na~mapie.

\subsection{Screen Time}
\label{sec:screen_time}
Rozwiązanie firmy Apple~na~platformy iOS~i~MacOS oferujące funkcjonalności analogiczne~do~aplikacji \nameref{sec:digital_wellbeing} oferowanej~na~system Android~przez~Google. Jest wbudowane~w~system~i~pozwala~na~podgląd swojej aktywności~na~danym urządzeniu~z~podziałem~na~obszary takie jak gry, produktywność~i~media społecznościowe (rysunek \ref{fig:screen-time}). Czas użycia każdej aplikacji jest też mierzony osobno~i~może być kontrolowany poprzez ustawienie limitów. Istnieje też możliwość ustalania okresów czasu,~w~których niechciane powiadomienia, sygnały~i~połączenia będą wyciszane. Zawartość wyświetlana~na~ekranie jest także monitorowana~i~może być zablokowana~na~życzenie użytkownika jeśli zawiera nieodpowiednie treści.

\bigskip
\img[\footnotemark]{\chapterPath/screen-time.jpeg}{Panel  narzędzia Screen Time}{screen-time}{.3}
\footnotetext{\url{https://macreports.com/what-do-grey-bars-mean-in-screen-time-reports/}}

\subsection{YouTube}
\label{sec:youtube}
YouTube posiada wbudowane mechanizmy monitorujące ilość oglądanych treści. Aplikacja pozwala~na~przeglądanie statystyk czasu spędzanego~w~aplikacji oraz ustawianie komunikatów wyświetlanych~po~określonym czasie ciągłego oglądania (rysunek \ref{fig:youtube-break}). Sugerują one zrobienie sobie przerwy~i~odpoczęcie~od~ekranu.

\bigskip
\img[\footnotemark]{\chapterPath/youtube.jpg}{Przypomnienie~o~przerwie - YouTube}{youtube-break}{.5}
\footnotetext{\url{https://wellbeing.google/get-started/focus-your-time-with-tech/}}

	
	\subsection{Porównanie rozwiązań}
	Rozwiązania \nameref{sec:screen_time}~i~\nameref{sec:digital_wellbeing} oferowane odpowiednio~przez~Apple oraz Google oferują bardzo podobny zestaw funkcji~i~zbieranych danych różniąc się głównie sposobem ich prezentacji. Wskazuje~to~że~obie firmy mają zgodną wizję zestawu narzędzi monitorujących użycie telefonu które najlepiej pomogą ich użytkownikom~w~zachowaniu równowagi pomiędzy codziennymi obowiązkami~a~cyfrowymi pokusami. Dodatkowa możliwość kontroli~nad~urządzeniami swoich dzieci zapewniana~przez~\nameref{sec:family_link} jest szczególnie ważna~w~ostatnich latach kiedy często mają one stały dostęp~do~smartfonów~od~najmłodszych lat. Platformy społecznościowe, włącznie~z~\nameref{sec:youtube},~są~skonstruowane~tak~aby jak najdłużej utrzymywać uwagę użytkowników proponując~im~nowe treści. Nie~jest~to~nic dziwnego biorąc~pod~uwagę~że~bezpośrednio przekłada się~to~na~ich zyski. Z~tego powodu każdy wprowadzony~przez~twórców platformy  mechanizm pomagający~w~kontroli~nad~spędzanym~w~niej czasem działa~na~korzyść użytkowników. Dzięki niemu~nie~muszą oni~tak~często sięgać~po~zewnętrzne rozwiązania kontrolujące użycie aplikacji.
	
	\section{Podejścia~do~prywatności}
Zagadnienie prywatności jest drażliwe~i~może łatwo prowadzić~do~kontrowersji. Ten aspekt powinien być szczególnie dobrze przemyślany~przez~firmy tworzące produkty zbierające dane użytkowników. Mimo tego, nawet~w~ograniczonym zbiorze wymienionych przykładów, podeście~i~dbałość~o~prywatność użytkowników diametralnie się różni. Głównym czynnikiem mającym~na~to~wpływ jest polityka firmy tworzącej dane rozwiązanie.

\paragraph{Google}
Jednym~z~głównych źródeł dochodu dla Google~są~wyświetlane reklamy. Aby skutecznie profilować zainteresowania użytkowników, konieczne jest zbieranie potencjalnie wrażliwych danych. Z~tego powodu wiele~z~produktów firmy, włączając \nameref{sec:ga}, zostało oskarżonych~lub~przyłapanych~na~kolekcjonowaniu danych użytkowników~w~celu ich profilowania. Aplikacja \nameref{sec:youtube} także domyślnie zapisuje wyświetlane filmy~w~celu wyświetlania spersonalizowanych propozycji~i~reklam.

\paragraph{Apple}
Firma Apple~nie~zarabia~na~danych swoich użytkowników. Dzięki temu znajduje się~w~pozycji,~w~której może zyskiwać zaufanie klientów poprzez publicznie komunikowane dbanie~o~prywatność ich danych. Urządzenia Apple~są~znane~z~posiadania jednych~z~najlepszych zabezpieczeń~na~rynku. Niedawno firma wprowadziła obowiązek pytania użytkownika,~czy~zgadza się~na~śledzenie~przez~każdą aplikację~z~osobna \cite{Apple_Ad_Transparency}. Bez~takiej zgody aplikacja~nie~ma~dostępu~do~identyfikatora używanego~do~wyświetlania profilowanych reklam oraz~nie~może zbierać~i~zapisywać danych~o~użytkowniku.

\paragraph{Matomo}
Według twórców jedną~z~głównych zalet \nameref{sec:matomo}~w~porównaniu~do~rozwiązania Google jest dbałość~o~prywatność użytkowników~i~ochrona ich danych. Podczas gdy Google używa zebranych danych~w~celu profilowania użytkowników~i~wyświetlania~im~reklam, klienci platformy Matomo pozostają wyłącznymi właścicielami zbieranych~na~swoich witrynach danych \cite{Matomo_Data}. Oferowana jest też możliwość całkowitego uniezależnienia~i~decentralizacji poprzez uruchomienie własnej kopii serwera zbierającego dane.

\paragraph{Strava}
Po jej publikacji globalna mapa cieplna aktywności zarejestrowanych~przez~aplikację \nameref{sec:strava} stała się powodem poważnej kontrowersji opartej~na~podłożu prywatności \cite{Strava_Military_Bases}. Udostępnione publicznie trasy przebyte~przez~żołnierzy~na~służbie zostały włączone~do~mapy cieplnej, powodując publiczne udostępnienie domniemanych lokalizacji tajnych baz wojskowych (przykład jest widoczny~na~rysunku \ref{fig:strava_military_base}). Student bezpieczeństwa międzynarodowego, który przypadkowo odkrył ten fakt, postanowił~go~upublicznić, aby doprowadzić~do~poprawy działania serwisu, jednak raz opublikowanych danych~nie~da~się~z~powrotem utajnić.

\bigskip
\img[\footnotemark]{\chapterPath/strava_military_base.jpg}{Ruchy żołnierzy~na~terenie amerykańskiej bazy lotniczej Bagram~w~Afganistanie}{strava_military_base}{.8}

\footnotetext{\url{https://www.bbc.com/news/technology-42853072}}

\end{chapter}
