\section{Wstęp}
Na rynku dostępnych jest wiele rozwiązań oferujących usługi w zakresie monitorowania aplikacji, urządzeń i ich użytkowników. Niektóre z nich są wbudowane lub oferowane razem z systemami operacyjnymi, inne wymagają integracji z aplikacją docelową przez jej twórców. Można je podzielić na następujące kategorie:

\paragraph{Dane o użytkowaniu pojedynczej aplikacji} 
Ich celem jest zebranie jak największej ilości przydatnych dla twórców aplikacji informacji o jej wykorzystaniu oraz wykonywanych w niej czynnościach. Zbierane dane często dotyczą też użytkowników, ich nawyków, lokalizacji oraz urządzeń. Najczęściej mają one formę usług dających dostęp do zebranych danych poprzez interfejs webowy. 

\paragraph{Dane z czujników w urządzeniach} 
Do tej kategorii możemy zaliczyć wszystkie aplikacje wykorzystujące, poprzez zbieranie i przetwarzanie, dane z licznych czujników znajdujących się w smartphonach. Zastosowania takich danych mogą być bardzo różne jednak jedno z popularniejszych jest śledzenie ćwiczeń i aktywności sportowych.

\paragraph{Użytkowanie urządzenia i aplikacji} 
Rozwiązania tego typu często zintegrowane są z systemem operacyjnym urządzenia na którym działają. Zbierają one statystyczne dane o użyciu telefonów, takie jak czas bezczynności i długość okresów zapalonego ekranu. Na bardziej szczegółowym poziomie jest to zwykle udział konkretnych aplikacji w całkowitym wykorzystaniu urządzenia.
\bigskip

W tym rozdziale wymienione i opisane zostały przykłady rozwiązań z każdej z powyższych kategorii z wyszczególnieniem funkcji które oferują w zakresie monitorowania aplikacji mobilnych i ich użytkowników.
