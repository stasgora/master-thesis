Na rynku dostępnych jest wiele rozwiązań oferujących usługi~w~zakresie monitorowania aplikacji, urządzeń~i~ich użytkowników. Niektóre~z~nich~są~wbudowane~lub~oferowane razem~z~systemami operacyjnymi, inne wymagają integracji~z~aplikacją docelową~przez~jej twórców.

\paragraph{Dane~o~użytkowaniu pojedynczej aplikacji} 
Ich celem jest zebranie jak największej ilości przydatnych dla twórców aplikacji informacji. Dotyczą one ogólnego wykorzystania aplikacji oraz wykonywanych~w~niej czynności. Często zbierane~są~też dane~o~użytkownikach, ich nawykach, lokalizacji~i~urządzeniach. Tego typu rozwiązania zwykle przyjmują formę usług dających dostęp~do~zebranych danych poprzez serwis internetowy. 

\paragraph{Dane~z~czujników~w~urządzeniach} 
Do tej kategorii możemy zaliczyć wszystkie aplikacje zbierające~i~wykorzystujące dane~z~licznych czujników znajdujących się~w~smartfonach. Jednym~z~popularnych przykładów zastosowania tego typu danych jest śledzenie ćwiczeń~i~aktywności sportowych.

\paragraph{Użytkowanie urządzenia~i~aplikacji} 
Rozwiązania tego typu często zintegrowane~są~z~systemem operacyjnym urządzenia,~na~którym działają. Zbierają one statystyczne dane~o~użyciu telefonu, takie jak czas bezczynności, długość okresów,~w~których zapalony jest ekran oraz udział konkretnych aplikacji~w~całkowitym wykorzystaniu urządzenia.
\bigskip

W tym rozdziale wymienione~i~opisane zostały przykłady rozwiązań~z~każdej~z~powyższych kategorii~z~wyszczególnieniem funkcji, które oferują~w~zakresie monitorowania aplikacji mobilnych~i~ich użytkowników.
