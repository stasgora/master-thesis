\section{Podejścia~do~prywatności}
Zagadnienie prywatności jest drażliwe~i~może łatwo prowadzić~do~kontrowersji. Ten aspekt powinien być szczególnie dobrze przemyślany~przez~firmy tworzące produkty zbierające dane użytkowników. Mimo tego nawet~w~ograniczonym zbiorze wymienionych przykładów, podeście~i~dbałość~o~prywatność użytkowników diametralnie się różni. Głównym czynnikiem mającym~na~to~wpływ jest polityka firmy tworzącej dane rozwiązanie.

\paragraph{Google}
Jednym~z~głównych źródeł dochodu dla Google~są~wyświetlane reklamy. Aby skutecznie profilować zainteresowania użytkowników, konieczne jest zbieranie potencjalnie wrażliwych danych. Z~tego powodu wiele~z~produktów firmy, włączając \nameref{sec:ga}, zostało oskarżonych~lub~przyłapanych~na~kolekcjonowaniu danych użytkowników~w~celu ich profilowania. Aplikacja \nameref{sec:youtube} także domyślnie zapisuje wyświetlane filmy~w~celu wyświetlania spersonalizowanych propozycji~i~reklam.

\paragraph{Apple}
Firma Apple~nie~zarabia~na~danych swoich użytkowników. Dzięki temu znajduje się~w~pozycji,~w~której może zyskiwać zaufanie klientów poprzez publicznie komunikowane dbanie~o~prywatność ich danych. Urządzenia Apple~są~znane~z~posiadania jednych~z~najlepszych zabezpieczeń~na~rynku. Niedawno firma wprowadziła obowiązek pytania użytkownika,~czy~zgadza się~na~śledzenie~przez~każdą aplikację~z~osobna \cite{Apple_Ad_Transparency}. Bez~takiej zgody aplikacja~nie~ma~dostępu~do~identyfikatora używanego~do~wyświetlania profilowanych reklam oraz~nie~może zbierać~i~zapisywać danych~o~użytkowniku.

\paragraph{Matomo}
Według twórców jedną~z~głównych zalet \nameref{sec:matomo}~w~porównaniu~do~rozwiązania Google jest dbałość~o~prywatność użytkowników~i~ochrona ich danych. Podczas gdy Google używa zebranych danych~w~celu profilowania użytkowników~i~wyświetlania~im~reklam, klienci platformy Matomo pozostają wyłącznymi właścicielami zbieranych~na~swoich witrynach danych \cite{Matomo_Data}. Oferowana jest też możliwość całkowitego uniezależnienia~i~decentralizacji poprzez uruchomienie własnej kopii serwera zbierającego dane.

\paragraph{Strava}
Po jej publikacji globalna mapa cieplna aktywności zarejestrowanych~przez~aplikację \nameref{sec:strava} stała się powodem poważnej kontrowersji opartej~na~podłożu prywatności \cite{Strava_Military_Bases}. Udostępnione publicznie trasy przebyte~przez~żołnierzy~na~służbie zostały włączone~do~mapy cieplnej, powodując publiczne udostępnienie domniemanych lokalizacji tajnych baz wojskowych (przykład jest widoczny~na~rysunku \ref{fig:strava_military_base}). Student bezpieczeństwa międzynarodowego, który przypadkowo odkrył ten fakt, postanowił~go~upublicznić, aby doprowadzić~do~poprawy działania serwisu, jednak raz opublikowanych danych~nie~da~się~z~powrotem utajnić.

\bigskip
\img[\footnotemark]{\chapterPath/strava_military_base.jpg}{Ruchy żołnierzy~na~terenie amerykańskiej bazy lotniczej Bagram~w~Afganistanie}{strava_military_base}{.8}

\footnotetext{\url{https://www.bbc.com/news/technology-42853072}}
