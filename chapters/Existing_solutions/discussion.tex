\section{Porównanie i dyskusja}

\subsection{Podejście do prywatności}
Zagadnienie prywatności jest drażliwe i może łatwo prowadzić do kontrowersji. Ten aspekt powinien być szczególnie dobrze przemyślany przez firmy tworzące produkty zbierające dane ich użytkowników. Mimo to nawet w ograniczonym zbiorze wymienionych przykładów ich podeście i dbałość o prywatność użytkowników diametralnie się różni. Głównym czynnikiem mającym na to wpływ jest polityka firmy tworzącej dane rozwiązanie.

\paragraph{Google}
Jednym z głównych źródeł dochodu dla Google są profilowane reklamy. Aby skutecznie profilować zainteresowania użytkowników konieczne jest zbieranie potencjalnie wrażliwych danych. Z tego powodu wiele z produktów firmy, włączając \nameref{sec:ga}, zostało oskarżonych lub przyłapanych na kolekcjonowaniu danych użytkowników w celu ich profilowania. Aplikacja \nameref{sec:youtube} także domyślnie zapisuje wyświetlane filmy w celu wyświetlania spersonalizowanych propozycji i reklam.

\paragraph{Apple}
Firma Apple nie zarabia na danych swoich użytkowników. Dzięki temu znajduje się w pozycji w której może zyskiwać zaufanie klientów poprzez dbanie o prywatność ich danych. 

\paragraph{Matomo}

\paragraph{Strava}
Po jej publikacji, globalna mapa cieplna aktywności zarejestrowanych przez aplikację \nameref{sec:strava} stała się powodem poważnej kontrowersji opartej na podłożu prywatności. Udostępnione publicznie trasy przebyte przez żołnierzy na służbie zostały włączone do mapy cieplnej, powodując publiczne udostępnienie domniemanych lokalizacji tajnych baz wojskowych \cite{Strava_Military_Bases}. Student bezpieczeństwa międzynarodowego który przypadkowo odkrył ten fakt postanowił go upublicznić aby doprowadzić do poprawy działania serwisu, jednak raz opublikowanych danych nie da się z powrotem utajnić.

\img[\footnotemark]{\chapterPath/strava_military_base.jpg}{Ruchy żołnierzy w amerykańskiej bazie lotniczej Bagram w Afagistanie}{strava_military_base}{.8}

\footnotetext{\url{https://www.bbc.com/news/technology-42853072}}
