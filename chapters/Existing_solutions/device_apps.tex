\section{Aplikacje monitorujące użycie urządzenia}

\subsection{Digital Wellbeing}
\label{sec:digital_wellbeing}
Aplikacja stworzona przez Google na system Android której celem jest monitorowanie i raportowanie nawyków użycia telefonu przez użytkownika a także pomoc w ich kontroli i zmianie. Narzędzie zbiera informacje o codziennym czasie użycia urządzenia z podziałem na zainstalowane aplikacje oraz inne statystyki takie jak ilość odblokowań telefonu i otrzymanych powiadomień. 
\bigskip
\img[\footnotemark]{\chapterPath/digital-wellbeing.png}{Panel aplikacji Digital Wellbeing}{digital-wellbeing}{.3}
\footnotetext{\url{https://wellbeing.google/}}

Aplikacja posiada też szereg funkcji mających dać użytkownikowi większą kontrolę nad swoimi nawykami. Na aplikacje można nakładać dzienne limity czasu użycia po których zostają one dezaktywowane. Tryb skupienia pozwala na czasowe wyciszenie wybranych aplikacji a tryb czarno-biały wyszarza cały ekran neutralizując znajdujące się na nim przyciągające uwagę elementy.

\subsection{Family Link}
Rozwiązanie komplementarne do aplikacji \nameref{sec:digital_wellbeing} pozwalające na monitorowanie i kontrolę urządzeń dzieci. Po instalacji i konfiguracji rodzic jest w stanie przeglądać aktywność dzieci z podziałem na czas spędzony w zainstalowanych aplikacjach. Ma też możliwość weryfikacji i zatwierdzenia lub odrzucenia prób pobrania nowych aplikacji przez dziecko. Wbudowane funkcje kontroli pozwalają na ustawianie dziennych limitów korzystania z telefonu oraz ręcznego blokowania dostępu do niego. Dostępna jest też funkcja dająca wgląd w aktualną lokalizację urządzenia dziecka na mapie.

\subsection{Screen Time}
Rozwiązanie firmy Apple na platformy iOS i MacOS oferujące funkcjonalności analogiczne do aplikacji \nameref{sec:digital_wellbeing} oferowanej na system Android przez Google. Jest wbudowane w system i pozwala na przeglądanie czasu swojej aktywności na danym urządzeniu z podziałem na obszary takie jak gry, produktywność i media społecznościowe. Czas użycia każdej aplikacji jest też mierzony osobno i może być kontrolowany poprzez ustawienie limitów. Istnieje też możliwość ustalania okresów czasu w których niechciane powiadomienia, sygnały i połączenia będą wyciszane. Zawartość wyświetlana na ekranie jest także monitorowana i może być zablokowana na życzenie użytkownika jeśli zawiera nieodpowiednie treści.

\bigskip
\img[\footnotemark]{\chapterPath/screen-time.jpeg}{Panel  narzędzia Screen Time}{screen-time}{.3}
\footnotetext{\url{https://macreports.com/what-do-grey-bars-mean-in-screen-time-reports/}}

\subsection{YouTube}
YouTube posiada wbudowane mechanizmy monitorujące oglądane treści. Pozwala na przeglądanie statystyk czasu spędzanego w aplikacji oraz ustawianie komunikatów wyświetlanych po określonym czasie ciągłego oglądania. Sugerują one zrobienie sobie przerwy i odpoczęcie od ekranu.

\bigskip
\img[\footnotemark]{\chapterPath/youtube.jpg}{Przypomnienie o przerwie - YouTube}{youtube-break}{.5}
\footnotetext{\url{https://wellbeing.google/get-started/focus-your-time-with-tech/}}
