\section{Aplikacje monitorujące użycie telefonu}

\subsection{Digital Wellbeing}
\label{sec:digital_wellbeing}
Aplikacja stworzona~przez~Google~na~system Android, której celem jest monitorowanie~i~raportowanie nawyków użycia telefonu~przez~użytkownika,~a~także pomoc~w~ich kontroli~i~zmianie. Narzędzie zbiera informacje~o~codziennym wykorzystaniu urządzenia~z~podziałem~na~zainstalowane aplikacje. Te~statystyki~są~wizualizowane~w~formie wykresu pokazanego~na~rysunku \ref{fig:digital-wellbeing}. Monitorowane~są~też inne statystyki takie jak liczba odblokowań telefonu~i~otrzymanych powiadomień. 
\bigskip
\img[\footnotemark]{\chapterPath/digital-wellbeing.png}{Panel aplikacji Digital Wellbeing}{digital-wellbeing}{.3}
\footnotetext{\url{https://wellbeing.google/}}

Aplikacja posiada też szereg funkcji mających dać użytkownikowi większą kontrolę~nad~swoimi nawykami. Wyczerpanie dziennych limitów czasu użycia nakładanych~na~aplikacje powoduje~ich~dezaktywację~do~końca dnia. Tryb skupienia pozwala~na~czasowe wyciszenie wybranych aplikacji~a~tryb czarno-biały wyszarza cały ekran neutralizując znajdujące się~na~nim rozpraszające elementy.

\subsection{Family Link}
\label{sec:family_link}
Rozwiązanie komplementarne~do~aplikacji \nameref{sec:digital_wellbeing} pozwalające~na~monitorowanie~i~kontrolę urządzeń dzieci. Po~instalacji~i~konfiguracji rodzic jest~w~stanie przeglądać aktywność dzieci~z~podziałem~na~czas spędzony~w~zainstalowanych aplikacjach. Ma~też możliwość weryfikacji~i~zatwierdzenia~lub~odrzucenia prób pobrania nowych aplikacji~przez~dziecko. Wbudowane funkcje kontroli pozwalają~na~ustawianie dziennych limitów korzystania~z~telefonu oraz ręcznego blokowania dostępu~do~niego. Dostępna jest też funkcja dająca wgląd~w~aktualną lokalizację urządzenia dziecka~na~mapie.

\subsection{Screen Time}
\label{sec:screen_time}
Rozwiązanie firmy Apple~na~platformy iOS~i~MacOS oferujące funkcjonalności analogiczne~do~aplikacji \nameref{sec:digital_wellbeing} oferowanej~na~system Android~przez~Google. Jest wbudowane~w~system~i~pozwala~na~podgląd swojej aktywności~na~danym urządzeniu~z~podziałem~na~obszary takie jak gry, produktywność~i~media społecznościowe (rysunek \ref{fig:screen-time}). Czas użycia każdej aplikacji jest też mierzony osobno~i~może być kontrolowany poprzez ustawienie limitów. Istnieje też możliwość ustalania okresów czasu,~w~których niechciane powiadomienia, sygnały~i~połączenia będą wyciszane. Zawartość wyświetlana~na~ekranie jest także monitorowana~i~może być zablokowana~na~życzenie użytkownika jeśli zawiera nieodpowiednie treści.

\bigskip
\img[\footnotemark]{\chapterPath/screen-time.jpeg}{Panel  narzędzia Screen Time}{screen-time}{.3}
\footnotetext{\url{https://macreports.com/what-do-grey-bars-mean-in-screen-time-reports/}}

\subsection{YouTube}
\label{sec:youtube}
YouTube posiada wbudowane mechanizmy monitorujące ilość oglądanych treści. Aplikacja pozwala~na~przeglądanie statystyk czasu spędzanego~w~aplikacji oraz ustawianie komunikatów wyświetlanych~po~określonym czasie ciągłego oglądania (rysunek \ref{fig:youtube-break}). Sugerują one zrobienie sobie przerwy~i~odpoczęcie~od~ekranu.

\bigskip
\img[\footnotemark]{\chapterPath/youtube.jpg}{Przypomnienie~o~przerwie - YouTube}{youtube-break}{.5}
\footnotetext{\url{https://wellbeing.google/get-started/focus-your-time-with-tech/}}
