\subsection{Google Analytics}
Google Analytics jest wiodącą usługą z zakresu monitorowania ruchu w aplikacjach i stronach internetowych. Jest kompleksowym rozwiązaniem oferującym wiele funkcji i narzędzi pomagających w analizie zbieranych danych.

Pierwszym środowiskiem które wspierał Google Analytics były strony internetowe jednak od tej pory wsparcie zostało rozszerzone na platformy mobilne z wydaniem ``Google Analytics for Mobile Apps''. Oprócz tego Google Analytics jest częścią platformy Firebase omówionej w kolejnej sekcji.

\paragraph{Wydarzenia}
Definiowane przez twórcę aplikacji wydarzenia powiązane z konkretnymi akcjami wykonywanymi przez użytkownika. Istnieje możliwość ich kategoryzowania oraz dołączenia do nich prostych atrybutów które dostarczą więcej informacji o kontekście w którym zdarzenie zostało powstało. Przykładami akcji które może być warto zaraportować jako wydarzenia może być: założenie konta, logowanie, pobranie oferowanej treści, dokonanie zakupu oraz wypełnienie formularza.

\paragraph{Wyświetlenia ekranów}
Szczególnym typem wydarzenia jest przejście na nowy ekran aplikacji. Google Analytics automatycznie przetwarza te informacje na statystyki czasu przebywania na poszczególnych ekranach umożliwiające zidentyfikowanie popularnych i nieużywanych części aplikacji.

\paragraph{Sesje}
Grupa interakcji złożona z wyświetleń ekranu, wydarzeń i transakcji wykonanych przez jednego użytkownika w ograniczonym czasie jest traktowana jako sesja. Służy ona do kategoryzacji innych danych zbieranych przez Google Analytics, jak na przykład wspomnianych już statystyk wyświetleń ekranów aplikacji.

\paragraph{Odbiorcy}
Zawiera wszystkie informacje zebrane na temat użytkowników aplikacji. Umożliwia kategoryzowanie odbiorców w grupy docelowe na podstawie zebranych o nich metryk, takich jak wiek, płeć, lokalizacja, posiadane urządzenie, używana wersja aplikacji, demografia i zainteresowania a także wydarzeń które wygenerowali. Google Analytics automatycznie zbiera niektóre z tych danych, w zależności od potrzeb istnieje jednak możliwość dodawania i wysyłania własnych.

\paragraph{Lejki}
Służą do analizy i wizualizacji serii kroków wykonywanych przez użytkowników składających się na ważną z biznesowego punktu widzenia akcję, taką jak dokonanie zakupu lub założenie konta. Umożliwiają łatwe zidentyfikowanie problematycznych miejsc w których użytkownicy mają problemy, wahają się lub co gorsza  zupełnie rezygnują.

\paragraph{Retencja i zaangażowanie} 
Retencja użytkowników to ilość powracających do aplikacji użytkowników w stosunku do nowych. Zaangażowanie jest mierzone jako zmiana ilości czasu i akcji które użytkownicy wykonują w aplikacji w czasie. Obie informacje dają bardzo ważny sygnał na temat jakości i atrakcyjności aplikacji.

\paragraph{Zdobywanie użytkowników}
Informacja w jaki sposób użytkownicy dowiadują się o aplikacji jest kluczowa przy planowaniu jej reklamy. Google Analytics pozwala na śledzenie źródła z którego użytkownik pobrał aplikację. Jeśli dostępna jest ona na wielu platformach i sklepach pokazuje też statystyki popularności w każdym z nich.
