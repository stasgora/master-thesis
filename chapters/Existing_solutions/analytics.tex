\subsection{Google Analytics}
\label{sec:ga}
Google Analytics jest wiodącą usługą z zakresu monitorowania ruchu w aplikacjach i na stronach internetowych. To kompleksowe rozwiązanie oferujące wiele narzędzi pomagających w analizie i wykorzystaniu zbieranych danych. Dostarcza wiedzy o wielu aspektach wykorzystania monitorowanego produktu, zaczynając od sposobów w jaki użytkownicy na niego trafiają, poprzez charakterystykę grup odbiorców i szczegółów użycia przez nich aplikacji do analizy ich zaangażowania i powodów dla których podejmują takie a nie inne decyzje.

Pierwszym środowiskiem które wspierał Google Analytics były strony internetowe jednak od tej pory wsparcie zostało rozszerzone na platformy mobilne z wydaniem ``Google Analytics for Mobile Apps''. Oprócz tego Google Analytics jest częścią platformy \nameref{sec:firebase} omówionej w kolejnej sekcji.

\paragraph{Wydarzenia}
\label{par:ga-events}
Definiowane przez twórcę aplikacji wydarzenia powiązane z konkretnymi akcjami wykonywanymi przez użytkownika. Istnieje możliwość ich kategoryzowania oraz dołączenia do nich prostych atrybutów które dostarczą więcej informacji o kontekście w którym zdarzenie zostało powstało. Przykładami akcji które może być warto zaraportować jako wydarzenia są: założenie konta, logowanie, pobranie oferowanej treści, dokonanie zakupu oraz wypełnienie formularza.

\paragraph{Wyświetlenia ekranów}
Szczególnym typem wydarzenia jest przejście na nowy ekran aplikacji. Poprzez dodanie do niego parametrów takich jak sygnatura czasowa oraz identyfikator użytkownika Google Analytics jest w stanie stworzyć statystyki czasu przebywania na poszczególnych ekranach aplikacji. Umożliwiają one łatwe zidentyfikowanie popularnych i nieużywanych części aplikacji.

\paragraph{Sesje}
Jest zdefiniowana jako grupa interakcji złożona z wyświetleń ekranu, interakcji, wydarzeń i transakcji wykonanych przez jednego użytkownika w ograniczonym czasie. Sesje są automatycznie kończone po 30 minutach bez aktywności lub o północy. Ilość sesji w kolejnych okresach czasu jest intuicyjną metryką ruchu w aplikacji. Możliwe jest także odniesienie innych metryk do sesji, na przykład sprawdzenie średniej ilości stron wyświetlonych w pojedynczej sesji.

\paragraph{Odbiorcy}
Zawiera wszystkie informacje zebrane na temat użytkowników aplikacji. Umożliwia kategoryzowanie odbiorców w grupy docelowe na podstawie zebranych o nich metryk, takich jak wiek, płeć, lokalizacja, posiadane urządzenie, używana wersja aplikacji, demografia i zainteresowania a także wydarzeń które wygenerowali. Google Analytics automatycznie zbiera niektóre z tych danych, w zależności od potrzeb istnieje jednak możliwość dodawania i wysyłania własnych.

\paragraph{Lejki}
Służą do analizy i wizualizacji serii kroków wykonywanych przez użytkowników składających się na ważną z biznesowego punktu widzenia akcję, taką jak dokonanie zakupu lub założenie konta. Umożliwiają łatwe zidentyfikowanie problematycznych miejsc w których użytkownicy mają problemy, wahają się lub co gorsza  zupełnie rezygnują.

\paragraph{Retencja i zaangażowanie} 
Retencja użytkowników to ilość powracających do aplikacji użytkowników w stosunku do nowych. Z kolei zaangażowanie jest mierzone jako zmiana ilości czasu i akcji które użytkownicy wykonują w aplikacji w czasie. Obie informacje dają bardzo ważny sygnał na temat jakości i atrakcyjności aplikacji z punktu widzenia użytkowników.

\paragraph{Zdobywanie użytkowników}
Informacja w jaki sposób użytkownicy dowiadują się o aplikacji jest kluczowa przy planowaniu jej reklamy. Google Analytics pozwala na śledzenie źródła z którego użytkownik pobrał aplikację. Jeśli dostępna jest ona na wielu platformach i sklepach pokazuje też statystyki popularności w każdym z nich.
