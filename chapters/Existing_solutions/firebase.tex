\subsection{Firebase}
\label{sec:firebase}
Firebase jest platformą firmy Google oferującą szereg usług skierowanych do twórców aplikacji mobilnych i webowych. Początkowo oferowała uwierzytelnianie użytkowników oraz hosting, jednak od tego czasu została rozszerzona o szereg innych usług z których część jest związana z monitorowaniem aplikacji i urządzeń. \nameref{sec:ga}, które zostało opisane w osobnym podrozdziale, jest zintegrowane i stanowi część pakietu usług oferowanego w ramach Firebase.

\img{\chapterPath/firebase-crashlytics.png}{Panel Crashlytics w konsoli Firebase}{firebase-crashlytics}{.9}

\paragraph{Crashlytics}
Usługa wykrywająca błędy natury programistycznej występujące w aplikacji jako następstwo akcji wykonanych przez użytkownika. Zebrane informacje związane z błędami są wysyłane i przetwarzane do użytecznej dla twórcy formy. Narzędzie oferuje możliwość powiadomienia twórców o pojawieniu się nowego błędu oraz statystyki ilości dotkniętych nim użytkowników \cite{Fb_Crashlytics}.

Powtarzające się błędy są grupowane a ich wystąpienia zostają oznaczone na osi czasu. Do każdego dołączony jest zrzut stosu zawierający kolejne wykonywane w kodzie funkcje ułatwiający znalezienie przyczyny błędu. Kolejna oś czasu zawiera {\it \nameref{par:ga-events}} które miały miejsce przed wystąpieniem błędu. Pokazuje ona otwierane strony i wykonywane akcje w kolejności występowania i daje intuicyjny pogląd tego czym w momencie błędu zajmował się użytkownik. 

\paragraph{Monitoring wydajności}
Narzędzie automatyczne zbiera metryki dotyczące wydajności działania aplikacji takie jak:
\begin{itemize}
	\item Czas uruchamiania aplikacji mierzony od momentu jej otwarcia do pełnego załadowania
	\item Długości działania na ekranie użytkownika oraz w tle
	\item Dla każdego ekranu procent wolno ładujących i zawieszających się klatek
	\item Czas trwania zapytań sieciowych
	\item Rozmiar wysyłanych i odbieranych w trakcie zapytań danych
	\item Ilość zapytań zakończonych sukcesem w stosunku do ich całkowitej ilości
\end{itemize}
\bigskip

Istnieje możliwość mierzenia dodatkowych zadań wykonywanych przez aplikację, zdefiniowanych przez jej twórcę. Domyślną metryką jest czas, można też jednak dodawać własne według swoich potrzeb. Do wszystkich pomiarów dodawane są atrybuty dotyczące urządzenia i systemu na którym były wykonywane, używanej wersji aplikacji oraz przybliżonej lokalizacji geograficznej. Możliwe jest też dodawanie własnych parametrów. Wszystkie zebrane metadane służą do kategoryzacji zebranych pomiarów umożliwiając ich  filtrowanie i ułatwiając identyfikację źródła problemu z wydajnością \cite{Fb_Pref_Monitor}.

\paragraph{Testy A/B}
Rodzaj testów polegających na porównaniu dwóch lub więcej wersji elementu interfejsu. Każda z nich jest wyświetlana innej grupie odbiorców których reakcje, zebrane w formie metryk, wskazują na efektywność każdej z wersji. 

Firebase oferuje rozwiązanie znacznie ułatwiające proces przygotowywania, przeprowadzania i analizy testów A/B. Dzięki integracji z inną usługą oferowaną w ramach pakietu Firebase, zdalną konfiguracją, twórcy aplikacji są w stanie łatwo definiować grupy odbiorców którym zostanie wyświetlony dany wariant interfejsu będący przedmiotem testu. Dostęp do szerokiej gamy narzędzi Google Analytics, w szczególności \hyperref[par:ga-funnels]{Lejków} oraz  \hyperref[par:ga-events]{Wydarzeń}, pozwala na zebranie danych na których podstawie możliwe jest wybranie lepszej z testowanych wersji. Firebase pozwala także na testowanie wariantów powiadomień mobilnych które aplikacja wysyła do użytkowników w celu znalezienia ich najefektywniejszej treści \cite{Fb_AB_Testing}.
