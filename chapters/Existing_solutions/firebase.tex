\subsection{Firebase}
\label{sec:firebase}
Firebase jest platformą firmy Google oferującą szereg usług skierowanych~do~twórców aplikacji mobilnych~i~webowych. Początkowo dostępne było uwierzytelnianie użytkowników oraz hosting, jednak~od~tego czasu platforma została rozszerzona~o~szereg innych usług~z~których część jest związana~z~monitorowaniem aplikacji~i~urządzeń. \nameref{sec:ga}, które zostało opisane~w~osobnym podrozdziale, zostało dołączone jako jedna~z~usług oferowana~w~ramach Firebase (jak widać~w~menu~na~rysunku \ref{fig:firebase-crashlytics}).

\bigskip
\img{\chapterPath/firebase-crashlytics.png}{Panel Crashlytics~w~konsoli Firebase}{firebase-crashlytics}{.9}

\paragraph{Crashlytics}
Usługa wykrywająca występujące~w~aplikacji błędy natury programistycznej. Zebrane informacje związane~z~błędami~są~wysyłane~i~prezentowane~w~użytecznej dla twórcy aplikacji formie. Narzędzie oferuje statystyki ilości dotkniętych błędem użytkowników oraz możliwość powiadomienia twórców aplikacji~o~pojawieniu się nowego problemu \cite{Fb_Crashlytics}.

Powtarzające się błędy~są~grupowane~a~ich wystąpienia zostają oznaczone~na~osi czasu. Do~każdego z nich dołączany jest pomocny w znajdowaniu przyczyny błędu zrzut stosu wywołań funkcji. Istnieje też możliwość przeglądnięcia zebranych~przez~Google Analytics \hyperref[par:ga-events]{wydarzeń} które miały miejsce bezpośrednio przed wystąpieniem błędu, takich jak otwierane strony~i~wykonywane akcje. 

\paragraph{Monitoring wydajności}
Narzędzie automatyczne zbierające metryki dotyczące wydajności działania aplikacji takie jak:
\begin{itemize}
	\item Czas uruchamiania aplikacji mierzony~od~momentu jej otwarcia~do~pełnego załadowania
	\item Długości działania~na~ekranie użytkownika oraz~w~tle
	\item Dla każdego ekranu procent wolno ładujących~i~zawieszających się klatek
	\item Czas trwania zapytań sieciowych
	\item Rozmiar wysyłanych~i~odbieranych~w~trakcie zapytań danych
	\item Ilość zapytań zakończonych sukcesem~w~stosunku~do~ich całkowitej ilości
\end{itemize}
\bigskip

Istnieje możliwość definiowania i mierzenia dodatkowych wykonywanych~przez~aplikację zadań. Domyślną metryką jest czas, można też jednak dodawać inne. Do~wszystkich pomiarów dodawane~są~atrybuty dotyczące urządzenia~i~systemu~na~którym były wykonywane, używanej wersji aplikacji oraz przybliżonej lokalizacji geograficznej. Możliwe jest też dodawanie własnych parametrów. Wszystkie zebrane metadane służą~do~kategoryzacji zebranych pomiarów umożliwiając ich  filtrowanie~i~ułatwiając identyfikację źródeł problemów~z~wydajnością \cite{Fb_Pref_Monitor}.

\paragraph{Testy A/B}
Rodzaj testów polegających~na~porównaniu dwóch~lub~więcej wersji elementu interfejsu. Każda~z~nich jest wyświetlana innej grupie odbiorców których reakcje, zebrane~w~formie metryk, wskazują~na~efektywność każdej~z~wersji. 

Firebase oferuje rozwiązanie znacznie ułatwiające proces przygotowywania, przeprowadzania~i~analizy testów A/B. Dzięki integracji~z~inną usługą oferowaną~w~ramach pakietu Firebase, zdalną konfiguracją, twórcy aplikacji~są~w~stanie łatwo definiować grupy odbiorców którym zostanie wyświetlony dany wariant interfejsu będący przedmiotem testu. Dostęp~do~szerokiej gamy narzędzi Google Analytics,~w~szczególności \hyperref[par:ga-funnels]{lejków} oraz  \hyperref[par:ga-events]{wydarzeń}, pozwala~na~zebranie danych~na~których podstawie możliwe jest wybranie lepszej~z~testowanych wersji. Firebase pozwala także~na~testowanie wariantów powiadomień mobilnych które aplikacja wysyła~do~użytkowników~w~celu znalezienia ich najefektywniejszej zawartości \cite{Fb_AB_Testing}.
