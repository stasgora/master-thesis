\section{Aplikacje przetwarzające dane~z~czujników}

\subsection{Strava}
\label{sec:strava}
Popularna usługa pozwalająca~na~śledzenie aktywności fizycznych. Jest oparta~o~moduł GPS, dzięki któremu zapisuje trasę przebytą~przez~użytkownika. Biorąc~pod~uwagę typ aktywności wybrany~przez~użytkownika, jego wagę, szybkość przemieszczania oraz różnicę elewacji aplikacja szacuje liczbę spalonych podczas treningu kalorii. Strava posiada wersję~na~inteligentne zegarki, które dzięki wyposażeniu~w~zaawansowane czujniki dostarczają aplikacji dodatkowych metryk takich jak tętno użytkownika. 

Historia przebytych tras jest zapisywana~i~może być udostępniana innym. Publiczne aktywności~są~automatycznie grupowane~ze~względu~na~lokalizację, typ~i~czas. W~2017 roku została udostępniona globalna mapa cieplna (widoczna~na~rysunku \ref{fig:strava-heatmap}) zawierająca publicznie udostępnione dane użytkowników~z~dwóch poprzednich lat \cite{Strava_Heatmap}. Według autorów~to~największy tego typu zbiór danych, zawierający 700 milionów zapisanych aktywności~o~całkowitej długości 16 miliardów kilometrów.

\bigskip
\img[\footnotemark]{\chapterPath/strava-heatmap.jpeg}{Strava - Mapa cieplna zarejestrowanych aktywności fizycznych~w~Moskwie}{strava-heatmap}{.9}
\footnotetext{\url{https://medium.com/strava-engineering/the-global-heatmap-now-6x-hotter-23fc01d301de}}

\subsection{Samsung Health}
Aplikacja firmy Samsung towarzysząca produktom takim jak zegarek Galaxy Watch oraz opaska Galaxy Fit. Przetwarza~i~magazynuje dane zbierane~przez~zawarte~w~nich czujniki. Podobnie jak \nameref{sec:strava} pozwala~na~zapisywanie przebytych tras~i~aktywności sportowych. Samsung projektuje~i~wytwarza urządzenia,~na~których działa aplikacja, dzięki czemu jest~w~stanie w pełni wykorzystać potencjał zawartych~w~nich czujników.

\paragraph{Puls} Obecność pulsometru~w~nowszych modelach linii Galaxy umożliwia mierzenie pulsu, pulsoksymetrii (wysycenia krwi tlenem) oraz poziomu stresu użytkownika. 

\paragraph{Kroki} Wbudowany~w~smartfon~lub~opaskę pedometr pozwala~na~mierzenie liczby kroków wykonanych każdego dnia, porównując~ją~do~ustawionego celu (domyślnie 10~000 kroków). Na~podstawie liczby kroków obliczany jest także przebyty dystans oraz liczba spalonych kalorii. Dane historyczne~są~widoczne~na~grafie.

\paragraph{Sen} Jeśli użytkownik~nie~zdejmuje zegarka~lub~opaski~na~noc, aplikacja jest~w~stanie monitorować jakość snu oraz wykrywać jego fazy~na~podstawie wykonywanych~przez~śpiącego ruchów. Zebrane dane pomagają dostosować harmonogram snu użytkownika~tak,~aby czuł się~on~bardziej wypoczęty.

\paragraph{Inne} Oprócz danych automatycznie zbieranych~z~czujników aplikacja służy jako dziennik mogący rejestrować ręcznie wprowadzane~przez~użytkownika dane. Monitorowanie diety wymaga wpisywania spożytych kalorii~i~składników odżywczych. Możliwe jest też wprowadzanie aktualnej wagi oraz ilości wypitej wody. 

\subsection{Sleep Cycle}
\label{sec:sleep_cycle}
Aplikacja Sleep Cycle może być zakwalifikowana jako budzik. Tym,~co~odróżnia~ją~od~innych tego typu rozwiązań, jest innowacyjne użycie podstawowych sensorów zawartych~w~smartfonach~w~celu poprawy jakości snu użytkownika. 

\paragraph{Monitorowanie snu}
Sen składa się~z~występujących~na~przemian faz snu głębokiego~i~płytkiego (w których najczęściej występują sny). Liczba wykonywanych ruchów~i~wydawanych dźwięków, włączając~w~to~oddychanie, różni się znacznie~w~zależności~od~aktualnej fazy snu. Dzięki tym właściwościom aplikacja Sleep Cycle jest~w~stanie monitorować sen~i~rozpoznawać jego fazy (rysunek \ref{fig:sleep-cycle}). Używa~do~tego jednego~z~dwóch sensorów.

\begin{itemize}
	\item {\bf Akcelerometr} W~tym trybie telefon musi leżeć~na~łóżku śpiącego~przez~całą noc. Ruchy śpiącego~są~rejestrowane poprzez uginanie się materaca,~na~którym spoczywa smartfon.
	\item {\bf Mikrofon} Telefon analizuje oddech~i~inne dźwięki wydawane~przez~śpiącego. Te~dane, jako mniej bezpośrednie~i~bardziej zróżnicowane wskaźniki snu,~są~przetwarzane~przez~sieć neuronową~w~celu interpretacji.
\end{itemize}

\noindent Po wstępnym okresie kalibracji aplikacja rozpoczyna ocenę jakości snu~na~podstawie danych~z~poprzednich nocy.

\bigskip
\img[\footnotemark]{\chapterPath/Sleep_Cycle.jpg}{Sleep Cycle - Dziennik snu~z~wykresem zarejestrowanych faz}{sleep-cycle}{.4}
\footnotetext{\url{https://www.sleepcycle.com/press/}}

\paragraph{Budzenie śpiącego}
Faza snu,~z~której zostaniemy obudzeni,~ma~duży wypływ~na~uczucie wypoczęcia~lub~jego brak. Wyrwaniu~z~fazy głębokiego snu często towarzyszy uczucie zmęczenia~i~niewyspania się. Zwykłe budziki nastawiane~na~konkretną godzinę mogą, dzwoniąc trafić~na~dowolny moment~w~cyklu faz snu, powodując ten nieprzyjemny stan. 

Dzięki rozpoznawaniu fazy snu śpiącego~w~czasie rzeczywistym Sleep Cycle rozwiązuje ten problem. Nastawiając budzik~na~konkretną godzinę, użytkownik wybiera~tu~koniec okna czasowego,~w~którym chce zostać obudzony. Domyślnie~ma~ono pół godziny, jednak~ta~wartość jest konfigurowalna. Dokładny moment zadzwonienia budzika jest automatycznie wybierany~przez~aplikację~na~podstawie aktualnego momentu cyklu faz snu,~w~którym znajduje się śpiący. Celem takiego działania jest obudzenie użytkownika~w~jak najpłytszej fazie snu,~co~poprawi jego samopoczucie~i~sprawi,~że~będzie się czuł bardziej wypoczęty~po~wstaniu.
