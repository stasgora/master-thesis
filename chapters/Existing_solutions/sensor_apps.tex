\section{Aplikacje przetwarzające dane z czujników}

\subsection{Strava}
Popularna usługa pozwalająca na śledzenie aktywności fizycznej. Działa na podstawie modułu GPS, dzięki któremu zapisuje trasę przebytą przez użytkownika. Biorąc pod uwagę typ aktywności wybrany przez użytkownika, jego wagę, szybkość przemieszczania oraz różnicę elewacji aplikacja szacuje ilość spalonych podczas treningu kalorii. Strava posiada wersję na inteligente zegarki, które dzięki wyposażeniu w dodatkowe czujniki dostarczają aplikacji dodatkowych metryk takich jak tętno. 

Historia przebytych tras jest zapisywana i może być udostępniana innym użytkownikom. Publiczne aktywności są automatycznie grupowane ze względu na lokalizację, typ i czas. W 2017 roku została udostępniona globalna mapa cieplna zawierająca publicznie udostępnione dane użytkowników z dwóch poprzednich lat \cite{Strava_Heatmap}. Według autorów to największy tego typu zbiór danych, zawierający 700 milionów zapisanych aktywności o całkowitej długości 16 miliardów kilometrów.

\img{\chapterPath/strava-heatmap.jpeg}{Strava - Mapa cieplna zarejestrowanych aktywności fizycznych w Moskwie}{strava-heatmap}{.9}

\subsection{Samsung Health}

\subsection{Google Fit}

\subsection{Sleep Cycle}
