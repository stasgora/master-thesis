\subsection{Google Analytics}
\label{sec:ga}
Google Analytics jest wiodącą usługą z zakresu monitorowania ruchu w aplikacjach i na stronach internetowych. To kompleksowe rozwiązanie oferujące wiele narzędzi wspomagających analizę i wykorzystanie zbieranych danych. Dostarcza wiedzy o wielu aspektach wykorzystania monitorowanego produktu, zaczynając od sposobów w jaki użytkownicy go odkrywają, poprzez charakterystykę grup odbiorców i szczegółów użycia przez nich systemu do analizy zaangażowania i powodów dla których podejmują takie a nie inne decyzje.

Pierwszym środowiskiem które wspierał Google Analytics były strony internetowe jednak od tej pory wsparcie zostało rozszerzone na platformy mobilne z wydaniem ``Google Analytics for Mobile Apps''. Oprócz tego Google Analytics jest częścią platformy \nameref{sec:firebase} omówionej w kolejnym podrozdziale.

\paragraph{Wydarzenia}
\label{par:ga-events}
Definiowane przez twórcę aplikacji wydarzenia powiązane z konkretnymi akcjami wykonywanymi przez użytkownika. Istnieje możliwość ich kategoryzowania oraz dołączenia do nich prostych atrybutów które dostarczą więcej informacji o kontekście w którym zdarzenie zostało powstało. Przykładami akcji które mogą być warte zaraportowania jako wydarzenia są: założenie konta, logowanie, pobranie oferowanej treści, dokonanie zakupu oraz wypełnienie formularza \cite{GA_Events}.

\paragraph{Wyświetlenia ekranów}
Szczególnym typem wydarzenia jest przejście na nowy ekran aplikacji. Poprzez dodanie do niego parametrów takich jak sygnatura czasowa oraz identyfikator użytkownika, Google Analytics jest w stanie stworzyć statystyki czasu przebywania na poszczególnych ekranach aplikacji. Umożliwiają one łatwe zidentyfikowanie popularnych i nieużywanych części aplikacji \cite{GA_Pages}.

\paragraph{Sesje}
Są zdefiniowane jako grupy interakcji złożone z wyświetleń ekranu, wydarzeń i transakcji wykonanych przez jednego użytkownika w ograniczonym czasie. Sesje są automatycznie kończone po 30 minutach braku aktywności oraz o północy. Ilość sesji w kolejnych okresach czasu jest intuicyjną miarą ruchu w aplikacji. Możliwe jest także odniesienie innych metryk do sesji, jak na przykład sprawdzenie średniej ilości stron wyświetlonych w pojedynczej sesji \cite{GA_Sessions}.

\paragraph{Odbiorcy}
Ta sekcja zawiera wszystkie informacje zebrane na temat użytkowników aplikacji. Umożliwia kategoryzowanie odbiorców w grupy docelowe na podstawie zebranych o nich danych i metryk, takich jak wygenerowane wydarzenia, wiek, płeć, lokalizacja, wykorzystywane urządzenie, używana wersja aplikacji, dane demograficzne oraz zainteresowania. Powyższe metryki są zbierane automatycznie przez Google Analytics, w zależności od potrzeb istnieje też możliwość dodawania własnych, lepiej dostosowanych do profilu działalności firmy i zawartości monitorowanego produktu \cite{GA_Audiences}.

\paragraph{Lejki}
Służą do analizy i wizualizacji serii kroków wykonywanych przez użytkowników składających się na ważną z biznesowego punktu widzenia akcję, taką jak dokonanie zakupu lub założenie konta. Umożliwiają łatwe zidentyfikowanie problematycznych miejsc w których użytkownicy mają problemy, wahają się lub co gorsza  zupełnie rezygnują z wykonywanej czynności \cite{GA_Funnels}.

\paragraph{Retencja i zaangażowanie} 
Retencja to ilość powracających do aplikacji  użytkowników w stosunku do nowych klientów odwiedzających ją po raz pierwszy. Z kolei zaangażowanie jest mierzone jako zmiana długości czasu który użytkownicy spędzają w aplikacji oraz ilości akcji które wykonują. Obie informacje dają bardzo ważny sygnał na temat jakości, atrakcyjności i użyteczności aplikacji z punktu widzenia klientów \cite{GA_Retention}.

\paragraph{Zdobywanie użytkowników}
Informacja w jaki sposób użytkownicy dowiadują się o aplikacji jest kluczowa przy planowaniu jej reklamy i rozwoju. Google Analytics pozwala na śledzenie źródła z którego użytkownik pobrał aplikację. Jeśli dostępna jest ona na wielu platformach i sklepach pokazywane są też statystyki popularności w każdym z nich \cite{GA_Aquisition}.
