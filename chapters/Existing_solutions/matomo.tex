\subsection{Matomo}
\label{sec:matomo}
Matomo jest kompleksową platformą monitorującą konkurującą z Google Analytics. Podobnie jak ona oferuje duży zestaw funkcji oraz narzędzi i wspiera zarówno środowisko webowe jak i mobilne. Matomo oferuje wiele funkcji analogicznych do opisanych usług produktu firmy Google. Moduł Odwiedzający to odpowiednik \hyperref[par:ga-audiences]{odbiorców}, podczas gdy Kohorty są analogiczne do wskaźnika \hyperref[par:ga-retention]{retencji}. Pod tą samą nazwą dostępne są lejki, wydarzenia oraz metryki. Matomo oferuje też narzędzie pozwalające na automatyczną migrację danych z konta Google Analytics.

\paragraph{Zachowanie}
Zbiorczy moduł zbierający informacje dotyczące działań użytkowników. Zawiera statystyki dotyczące ekranów serwisu takie jak ilość wyświetleń, średni czas przeglądania oraz listę miejsc z których użytkownicy trafili na daną stronę. Pozwala na śledzenie zdefiniowanych wydarzeń, popularnych ścieżek wybieranych przez użytkowników oraz wskaźników retencji.

\paragraph{Kanały marketingowe}
Pozwalają na analizę czynników wpływających na zwrot z inwestycji \ang{return on investment - ROI}. Zaliczają się do nich wyniki wyszukiwania w przeglądarkach internetowych, odnośniki na innych stronach oraz kampanie reklamowe. Każdy z tych źródeł ma przypisaną liczbę konwersji oraz uzyskany dochód. Dodatkowo możliwe jest wyświetlenie zmiany wybranych wskaźników w czasie.

\paragraph{E-commerce}
Zbiera informacje dotyczące zakupów. Prezentuje historię zysków z podziałem na kraje oraz sprzedawane produkty. Zapisuje też profile odwiedzających i klientów pozwalając na lepsze dopasowanie prezentowanej im oferty. Moduł współpracuje z wieloma popularnymi platformami typu e-commerce znacznie przyśpieszając jego konfigurację.

\paragraph{Nagrania sesji}
Zapisane w formie filmu nagrania wizyt klientów. Wszystkie wykonane przez nich akcje, takie jak kliknięcia i ruchy kursora, przewijanie i zmienianie rozmiaru ekranu oraz interakcje z interfejsem są nagrywane i zaznaczane na filmie w momencie ich występowania. Ruch myszy pozostawia ślad rysując linię na ekranie.

\paragraph{Treści wideo}
Moduł zbiera szczegółowe informacje na temat odtworzeń zamieszczonych treści multimedialnych, w szczególności filmów. Zbierane dane zawierają ilość odtworzeń, czas oglądania, a także wykres procentowy liczby widzów oglądających dany fragment filmu. Ponadto wyświetlenia są nanoszone na mapę pozwalając na analizę popularności geograficznej.
